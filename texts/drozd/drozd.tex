

\def\bu{{\scriptscriptstyle\bullet}}

\def\Hom{\mathop\mathrm{Hom}\nolimits}
\def\rad{\mathop\mathrm{rad}\nolimits}
\def\END{\mathop{\mathcal{E}\mathit{nd}}\nolimits}
\def\coh{\mathop\mathrm{Coh}\nolimits}
\def\id{\mathrm{Id}}
\def\md{\mbox{-}\mathrm{mod}}
\def\End{\mathop\mathrm{End}\nolimits}

\def\La{\Lambda}		\def\la{\lambda}

\def\mZ{\mathbb{Z}}		\def\mP{\mathbb{P}}
\def\mX{\mathbb{X}}		\def\tmX{\tilde{\mathbb{X}}}
\def\aK{\Bbbk}
\def\kO{\mathcal{O}}		\def\tkO{\tilde{\mathcal{O}}}
\def\kA{\mathcal{A}}		\def\kD{\mathcal{D}}
\def\kT{\mathcal{T}}		\def\tkT{\tilde{\mathcal{T}}}
\def\gM{\mathfrak{m}}

\def\arl{\ar@{-}}
\def\+{\oplus}





\title{Від семінару Граве до похідних категорій}
\author{Ю.~А.~Дрозд}
\address{Інститут математики НАН України, м.~Київ, Україна, \\ Університет Гарварда, м.~Кембридж, Массачусеттс}
\email{y.a.drozd@gmail.com}
\orcid{0000-0002-4766-8791}

\shortAuthorsList{Ю.~Дрозд}

\abstract{english}{This article arose from my lecture at the First Grave Readings, in which I tried to trace the path that began with D.~Grave's lectures and seminar at Kyiv University and led to research in the most modern branches of mathematics. 
Of course I chose that one a branch of numerous directions developed by Grave's students and their scientific heirs, which is close to the Kyiv School of Theory of Representations and to my own research.
The choice of material in the article is also completely subjective and it does not pretend to be historical review. 
Rather, it is the memories of a participant in the events.}

\abstract{ukrainian}{Ця стаття виникла з моєї лекції на Перших Гравевських читаннях, у якій я намагався прослідкувати шлях, що розпочався з лекцій і семінару Д.~Граве в Київському університеті й привів до досліджень у найсучасніших галузях математики. 
Звичайно, я вибрав ту галузь з численних напрямків, розвинених учнями Граве та їх науковими спадкоємцями, яка близька до Київської школи теорії зображень і до моїх власних досліджень. Вибір матеріалу у статті також цілком суб'єктивний і вона не претендує на те, щоб бути історичним оглядом. 
Скоріше, це -- спогади учасника подій.}


\keywords{group, function}
\udc{512.5}
\msc{01A72,16G50,18G80}

\maketitle

\tableofcontents

\section{Початок}
Ця стаття є спробою прослідкувати шлях, який пройшла одна гілка алгебричної школи, створеної Дмитром Граве, саме та гілка, яка відродилася у Києві, колисці всієї цієї школи. Нагадаю, що ця школа бере початок у семінарі, який проходив у Київському університеті під керівництвом Д.~Граве. 
Ось що пише про це сам Д.~Граве у своїх <<Автобіографічних записках>>~\cite{AB}: 

\begin{enumerate}[nosep, label={}, wide=0pt]
\item 
<<\emph{Я вважав, що єдине правильне розуміння університету є те, що університет має бути лабораторією науки, в якій кожен професор має бути дослідником, а студент -- починаючим ученим, і я вирішив у 1912--1914 роках здійснити мою ідею під виглядом семінару з алгебри і теорії чисел.}>>
\end{enumerate}

З семінару Граве вийшли такі відомі вчені, як Б.~Делоне, М.~Кравчук, А.~Островський, М.~Чеботарьов, О.~Шмідт. 
Усі вони відзначились видатними внесками у сучасну алгебру і теорію чисел.

Слід зазначити, що перед цим Д.~Граве провів велику роботу по підготовці підґрунтя майбутнього семінару. 
Починаючи з свого приїзду до Києва, він розпочав викладання сучасної алгебри. 
Вже у 1902--1904 роках він працює над лекціями з теорії груп, а 1908 року у Києві виходить його книга <<Теория групп>>. 
У 1909 виходить <<Элементарный курс теории чисел>>, а у 1910 та 1913 роках -- двотомна <<Арифметическая теория алгебраических величин>> (зараз її назвали б <<алгебричною теорією чисел>>). 
Починаючи з 1911 року він регулярно публікує статті з алгебри та алгебричної теорії чисел. 
І останньою його великою працею став <<Трактат з алгебричного аналізу>>. 
У 1938 році вийшли перші два томи. Другий з них майже повністю присвячений саме основам алгебричної теорії чисел, включаючи загальну теорію ідеалів та одиниць, алгоритм Вороного і теорію $p$-адичних чисел. 
Третій том, за планом Д.~Граве, мав бути присвячений конкретному розгляду квадратичних полів і бінарних квадратичних форм, рядам Діріхле з теоремою про прості числа в арифметичній прогресії, символам Гільберта і геометричній теорії чисел, зокрема, роботам Г.~Вороного й Б.~Делоне. 
На жаль Дмитро Олександрович не встиг довести цю роботу до кінця.

Дослідження в галузі алгебричної теорії чисел стали справою життя Б.~Делоне. 
Його студентська робота, удостоєна Великої Золотої медалі -- <<Связь между теорией идеалов и теорией Галуа>>, а перша опублікована робота -- <<Об определении алгебраической области при помощи конгруэнтностей>> присвячена доведенню теореми Кронекера-Вебера про те, що довільне розширення поля раціональних чисел з абелевою групою Ґалуа вкладається у якесь поле поділу кола. 
Але найвидатнішими досягненнями Б.~Делоне в цій галузі стали дослідження цілочисельних бінарних кубічних форм і, у зв'язку з ними, кубічних полів алгебричних чисел. 
Сам Борис Миколайович вважав ці роботи своїми найкращими, незважаючи на те, що надалі він вніс фундаментальний внесок у геометрію чисел та багатовимірну кристалографію. 
Підсумком цього циклу досліджень Б.~Делоне та його учня Д.~Фаддєєва стала монографія <<Теория иррациональностей третьей степени>> (1940~р.).
Цікаво, шо у вступі, коментуючи одну з теорем, Б.~Делоне згадує, що вона пов'язана з <<припущенням, яке виникло з розгляду великої таблиці дискримінантів кубічних одиниць, обчисленої у 1918 році для мене за допомогою арифмометрів студентами Київського університету>>.

\section{Цілочисельні зображення}
Той напрямок, про який я буду надалі говорити, фактично походить від цієї монографії. 
Саме, у \S15 там встановлено зв'язок між бінарними квадратичними формами $ax^3+bx^2y+cxy^2+dy^3$, де $a,b,c,d$ -- цілі числа, і \emph{кубічними кільцями}, тобто
кільцями цілих алгебричних чисел, які лежать у кубічних розширеннях поля раціональних чисел%
\footnote{\,Насправді, і тут, і нижче, де йдеться про квадратичні кільця, розглядається ширший клас кілець, а саме такі, які лежать у напівпростих алгебрах над полем раціональних чисел. 
Утім, основні результати від цього практично не залежать.}.
У 1960-ті роки Д.~Фаддєєв повернувся до цього  результату у зв'язку з \emph{теорією цілочисельних зображень}.
Цілочисельне зображення кільця (або групи) -- це гомоморфізм у кільце цілочисельних матриць (відповідно, у групу обертовних цілочисельних матриць). Інтерес до цілочисельних зображень груп виник у зв'язку з роботами Є.~Федорова і А.~Шенфліса про \emph{кристалографічні групи}, вивчення яких вимагає, зокрема, класифікації цілочисельних зображень скінченних груп. 
Класична теорема Жордана каже, що при фіксованій розмірності дана скінченна група має лише скінченну кількість неізоморфних (нееквівалентних) цілочисельних зображень. 
У 1938 році Г.~Цассенгаус дав нове доведення цієї теореми~\cite{Zas}, а його учень Ф.~Дідеріксен описав зображення циклічної групи простого порядку~\cite{Die}. 
У тій же роботі Ф.~Дідеріксен доводив, що вже циклічна група порядку $4$ має нескінченну кількість неізоморфних нерозкладних цілочисельних зображень (звичайно, необмежених розмірностей). 
Утім, останнє твердження виявилось хибним, і у 1960 році учень Д.~Фаддєєва -- А.~Ройтер довів, що ця група має лише 9 нерозкладних цілочисельних зображень~\cite{Roiter0}. 
Перед цим З.~Боревич і Д.~Фаддєєв довели, виходячи з міркувань гомологічної алгебри, що нециклічна група завжди має нескінченно багато неізоморфних нерозкладних
цілочисельних зображень~\cite{BFhom}.
У 1962--1964 роках у роботах С.~Бермана і П.~Гудівка, А.~Хеллера і І.~Райнера та А.~Джонса було встановлено, що скінченна група має лише скінченну кількість нерозкладних неізоморфних цілочисельних зображень тоді й лише тоді, коли кожна її силовська $p$-підгрупа є циклічною порядку $p$ або $p^2$~\cite{BG0,BG1,HR1,HR2,Jones}.

На початку 1960-х років виник інтерес вже до цілочисельних зображень кілець, перш за все, кілець алгебричних чисел. 
У найпростішому варіанті це -- задача опису цілочисельних матричних розв'язків алгебричних рівнянь. 
Перший крок у цьому напрямку зробили З.~Боревич і Д.~Фаддєєв.
У 1960 році вони описали цілочисельні зображення \emph{квадратичних кілець}, тобто кілець цілих алгебричних чисел, які лежать у квадратичних розширеннях поля раціональних чисел~\cite{BF1}. 
Виявилось, що всі ці зображення реалізуються в ідеалах кільця. 
Надалі цей результат узагальнили і його автори~\cite{BFcyc}, і Г.~Басс~\cite{Bass} (ці узагальнення виявились еквівалентними). 
Відчуваючи потребу у деяких загальних підґрунтях, Д.~Фаддєєв опублікував велику роботу \cite{F1}, у який виклав загальні фундаментальні поняття й факти теорії цілочисельних зображень кілець. 
У тому самому випуску з'явилась і його стаття, присвячена кубічним кільцям~\cite{Fcub}. 
У ній викладено зв'язок кубічних кілець та кубічних форм, а також встановлено важливий результат, що якщо $I$ -- ідеал кубічного кільця $A$, то для довільного простого $p$ локалізація $I_p$ ізоморфна або $A_p$, або $A^*_p$, де $A^*=\Hom_\mZ(A,\mZ)$ -- дуальний ідеал. Встановлено також, що завжди $I^2_p\simeq A'_p$ для деякого надкільця $A'\supset A$%
\footnote{\,Пізніше мені вдалося узагальнити ці результати в роботі~\cite{ideals}.}.
На школі 1964~р. в Ужгороді, де Д.~Фаддєєв викладав ці результати, він запропонував розглянути задачу про зображення кубічних кілець, зокрема, знайти критерій того, що кубічне кільце має лише скінченну кількість неізоморфних нерозкладних цілочисельних зображень (як зараз кажуть, критерій \emph{зображувальної скінченності}). 
Те, що, на відміну від квадратичних кілець, це не може бути завжди так, було більш-менш зрозуміло. 
У Києві саме на початку 1960-х років під керівництвом А.~Ройтера сформувалася група молодих математиків, які розпочали активні дослідження з теорії цілочисельних зображень. 
До неї увішли В.~Кириченко, С.~Кругляк, Л.~Назарова і я.
У майбутньому саме з цієї групи вийшла Київська школа теорії зображень. 
У перебігу розпочатих досліджень мені вдалося знайти критерій зображувальної скінченності для кубічних кілець~\cite{cubic}.
\begin{theorem}
Нехай $\La$ -- кубічне кільце, $M$ -- його максимальне надкільце, $M/\La$ -- пряма сума циклічних груп порядків $m$ і $n$ де $m\mid n$.
$\La$ має скінченну кількість нерозкладних зображень тоді й лише тоді, коли $m$ вільне від квадратів.
\end{theorem}
Більш того, порівняно швидко А.~Ройтер і я узагальнили цей результат і отримали критерій зображувальної скінченності для довільних комутативних кілець~\cite{DR}.
\begin{theorem}
Нехай $M$ -- максимальне надкільце $\La$, $I=M/\La$, $I'=\rad I$. 
$\La$ має скінченну кількість нерозкладних зображень тоді й лише тоді, коли  $I$ має $2$ твірних, а $I'$ -- циклічний $\La$-модуль.
\end{theorem}
Рівносильний критерій, хоча й трохи в інших термінах, отримав одночасно шведський математик Г.~Якобінскі~\cite{Jac-fin}.
Паралельно були розроблені нові методи, що брали початок у згаданих вище роботах З.~Боревіча і Д.~Фаддєєва та  Г.~Басса, і створена теорія бассових і квазабассових кілець~\cite{DKR,qbass}. 
Це дало змогу В.~Кириченку і мені узагальнити критерій зображувальної скінченності на широкий клас некомутативних кілець~\cite{primary}. 
Такі були здобутки Київської школи у теорії цілочисельних зображень на першому етапі її розвитку.


\section{Ручні та дикі особливості кривих}
На деякий час інтерес до теорії цілочисельних зображень, включаючи й Київську школу, дещо спав, і основні дослідження перемістилися до зображень скінченновимірних алгебр. 
Це було ініційовано блискучою роботою А.~Ройтера, в якій він довів одну з класичних гіпотез Брауера-Тролла~\cite{Roiter-Br}, а також роботами П.~Ґабріеля~\cite{quiver} та Л.~Назарової і А.~Ройтера \cite{posets}, з яких розпочалося дослідження зображень сагайдаків і частково впорядкованих множин. 
Відродився інтерес до теорії цілочисельних зображень значною мірою завдяки роботі {Ґ.-М.~Ґройеля} і Г.~Кноррера~\cite{Gr-Kn}, в якій було встановлено несподівані зв'язки цієї теорії з \emph{теорією особливостей}. 
Саме, у ній були розглянуті \emph{особливості алгебричних кривих}, або, що те саме, комутативні алгебри $A$ над кільцем формальних рядів $R=\aK[[t]]$, і вирішувалось питання, коли така алгебра має лише скінченну кількість неізоморфних нерозкладних модулів Коена-Маколея. У даному випадку це -- те саме, що зображення алгебри $A$ над кільцем $R$. 
Звичайно, відповідь дано у роботі ~\cite{DR}, де розглянуто загальну ситуацію одновимірних нетерових кілець, але автори не були з нею знайомі. 
Натомість, користуючись технікою, близькою до роботи Г.~Якобінського~\cite{Jac-fin}, вони отримали критерій і, що особливо важливо, пов'язали його з класифікацією особливостей, розробленою В.~Арнольдом~\cite{Arn-critical}. 
Нагадаю, що В.~Арнольд увів поняття \emph{простої особливості} як такої, в достатньо малому околі якої є лише скінченна кількість нееквівалентних особливостей. Прості особливості виявились пов'язаними зі \emph{схемами Динкіна} $A_n,D_n,E_6,E_7,E_8$, добре відомими з теорії груп Лі. Це -- особливості плоских кривих, заданих рівняннями
\begin{align*}
A_n&  : \ x^2=y^{n+1},\\
D_n&  : \ x^2y=y^{n-1}\ \ (n\ge4),\\
E_6&  : \ x^3=y^4,\\
E_7&  : \ x^3=xy^3,\\
E_8&  : \ x^3=y^5,
\end{align*}
Результат Ґройеля і Кноррера був таким:
\begin{theorem}
Особливість алгебричної кривої має скінченну кількість неізоморфних нерозкладних модулів Коена-Маколея  тоді й лише тоді, коли вона \emph{домінує} просту особливість (тобто є її надкільцем).
\end{theorem}

На цей час у теорії зображень скінченновимірних алгебр з'ясувалося, що зображувально нескінченні алгебри розділяються на два істотно різні типи: \emph{ручні} й \emph{дикі}. 
Ручні алгебри -- це такі, у яких нерозкладні зображення кожної даної розмірності утворюють скінченну кількість однопараметричних сімей. 
Дикі алгебри можуть бути охарактеризовані двома способами (які, втім, виявились рівносильними)%
\footnote{\,Формальні означення можна знайти в огляді~\cite{canada}.}:
\begin{itemize}[leftmargin=*, itemsep=1ex]
\item (геометрично) як такі, для яких існують сім'ї неізоморфних нерозкладних зображень, що залежать від довільної кількості параметрів;
\item (алгебрично) як такі, що класифікація їх зображень містить у собі класифікацію зображень довільної скінченнопородженої алгебри.
\end{itemize}
У роботі~\cite{tw0} я довів, що будь-яка скінченновимірна алгебра над алгебрично замкненим полем є або ручною, або дикою.

Поняття ручних і диких природно переносяться на інші класифікаційні задачі, зокрема, у теорію модулів Коена-Маколея.
У 1990~р. під час семінару, організованого у Білефельді К.~Рінгелем, Ґ.-М.~Ґройель висунув деяку гіпотезу про зв'язок особливостей, ручних у цьому сенсі, з класифікацією Арнольда так званих \emph{унімодальних особливостей}. 
Утім, на цей час у мене вже був контрприклад до його гіпотези, про що я йому й розповів. 
Ми вирішили детальніше вивчити цю проблему, і нам вдалося знайти критерій ручності для особливостей кривих і встановити його зв'язок з класифікацією Арнольда. 
А саме, у цій класифікації особливу роль відіграють <<\emph{серійні однопараметричні особливості}>> або \emph{особливості типу $T_{pq}$}, тобто особливості плоских кривих, заданих рівняннми
\[
T_{pq}\ : \  x^p+y^q+\la x^2y^2=0, \quad\text{де }\ \frac{1}{p}+\frac{1}{q}\le\frac12,
\]
тут $\la\in\aK$ -- деякий параметр. 
При $(p,q)\in\{(4,4),(3,6)\}$ цей параметр ролі не відіграє: всі його значення, крім забороненого $\la=0$, дають ізоморфні особливості. 
При $(p,q)\in\{(4,4),(3,3)\}$ це вже не так. 
Забороненими тут є значення $\la=\pm2$ при $(p,q)=(4,4)$ і три значення, для яких $4\la^3=-27$ при $(p,q)=(3,6)$, а різні значення $\la$ дають неізоморфні особливості. 
У роботі~\cite{DG1993} доведено таку теорему.
\begin{theorem}
Особливість алгебричної кривої є ручною тоді й лише тоді, коли вона домінує якусь особливість типу $T_{pq}$.
\end{theorem}
На жаль, на відміну від робіт~\cite{Jac-fin,DR,Gr-Kn}, тут не було отримано опис модулів -- істотним інструментом було використання деформацій, що дало можливість отримати критерій, але без явної конструкції модулів.  
Втім, повний опис і досі отриманий лише для особливостей типу $T_{44}$~\cite{Diet,T44-1,T44-2}.


\section{Векторні розшарування й особливості поверхонь}
Наступним кроком мав стати розгляд \emph{поверхневих особливостей}. 
На цей час було відомо, що зображувально скінченні поверхневі особливості -- це так звані \emph{фактор-особливості} (\emph{quotient singularities}), тобто алгебри інваріантів $\aK[[x,y]]^G$ при дії скінченної групи $G$ на кільці формальних степеневих рядів від двох змінних \cite{Aus-quot,Esnault}. 
Для дослідження ручності підґрунтям стала робота К.~Кана~\cite{Kahn}, в якій встановлено зв'язок модулів Коена-Маколея над поверхневою особливістю з \emph{векторними розшаруваннями} над \emph{виключною кривою} розв'язання цієї особливості. 
Остання є проєктивною кривою, тож природно стало питання про класифікацію векторних розшарувань над проєктивними кривими. 
Така класифікація була відома лише для проєктивної прямої $\mP^1$, де вона найпростіша: нерозкладними є лише лінійні розшарування, а також для \emph{еліптичних кривих} (неособливих кривих роду $1$, або, що те саме, неособливих плоских кубік)~\cite{Atiyah}. 
У останньому випадку нерозкладні розшарування даного рангу й степеня утворюють сім'ю, параметризовану точками кривої. 
Отже, це -- ручний випадок. 
З іншого боку, доволі нескладно встановити, що класифікація векторних розшарувань над кривою роду $g>1$ вже є дикою. 
Про випадок кривих з особливостями (а такі найчастіше виникають як виключні криві розв'язань) не було відомо нічого.

Ми з Ґ.-М.~Ґройелем встановили такий результат~\cite{DG2001}.
\begin{theorem}
Зв'язна проєктивна крива є ручною відносно класифікації векторних розшарувань тоді й лише тоді, коли вона є або проєктивною прямою, або еліптичною кривою, або ланцюгом чи циклом проєктивних прямих з трансверсальними перетинами.
Всі інші проєктивні криві є дикими відносно класифікації векторних розшарувань.
\end{theorem}

Останній випадок означає, що всі незвідні компоненти кривої -- це проєктивні прямі, всі їх перетини є трансверсальними і, при деякій нумерації прямих, кожна перетинає наступну і, у випадку циклу, остання перетинає першу. 
У випадку циклу з $5$ компонентами це виглядає так:
\[
\xymatrix@R=1.5em@C=2.5em{ & {} \arl[dddl] & {}  \arl[dddr]\\
                {} \arl[rrr] &&& {} \\
                {} \arl[ddrr] &&& {} \arl[ddll] \\
                {} &&& {} \\
                & {} & {} }
\]
Якшо така крива має одну компоненту, це -- нодальна кубіка, якщо дві -- це перетин прямої з колом. 
При цьому у випадку ланцюга (рід такої кривої дорівнює $0$) всі нерозкладні розшарування є лінійними, а у випадку циклу (його рід дорівнює $1$), при фіксованому рангу і фіксованих степенях на кожній компоненті, нерозкладні розшарування утворюють скінченну кількість сімей, кожна з яких параметризована точками проєктивної прямої. 
При цьому вдалося й дати повний опис відповідних векторних розшарувань. 
Як і у випадку особливостей кривих, цей опис зводився до так званих \emph{в'язок ланцюгів}, досліджених у роботах Л.~Назарової і А.~Ройтера~\cite{NR} та В.~Бондаренка~\cite{Bond}.

Разом з роботою Кана це дозволило вирішити питання про ручні й дикі поверхневі особливості для важливого класу так званих \emph{мінімально еліптичних особливостей}~\cite{Lau-min}. 
Це горенштейнові поверхневі особливості роду $1$. 
Серед них виділяються
\begin{itemize}[leftmargin=*, itemsep=1ex]
\item  \emph{прості еліптичні особливості} -- такі, що виключна крива є еліп\-тичною кривою;
\item  \emph{каспідальні особливості} -- такі, що виключна крива є циклом проєктивних прямих.
\end{itemize}

У роботі~\cite{DGK} встановлено такий результат.
\begin{theorem}
Мінімально еліптична поверхнева особливість є ручною тоді й лише тоді, коли вона є або \emph{простою еліптичною}, або
\emph{каспідальною}. В усіх інших випадках вона є дикою.
\end{theorem}
Зокрема, якщо це -- особливість поверхні у тривимірному просторі, то це -- особливість типу $T_{p,q,r}$, тобто задається рівнянням
\[
T_{p,q,r}\ : \ x^p+y^q+z^r+\la xyz=0,\ \text{\ де }\ \frac{1}{p}+\frac{1}{q}+\frac{1}{r}\le1,\ p\le q\le r,\ \la\notin\{0,1\}.
\]
Якщо $ \frac{1}{p}+\frac{1}{q}+\frac{1}{r}=1$, це проста еліптична особливість, якщо нерівність строга -- каспідальна особливість. 
У останньому випадку всі значення $\la$ (включаючи $\la=1$) визначають ізоморфні особливості.

Для ручних поверхневих особливостей також було дано класифікацію модулів (для простих еліптичних особливостей це зробив ще К. Кан у тій самій роботі~\cite{Kahn}).

\section{Похідні категорії}
У роботі~\cite{DG2001} ми з Ґ.-М.~Ґройелем звернули увагу на те, що опис векторних розшарувань над циклами проєктивних прямих
має багато спільного з описом зображень деяких скінченновимірних алгебр. У найпростішому випадку нодальної кубіки -- це алгебра,
задана сагайдaком зі співвідношеннями
\begin{equation}\label{e5}
\xymatrix@=1em{ \bu \ar@/^/[rr]^{a_1} \ar@/_/[rr]_{a_2} && \bu \ar@/^/[rr]^{b_1} \ar@/_/[rr]_{b_2} && \bu} \quad b_1a_1=b_2a_2=0.
\end{equation}
Звичайно, ми знали про класичну роботу А.~Бейлінсона~\cite{Beilinson}, в якій було встановлено зв'язки між векторними розшаруваннями над проєктивним простором $\mP^n$ та зображеннями деяких скінченновимірних алгебр, зокрема, між $\mP^1$ та \emph{сагайдаком Кронекера} $ \xymatrix@=1em{ \bu \ar@/^/[rr] \ar@/_/[rr] && \bu  }$. 
Але основним у цій роботі було доведення того, що еквівалентними є \emph{похідні категорії} когерентних пучків над $\mP^n$ та модулів над відповідною алгеброю.
Така еквівалентність задається функтором $\Hom(\kT,-)$, де $\kT=\bigoplus_{k=0}^n\kO_{\mP^n}(k)$ -- так званий \emph{тілтінг-пучок}.
У випадку нодальної кубіки й алгебри \eqref{e5} така еквівалентність була напевно неможливою, оскільки глобальна гомологічна розмірність цієї алгебри скінченна (дорівнює $2$), а глобальна гомологічна розмірність нодальної кубіки нескінченна.

Роз'яснення цієї загадки було дано у роботі І.~Бурбана і моїй~\cite{BDder}, яка дала початок циклу робіт, присвячених похідним категоріям.
У цій роботі було побудоване нове \emph{категорне розв'язання} особливостей нодальної кривої. 
На той час конструкція категорних розв'язань у різних виглядах вже була відома, але всі вони мали один недолік. 
Побудовані категорії залишалися деякими досить складними категорними конструкціями, з якими було досить важко ефективно працювати. 
Наше розв'язання будувалось у <<внутрішніх рамках>>  алгебричної геометрії, точніше, \emph{некомутативної алгебричної геометрії}.
Саме, ми розглянули звичайне геометричне розв'язання нодальної кривої $X$ -- домінантне раціональне відображення $\pi:\mP^1\to X$. 
Воно є ізоморфізмом поза особливою точкою, а остання має два пообрази. 
Позначимо через $\kO$ пучок регулярних функцій на $X$, через $\tkO$ прямий образ при відображенні $\pi$ пучка регулярних функцій на $\mP^1$ і розглянемо пучок ендоморфізмів $\kA=\END_\kO(\kO\+\tkO)$. 
Пара $\mX=(X,\kA)$ є прикладом \emph{некомутативної алгебричної кривої}.
Для неї, як і для звичайних алгебричних кривих можна розглядати категорію когерентних пучків $\coh\mX$ та її похідну категорію $\kD(\coh\mX)$. 
Для кривої $\mX$ гомологічна розмірність вже скінченна (дорівнює $2$), тобто в цьому розумінні вона неособлива.
З іншого боку, категорії $\coh X$ та $\coh\mX$ пов'язані так званою \emph{діаогамою прикріплення} (\emph{recollement})
\[
\xymatrix{ \ker F \ar[rr]|I && \coh\mX \ar@/_/[ll]_{I^*} \ar@/^/[ll]^{I^!} \ar[rr]|F && \coh X \ar@/_/[ll]_{F^*} \ar@/^/[ll]^{F^!}      },
\]
в якій функтор $F$ є точним і сюр'єктивним, $F^*$ та $F^!$ -- це його лівий і правий спряжені, $I$ -- занурення,  $I^*$ та $I^!$ --
це його лівий і правий спряжені, причому $FF^*\simeq FF^!\simeq \id$. Отже, некомутативну криву $\mX$ можна розглядати як
розв'язання кривої $X$. Більш того, категорія $\ker F$ у цьому випадку еквівалента категорії  модулів над $\aK\times\aK$, де $\aK$
-- основне поле. Отже, категорії $\coh X$ та $\coh\mX$ мало відрізняються. Нарешті, у похідній категорії $\kD(\coh\mX)$ є
\emph{тілтінг-комплекс} $\kT^+=\tkT\+(\kO/\gM)[-1]$, де $\gM$ -- максимальний ідеал, що відповідає особливій точці.  Згідно загальної
теорії, тоді $\kD(\coh\mX)\simeq\kD(\La\md)$, де $\La$ -- дуальна алгебра до алгебри ендоморфізмів $\End(\kT^+)$, яка якраз і є
алгеброю \eqref{e5}.

Надалі ми з І.~Бурбаном і В.~Гавраном узагальнили цю конструкцію для довільних (у тому числі, й некомутативних) кривих~\cite{BDG}. 
Зокрема, для кожної некомутативної кривої $\mX$ ми побудували криву $\tmX$, яка дає категорне розв'язання особливостей кривої $\mX$, а у раціональному випадку також скінченновимірну алгебру $\La$ таку, що $\kD(\coh\tmX)\simeq\kD(\La\md)$.

Одночасно ми почали вивчати будову похідних категорій, зокрема, питання про їх ручність чи дикість. 
Ще у 1990 році я встановив, які саме локальні кільця (можливо, некомутативні) є ручними відносно класифікації всіх скінченнопороджених модулів~\cite{pure}.
Серед звичайних особливостей ними виявилися лише <<прості вузли>> (трансверсальні перетини). 
Ми назвали такі кільця (некомутативні аналоги трансверсальних перетинів) <<нодальними особливостями>> і дали для них повний опис похідних категорій~\cite{BDnodal}. 
Для нодальних кривих (у тому числі й некомутативних), тобто таких, у яких всі особливості є нодальними, Д.~Волошин і я знайшли критерій ручності відносно класифікації векторних розшарувань і в ручному випадку описали такі розшарування та похідну категорію категорії когерентних пучків~\cite{DVol1,DVol2}. 
Слід зауважити, що, як виявилось, некомутативні нодальні криві відіграють істотну роль у так званій \emph{теорії дзеркальної симетрії} (дивись, наприклад, роботу~\cite{LP}).

Ось як з семінару, організованому в нашому університеті Д.~Граве виросли дослідження, які врешті решт привели до нових результатів у такій <<модерній>> області.

