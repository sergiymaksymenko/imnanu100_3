\newcommand\DSp{\mathcal{D}}
\newcommand\Ccal{\mathcal{C}}
\newcommand\Hcal{\mathcal{H}}
\newcommand\bC{\mathbb{C}}
\newcommand\bN{\mathbb{N}}
\newcommand\bR{\mathbb{R}}
\newcommand\bZ{\mathbb{Z}}
\newcommand\bfE{\mathbf{E}}
\newcommand\bfP{\mathbf{P}}
\newcommand\ortp[2]{:\!\langle#1,#2\rangle\!:\ }




\author{М.~О.~Качановський}
\address{Інститут математики НАН України, м.~Київ}
\email{nkachano@gmail.com}
\orcid{0000-0001-7354-5384}


\title[Формули типу Кларка"=Окона]{Формули типу Кларка"=Окона \\ на просторах регулярних основних \\ і узагальнених функцій \\ в аналізі білого шуму Леві}

%------ abstracts
\abstract{english}{In the classical Gaussian analysis the Clark-Ocone formula can be written in the form
$$
F=\bfE{F}+\int\bfE\big(\partial_t F|_{\mathcal F_t}\big)dW_t,
$$
where a function (a random variable) $F$ is square integrable with
respect to the Gaussian measure and differentiable by Hida;
$\bfE$ denotes the expectation;
$\bfE\big(\circ|_{\mathcal F_t}\big)$ -- the conditional expectation with respect to the full
$\sigma$-algebra $\mathcal F_t$ that is generated by the Wiener process $W$ up to the point of
time $t$;
$\partial_\cdot F$ is the Hida derivative of $F$;
$\int\circ (t)dW_t$ denotes the It\^o stochastic integral with respect to the Wiener process.
This formula has many applications, in particular, in the stochastic analysis and in the
financial mathematics.

In this paper we generalize the Clark-Ocone formula to spaces of regular test and
generalized functions of the L\'evy white noise analysis. More exactly, we obtain different
Clark-Ocone type formulas on the above-mentioned spaces, study the properties of the integrands
in these formulas, establish the conditions under which a Clark-Ocone type formula takes
a classical form, etc. In particular, we show that the restrictive condition of differentiability
by Hida for a random variable is not really significant.}

\abstract{ukrainian}{
У класичному гауссівському аналізі формулу Кларка"=Окона можна записати у вигляді
$$
F=\bfE{F}+\int\bfE\big(\partial_t F|_{\mathcal F_t}\big)dW_t,
$$
де функція (випадкова величина) $F$ є квадратично інтегровною за гауссівською мірою та
диференційовною за Хідою; $\bfE$ позначає математичне сподівання;
$\bfE\big(\circ|_{\mathcal F_t}\big)$ -- умовне математичне сподівання відносно повної
$\sigma$"=алгебри $\mathcal F_t$, породженої вінерівським процесом $W$ до моменту часу $t$;
$\partial_\cdot F$ -- похідна Хіди $F$;
$\int\circ (t)dW_t$ позначає стохастичний інтеграл Іто за вінерівським процесом. Ця формула
має багато застосувань, зокрема, у стохастичному аналізі та у фінансовій математиці.

В цій статті ми узагальнюємо формулу Кларка"=Окона на простори регулярних основних і узагальнених
функцій в аналізі білого шуму Леві. Точніше, ми отримуємо різні формули типу Кларка"=Окона
на вищезгаданих просторах, вивчаємо властивості підінтегральних функцій у цих формулах,
встановлюємо умови, за яких формула типу Кларка"=Окона приймає класичний вигляд, тощо. Зокрема, ми
показуємо, що обмежувальна умова диференційовності за Хідою для випадкової величини не є
суттєвою.}


\keywords{процес Леві; властивість хаотичного розкладу; розширений стохастичний інтеграл; стохастична похідна; формула Кларка"=Окона}
\udc{517.98}
\msc{46F05, 46F25, 60H40, 60G51, 60H05}
%% DOI of the current paper
% \doi{}

\maketitle


%--------------- parer text
\section*{Вступ}

Позначимо через $\DSp$ простір Шварца, що складається з усіх дійснозначних нескінченно"=диференційовних функцій на $\bR_+:=[0,+\infty)$ з компактними носіями.
Добре відомо (напр.,~\cite{BUS90}), що $\DSp$ можна наділити топологією проективної границі, породженою сім'єю соболевських просторів.
Нехай $\DSp'$ -- множина лінійних неперервних функціоналів на $\DSp$.
Відзначимо, що $\DSp'$ та $\DSp$ є негативним та позитивним просторами ланцюжка
\begin{equation}\label{e0_1}
\DSp'\supset L^2(\bR_+)\supset\DSp,
\end{equation}
де $L^2(\bR_+)$ -- простір (класів) дійснозначних функцій на $\bR_+$, квадратично інтегровних за мірою Лебега (напр.,~\cite{BUS90}).

Позначимо через $\langle\cdot,\cdot\rangle$ дуальне спарювання між елементами $\DSp'$ та $\DSp$, породжене скалярним добутком у $L^2(\bR_+)$; через нижній індекс $\bC$ будемо позначати комплексифікації лінійних топологічних просторів (наприклад, елементами $\DSp_{\bC}$ є $a+bi$, $a,b\in\DSp$); через $\Ccal(\DSp')$ -- циліндричну $\sigma$"=алгебру на $\DSp'$.

Нехай $\gamma$ -- стандартна гауссівська міра на $(\DSp',\Ccal(\DSp'))$ ($\Ccal(\DSp')$ вважаємо поповненою відносно $\gamma$), тобто ймовірнісна ($\gamma(\DSp')=1$) міра з перетворенням Лапласа
\begin{equation*}
    l_\gamma(\lambda):=\int_{\DSp'}e^{\langle x,\lambda\rangle}\gamma (dx)
                      = e^{\langle\lambda,\lambda\rangle/2},
                    \quad
    \lambda\in\DSp_{\bC}.
\end{equation*}
Як добре відомо (напр.,~\cite{Cl70,Oc84,Lo99}), кожну квадратично інтегровну за $\gamma$ та диференційовну за Хідою комплекснозначну функцію (випадкову величину) $F$ на $\DSp'$ можна представити у вигляді
\begin{equation}\label{e0_2}
    F=\bfE F+\int\bfE\big(\partial_t F|_{\mathcal F_t}\big)dW_t,
\end{equation}
де $\bfE$ позначає математичне сподівання; $\bfE\big(\circ|_{\mathcal F_t}\big)$ -- умовне математичне сподівання відносно повної $\sigma$"=алгебри $\mathcal F_t$, породженої вінерівським процесом $W$ до моменту часу $t$ (тобто $\mathcal F_t$ -- поповнення відносно $\gamma$ $\sigma$"=алгебри $\sigma (W_u:u\leq t)$); $\partial_\cdot F$ -- похідна Хіди $F$; $\int\circ (t)dW_t$ позначає стохастичний інтеграл Іто за вінерівським процесом (для інтегралів по $\bR_+$ ми, як правило, не вказуємо границі інтегрування задля спрощення позначень).
Формула~\eqref{e0_2} називається {\it формулою Кларка"=Окона}.
Як бачимо, ця формула, зокрема, дозволяє поновити версію підінтегральної функції (ця функція не є єдиною, взагалі кажучи), якщо відомий результат стохастичного інтегрування.

Як відомо (напр.,~\cite{NOP09,Zh09}), формула~\eqref{e0_2} залишається справедливою (з точністю до зрозумілих модифікацій), якщо замість гауссівської міри розглядається пуассонівська.
Відзначимо також, що можна легко уникнути обмежувального припущення, що випадкова величина $F$ має бути диференційовною за Хідою: достатньо узагальнити формулу Кларка"=Окона на певні простори узагальнених функцій (при цьому $F$ може залишатись квадратично інтегровною), див., напр.,~\cite{FOS00,NOP04}.

Формула Кларка"=Окона та її узагальнення мають численні застосування, зокрема, у стохастичному аналізі та у фінансовій математиці, див., напр.,~\cite{KO91,AOPU00,NOP04,PT07,ET08,Os08,MN08,AGR09,NOP09,Zh09} і посилання там.
Задля задоволення потреб застосувань побудовано різноманітні формули типу Кларка"=Окона на різних просторах, з використанням різних стохастичних похідних та зі стохастичними інтегралами за різними випадковими процесами й мірами, див., зокрема,~\cite{KOL91,Lo99,AOPU00,FOS00,BNLOP03,Lo04,NOP04,MN08,Zh09,NOP09,K11c,K11_2c,K12c}.
Наприклад, в~\cite{Lo99,Lo04} отримано формулу типу Кларка"=Окона, пов'язану з процесами Леві, яка містить стохастичні інтеграли за вінерівським процесом та за компенсованою пуассонівською випадковою мірою.

В~\cite{NOP04} запропоновано інший підхід до побудови формул типу Кларка"=Окона в аналізі Леві, який базується на так званому розкладі Нуаларта-Скоутенса квадратично інтегровних випадкових величин~\cite{NS00,S00}; зараз згадані формули містять інтеграли за спеціальними випадковими процесами.
Варто відзначити, що автори~\cite{NOP04} також узагальнюють свої результати на певні простори узагальнених випадкових величин.

В роботах автора~\cite{K11c,K11_2c,K12c} побудовано формули типу Кларка"=Окона на просторах регулярних основних, квадратично інтегровних та регулярних узагальнених функцій майкснерівського аналізу білого шуму.
Цей аналіз пов'язаний з узагальненою мірою Майкснера $\mathbf m$~\cite{R05} та з відповідним випадковим процесом Майкснера, похідною якого (в сенсі узагальнених функцій~\cite{GV61}) є майкснерівський білий шум (мірою цього шуму як узагальненого випадкового процесу~\cite{GS73} є $\mathbf m$).

Зауважимо, що підклас процесів Майкснера, який складається зі стаціонарних випадкових процесів, є доволі широким підкласом процесів Леві. Проте, побудови~\cite{K11c,K11_2c,K12c} суттєво відрізняються від побудов~\cite{Lo99,Lo04} та~\cite{NOP04}: ми намагались зберегти, наскільки це можливо, класичну форму формул типу Кларка"=Окона, і тому використовували стохастичну похідну Хіди та стохастичне інтегрування лише за процесом Майкснера.

Дана робота в певному сенсі є продовженням досліджень~\cite{K11c,K11_2c,K12c}, зараз наша мета полягає в отриманні та вивченні формул типу Кларка"=Окона на просторах так званого регулярного параметризованого оснащення простору квадратично інтегровних випадкових величин в аналізі білого шуму Леві~\cite{K13_2,K21}. Зокрема, ми встановлюємо необхідну і достатню умову, за якої можливо отримати формули типу Кларка"=Окона з використанням інтегрування лише за випадковим процесом Леві; отримуємо різні формули типу Кларка"=Окона та вивчаємо властивості підінтегральних функцій у цих формулах; а також з'ясовуємо необхідну і достатню умову, за якої формула типу Кларка"=Окона набуває класичного вигляду~\eqref{e0_2} (з процесом Леві замість вінерівського процесу).

Статтю організовано наступним чином.
В першому розділі ми наводимо необхідні попередні відомості: розглядаємо процес Леві та будуємо пов'язаний з ним ймовірнісний простір, зручний для подальшого викладу; описуємо запропоноване Є.~В.~Литвиновим~\cite{L03} узагальнення властивості хаотичного розкладу в аналізі білого шуму Леві, на основі якого побудоване регулярне параметризоване оснащення простору квадратично інтегровних випадкових величин~\cite{K13_2}; нагадуємо конструкцію згаданого оснащення; а також описуємо конструкції розширеного стохастичного інтеграла та стохастичної похідної Хіди на його просторах~\cite{K13,K13_2}.
Другий розділ присвячено побудові та вивченню формул типу Кларка"=Окона: розглянуто формулу Кларка"=Окона в найпростішому частинному випадку; встановлено необхідну і достатню умову, за якої можливо отримати формули типу Кларка"=Окона з використанням стохастичного інтегрування лише за випадковим процесом Леві; отримано згадані формули найпростішого та наближеного до класичного вигляду; а також визначено необхідну і достатню умову, за якої формула Кларка"=Окона в аналізі Леві набуває класичного вигляду.


\section{Попередні відомості}
В цій роботі будемо позначати через $\|\cdot\|_H$ або $|\cdot|_H$ норму в просторі $H$; через $(\cdot,\cdot)_H$ дійсний (тобто білінійний) скалярний добуток в просторі $H$; через $\langle\!\langle\cdot,\cdot\rangle\!\rangle_H$ або $\langle\cdot,\cdot\rangle_H$ дуальне спарювання, породжене скалярним добутком в просторі $H$; через $\mathcal B$ борелівську $\sigma$"=алгебру; через $1_\Delta$ індикатор множини або події $\Delta$; та через $\widehat\otimes$ симетричне тензорне множення.
Також ми використовуємо позначення $\mathop{\rm{pr\ lim}}$ (відповідно, $\mathop{\rm{ind\ lim}}$) для проективної (відповідно, індуктивної) границі сім'ї просторів; це позначення означає, що граничний простір наділено топологією проективної (відповідно, індуктивної) границі (див., напр.,~\cite{BUS90}).

\subsection{Процес Леві та його ймовірнісний простір}
Нехай
\[
    L=(L_t)_{t\in\bR_+}
\]
-- дійснозначний локально квадратично інтегровний процес Леві (тобто неперервний за ймовірністю випадковий процес на $\bR_+$ зі стаціонарними незалежними приростами і такий, що $L_0=0$, див., напр.,~\cite{B96}) без гауссівської частини та зсуву.
Добре відомо (напр.,~\cite{NOP04}), що характеристична функція $L$ має вигляд
\begin{equation}\label{e1_1}
    \bfE[e^{i\theta L_t}] =\exp\Big[t\int_{\bR}(e^{i\theta x}-1-i\theta x)\nu(dx)\Big],
\end{equation}
де $\nu$ -- міра Леві процесу $L$, що є мірою на $(\bR,\mathcal B(\bR))$.
Накладемо додатково такі умови:
{\it $\nu$ є мірою Радона з носієм, що містить нескінченну кількість точок;
$\nu(\{0\})=0$; існує $\varepsilon>0$ таке, що
\[ \int_{\bR}x^2e^{\varepsilon |x|}\nu(dx)<\infty;\]
та
\begin{equation}\label{e1_26}
\int_{\bR}x^2\nu(dx)=1.
\end{equation}
}

Визначимо міру білого шуму процесу $L$.
\begin{subdefinition}
Ймовірнісна міра $\mu$ на вимірному просторі $(\DSp',\Ccal(\DSp'))$ з перетворенням Фур'є
\begin{equation}\label{e1_2}
    \int_{\DSp'}e^{i\langle\omega,\varphi\rangle}\mu(d\omega)
        =\exp\biggl[\int_{\bR_+\times\bR}(e^{i\varphi(t)x}-1-i\varphi(t)x)dt\nu(dx)\biggr],
    \quad
    \varphi\in\DSp,
\end{equation}
називається мірою білого шуму Леві.
\end{subdefinition}
Коректність цього визначення (тобто існування $\mu$) випливає з теореми Бохнера-Мінлоса (напр.,~\cite{HOUZ96}), див.~\cite{L03}.
Нижче будемо вважати, що {\it циліндрична $\sigma$"=алгебра $\Ccal(\DSp')$ поповнена відносно $\mu$}.

Позначимо через
\[
    (L^2):=L^2(\DSp',\Ccal(\DSp'),\mu)
\]
простір (класів) квадратично інтегровних за $\mu$ комплекснозначних функцій на $\DSp'$.
Нехай $f\in L^2(\bR_+)$ та послідовність $(\varphi_k\in\DSp)_{k\in\bN}$ збігається до $f$ у $L^2(\bR_+)$, коли $k\to\infty$.
Можна показати (напр.,~\cite{L03,K13}), що
\[
    \langle\circ,f\rangle :=(L^2) - \lim\limits_{k\to\infty}\langle\circ,\varphi_k\rangle
\]
є коректно визначеним елементом $(L^2)$ (зокрема, $\langle\circ,f\rangle$ не залежить від того, якою саме послідовністю елементів $\DSp$ апроксимовано $f$).

Покладемо $1_{[0,0)}\equiv 0$ (формальний напівінтервал $[0,0)$ природно вважати порожньою множиною).
З~\eqref{e1_1} та~\eqref{e1_2} випливає, що
\[
    \big(\langle\circ,1_{[0,t)}\rangle\big)_{t\in\bR_+}
\]
можна ототожнити з процесом Леві на ймовірнісному просторі (ймовірнісній трійці) $(\DSp',\Ccal(\DSp'),\mu)$ (див., напр.,~\cite{NOP04,NOP09}).
Таким чином, для кожного $t\in\bR_+$ маємо $L_t=\langle\circ,1_{[0,t)}\rangle\in (L^2)$.

Зауважимо, що похідна у сенсі узагальнених функцій процесу Леві (тобто білий шум Леві)
\[
    \dot L_\cdot(\omega)=\langle\omega,\delta_\cdot\rangle\equiv\omega(\cdot),
\]
де $\delta$ є дельта-функцією Дірака.
Отже, $\dot L$ є узагальненим випадковим процесом (в сенсі~\cite{GV61}) з траєкторіями з $\DSp'$, а $\mu$ є мірою $\dot L$ у класичному сенсі~\cite{GS73}.

\subsection{Литвинівське узагальнення властивості хаотичного розкладу}\label{S1_2}
Як відомо, фундаментальну роль у гауссівському аналізі білого шуму відіграє так звана {\it властивість хаотичного розкладу} (ВХР), яка полягає, грубо кажучи, у наступному: кожну квадратично інтегровну випадкову величину можна єдиним чином розкласти в ряд з повторних стохастичних інтегралів Іто від невипадкових функцій (див. детальний виклад, напр., у \cite{M93}).
Використовуючи ВХР, можна будувати різні простори основних і узагальнених функцій, уводити та досліджувати різноманітні оператори і операції на цих просторах (зокрема, стохастичні інтеграли та похідні, віківське множення), тощо. В аналізі білого шуму Леві, на жаль, ВХР немає (точніше, серед процесів Леві тільки вінерівський та пуассонівський мають цю властивість, див. подробиці у~\cite{S81}); але побудовано низку її узагальнень (короткий опис таких узагальнень із відповідними посиланнями міститься у~\cite{K21}).

В цій роботі ми використовуємо одне з найкорисніших узагальнень ВХР в аналізі Леві, запропоноване Є.~В.~Литвиновим~\cite{L03}.
Коротко опишемо це узагальнення.

Розповсюдимо уведене вище позначення $\langle\cdot,\cdot\rangle$ на дуальні спарювання в симетричних тензорних степенях комплексифікації ланцюжка~\eqref{e0_1}.
Нехай $\bZ_+:=\bN\cup\{0\}$.
Позначимо через $\mathcal P$ множину комплекснозначних поліномів на $\DSp'$, яка складається з нуля та елементів вигляду
\begin{equation*}
    f(\omega) = \sum_{n=0}^{N_f}\langle\omega^{\otimes n},f^{(n)}\rangle,
    \quad \omega\in\DSp',\ f^{(n)}\in\DSp_{\bC}^{\widehat\otimes n},
\ N_f\in\bZ_+,\ f^{(N_f)}\not=0,
\end{equation*}
тут $N_f$ -- {\it степінь поліному $f$}; $\langle\omega^{\otimes 0},f^{(0)}\rangle:=f^{(0)}\in \DSp_{\bC}^{\widehat\otimes 0}:=\bC$.
Міра білого шуму Леві $\mu$ має голоморфне в нулі перетворення Лапласа (це випливає з~\eqref{e1_2} та властивостей  міри Леві $\nu$, див.~\cite{L03}), отже $\mathcal P$ є щільною множиною у $(L^2)$~\cite{S75}.

Позначимо через $\mathcal P_n$, $n\in\bZ_+$, множину поліномів степені не більше $n$, через $\overline{\mathcal P}_n$ -- замикання $\mathcal P_n$ в $(L^2)$. Нехай для $n\in\bN$
\[
    \bfP_n:=\overline{\mathcal P}_n \ominus\overline{\mathcal P}_{n-1} \ \text{(ортогональна різниця в $(L^2)$).}
\]
Покладемо також $\bfP_0:=\overline{\mathcal P}_0$.
Зрозуміло, що
\begin{equation}\label{e1_3}
(L^2)=\mathop{\oplus}_{n=0}^\infty\bfP_n.
\end{equation}

Нехай $f^{(n)}\in\DSp_{\bC}^{\widehat\otimes n}$, $n\in\bZ_+$.
Позначимо через
\[
    \ortp{\circ^{\otimes n}}{f^{(n)}}  \in (L^2)
\]
ортогональну проекцію монома $\langle\circ^{\otimes n},f^{(n)}\rangle$ на $\bfP_n$.
Визначимо дійсні (білінійні) скалярні добутки $(\cdot,\cdot)_{ext}$ на $\DSp^{\widehat\otimes n}_{\bC}$, $n\in\bZ_+$, поклавши для
$f^{(n)},g^{(n)}\in\DSp_{\bC}^{\widehat\otimes n}$
\begin{equation}\label{e1_4}
    (f^{(n)},g^{(n)})_{ext}:=\frac{1}{n!}
        \int_{\DSp'}\ortp{\omega^{\otimes n}}{f^{(n)}}  \ortp{\omega^{\otimes n}}{g^{(n)}} \mu(d\omega).
\end{equation}
Коректність цього визначення доведено (з точністю до очевидних модифікацій) у~\cite{L03}.

Позначимо через $|\cdot|_{ext}$ норми, що відповідають скалярним добуткам~\eqref{e1_4}, тобто $|\cdot|_{ext}:=\sqrt{(\cdot,\overline{\cdot})_{ext}}$.
\begin{subdefinition}
Для кожного $n\in\bZ_+$ визначимо гільбертів простір $\Hcal_{ext}^{(n)}$ як поповнення $\DSp^{\widehat\otimes n}_{\bC}$ за відповідною нормою $|\cdot|_{ext}$ (для скалярних добутків та норм у просторах $\Hcal_{ext}^{(n)}$ ми збережемо позначення $(\cdot,\cdot)_{ext}$ та $|\cdot|_{ext}$ відповідно).
\end{subdefinition}

Для кожного $F^{(n)}\in\Hcal_{ext}^{(n)}$ визначимо {\it віківський моном}
\[
    \ortp{ \circ^{\otimes n}}{F^{(n)}}
         \overset{\rm def} =
        (L^2)-\lim\limits_{k\to\infty} \ortp{\circ^{\otimes n}}{f^{(n)}_k},
\]
де $\DSp_{\bC}^{\widehat\otimes n}\ni f^{(n)}_k \mathop{\to}\limits_{k\to\infty} F^{(n)}$ в $\Hcal_{ext}^{(n)}$ (коректність цього визначення можна довести методом <<змішаних послідовностей>>).
Легко бачити що, зокрема,
\[
\ortp{\circ^{\otimes 0}}{F^{(0)}}  = \langle\circ^{\otimes 0},F^{(0)}\rangle=F^{(0)}
\qquad\text{та}\qquad
\ortp{\circ}{F^{(1)}}              = \langle\circ,F^{(1)}\rangle,
\]
(пор. з~\cite{L03}).

Оскільки, як неважко бачити,
\[
    \bfP_n = \big\{
        \ortp{\circ^{\otimes n}}{F^{(n)}}  \,  \mid \,
        F^{(n)} \in \Hcal_{ext}^{(n)}\big\}
\]
для кожного $n\in\bZ_+$, то з розкладу~\eqref{e1_3} випливає таке твердження.
\begin{subtheorem}[{\rm литвинівське узагальнення ВХР, пор. з~\cite{L03}}]\label{t1_2_1}
Випадкова величина $F\in (L^2)$ якщо та лише якщо існує єдина послідовність ядер $F^{(n)}\in\Hcal_{ext}^{(n)}$, $n\in\bZ_+$, така, що
\begin{equation}\label{e1_5}
    F = \sum_{n=0}^\infty \ortp{\circ^{\otimes n}}{F^{(n)}}
\end{equation}
(ряд збігається у $(L^2)$) та
\begin{equation}\label{e1_6}
   \|F\|_{(L^2)}^2 = \int_{\DSp'} |F(\omega)|^2\mu(d\omega) 
                   = \bfE|F|^2 
                   = \sum_{n=0}^\infty n! |F^{(n)}|_{ext}^2 
                   < \infty.
\end{equation}
\end{subtheorem}
\begin{subcorollary}
Для $F,G\in (L^2)$ дійсний (білінійний) скалярний добуток має вигляд
\begin{equation*}
    (F,G)_{(L^2)} = \int_{\DSp'} F(\omega)G(\omega)\mu(d\omega)
                  = \bfE[FG] 
                  = \sum_{n=0}^\infty n!(F^{(n)},G^{(n)})_{ext},
\end{equation*}
де $F^{(n)},G^{(n)}\in\Hcal_{ext}^{(n)}$ -- ядра з розкладів~\eqref{e1_5} для $F$ та $G$ відповідно. 
Зокрема, для $F^{(n)}\in\Hcal_{ext}^{(n)}$, $G^{(m)}\in\Hcal_{ext}^{(m)}$, $n,m\in\bZ_+$, 
\begin{equation*}
    \bigl( 
        \ortp{\circ^{\otimes n}}{F^{(n)}}, 
        \ortp{\circ^{\otimes m}}{G^{(m)}} 
    \bigr)_{(L^2)} 
    = \delta_{nm} n!(F^{(n)},G^{(n)})_{ext},
\end{equation*}
де $\delta_{nm}$ -- символ Кронекера.
\end{subcorollary}

\begin{subremark}
Розклад~\eqref{e1_5} є аналогом розкладу квадратично інтегровної випадкової величини за ортогональними поліномами Ерміта, який є еквівалентним розкладу за повторними стохастичними інтегралами Іто у гауссівському аналізі. 
В той же час віківські мономи з~\eqref{e1_5} є поліномами лише у тому випадку, коли процес Леві є стаціонарним процесом Майкснера. 
Зацікавлений читач може знайти детальну інформацію про це у~\cite{L03}.
\end{subremark}

Для отримання багатьох результатів, пов'язаних з просторами $\Hcal_{ext}^{(n)}$, необхідно рахувати скалярні добутки та норми у цих просторах. 
Наведена вище формула~\eqref{e1_4} практично непридатна для таких підрахунків; але, на щастя, існує відносно проста явна формула для згаданих скалярних добутків, отримана Є.~В.~Литвиновим у роботі~\cite{L03}. Наведемо цю формулу у трохи модифікованій формі, отриманій у~\cite{K13}. 
Нехай
\begin{gather*}
    p_n(x):=x^n+a_{n,n-1}x^{n-1}+\cdots+a_{n,1}x, \\
    a_{n,j}\in\bR,\ j\in\{1,\dots,n-1\},\ n\in\bN,
\end{gather*}
є ортогональними поліномами у просторі $L^2(\bR,\nu)$ (класів) квадратично інтегровних за мірою Леві $\nu$ (див.~\eqref{e1_1},~\eqref{e1_2}) дійснозначних функцій на $\bR$, тобто для довільних натуральних чисел $n,m$ таких, що $n\not=m$, 
\[ \int_{\bR}p_n(x)p_m(x)\nu(dx)=0.\]
Позначимо через $\|\cdot\|_\nu$ норму в $L^2(\bR,\nu)$.
Зауважимо, що $p_1(x)=x$, а тому згідно з~\eqref{e1_26}, $\|p_1\|_\nu=1$.

Для $F^{(n)},G^{(n)}\in\Hcal_{ext}^{(n)}$, $n\in\bN$, маємо
\begin{equation}\label{e1_7}
\begin{gathered}
(F^{(n)},G^{(n)})_{ext}
\equiv (F^{(n)},G^{(n)})_{\Hcal_{ext}^{(n)}}=\\
=\sum_{{k,l_j,s_j\in\bN:
\ j=1,\dots,k,\ l_1>l_2>\cdots>l_k,}\atop{l_1s_1+\cdots+l_ks_k=n}}
\frac{n!}{s_1!\cdots s_k!}
\Big(\frac{\|p_{l_1}\|_\nu}{l_1!}\Big)^{2s_1}
\cdots\Big(\frac{\|p_{l_k}\|_\nu}{l_k!}\Big)^{2s_k}\times\\
\times\int\limits_{\bR_+^{s_1+\cdots+s_k}}F^{(n)}(
\underset{l_1}{\underbrace{t_1,\dots,t_1}},\dots,
\underset{l_1}{\underbrace{t_{s_1},\dots,t_{s_1}}},\dots,
\underset{l_k}{\underbrace{t_{s_1+\cdots+s_k},\dots,
t_{s_1+\cdots+s_k}}})\times\\
\times G^{(n)}(
\underset{l_1}{\underbrace{t_1,\dots,t_1}},\dots,
\underset{l_1}{\underbrace{t_{s_1},\dots,t_{s_1}}},\dots,
\underset{l_k}{\underbrace{t_{s_1+\cdots+s_k},\dots,
t_{s_1+\cdots+s_k}}})\times\\
\times dt_1\cdots dt_{s_1+\cdots+s_k}.
\end{gathered}
\end{equation}
Зокрема, для $n=1$
$(F^{(1)},G^{(1)})_{ext}=(F^{(1)},G^{(1)})_{L^2(\bR_+)_{\bC}}$,
для $n=2$
\begin{gather*}
(F^{(2)},G^{(2)})_{ext}
=(F^{(2)},G^{(2)})_{L^2(\bR_+)_{\bC}^{\otimes 2}}
+\frac{\|p_2\|_\nu^2}{2}\int_{\bR_+}F^{(2)}(t,t)G^{(2)}(t,t)dt,
\end{gather*}
і т.~д. Зауважимо, що для кожного натурального $n>1$ простір $\Hcal_{ext}^{(n)}$
є симетричним підпростором простору (класів) квадратично інтегровних за певною
мірою Радона комплекснозначних функцій на $\bR_+^n$.

Позначимо $\Hcal:=L^2(\bR_+)$, тоді $\Hcal_{\bC}
=L^2(\bR_+)_{\bC}$. З~\eqref{e1_7} випливає, що
{\it $\Hcal_{ext}^{(1)}=\Hcal_{\bC}$, і для кожного
$n\in\bN\backslash\{1\}$ простір $\Hcal_{\bC}^{\widehat\otimes n}$
можна ототожнити із власним підпростором простору $\Hcal_{ext}^{(n)}$, який
складається зі "зникаючих на діагоналях"\ елементів (тобто таких $F^{(n)}$, які містять
представника (функцію) $f^{(n)}\in F^{(n)}$ таку, що $f^{(n)}(t_1,\ldots,t_n)=0$, якщо
існують $k,j\in\{1,\ldots,n\}$: $k\not=j$, але $t_k=t_j$)}.
У цьому сенсі {\it простір $\Hcal_{ext}^{(n)}$ є розширенням
(англ. extension) простору $\Hcal_{\bC}^{\widehat\otimes n}$}, цим
пояснюється, чому ми використовуємо індекси "$ext$"\ у наших позначеннях. В подальшому,
говорячи про вкладення просторів $\Hcal_{\bC}^{\widehat\otimes n}$ в простори
$\Hcal_{ext}^{(n)}$ та використовуючи позначення на кшталт
$\Hcal_{\bC}^{\widehat\otimes n}\subset\Hcal_{ext}^{(n)}$, завжди розуміємо
такі вкладення у щойно описаному сенсі.



\subsection{Регулярне оснащення простору квадратично інтегровних випадкових величин}
Позначимо
\begin{equation*}
\mathcal P_W:=\big\{f=\sum_{n=0}^{N_f}
{:\!\langle}\circ^{\otimes n},f^{(n)}{\rangle\!:}\ |\ f^{(n)}
\in\DSp_{\bC}^{\widehat\otimes n},N_f\in\bZ_+\big\}
\subset (L^2).
\end{equation*}
Нехай $\beta\in [0,1]$, $q\in\bZ$ у випадку $\beta\not=0$,
та $q\in\bZ_+$ якщо $\beta=0$.
Визначимо дійсні (білінійні) скалярні добутки
$(\cdot,\cdot)_{q,\beta}$ на $\mathcal P_W$, поклавши для
\begin{gather*}
f=\sum_{n=0}^{N_f}{:\!\langle}\circ^{\otimes n},f^{(n)}{\rangle\!:},
\ g=\sum_{n=0}^{N_g}{:\!\langle}\circ^{\otimes n},g^{(n)}{\rangle\!:}
\in\mathcal P_W\\
(f,g)_{q,\beta}:=\sum_{n=0}^{\min(N_f,N_g)}(n!)^{1+\beta}2^{qn}
(f^{(n)},g^{(n)})_{ext}.
\end{gather*}
Легко перевірити~\cite{F18}, що $(\cdot,\cdot)_{q,\beta}$ задовольняє аксіоми
скалярного добутку.

Позначимо через $(L^2)^\beta_q$ гільбертові простори, що є поповненнями $\mathcal P_W$ за
нормами, породженими скалярними добутками $(\cdot,\cdot)_{q,\beta}$, та покладемо
$(L^2)^\beta:=\mathop{\rm{pr\ lim}}_{q\to+\infty}(L^2)^\beta_q$. Легко бачити,
що справедливе таке твердження (пор. з Теоремою~\ref{t1_2_1} та її наслідком).
\begin{subproposition}\label{p1_3_1}
1. Випадкова величина $F\in (L^2)^\beta_q$ якщо та лише якщо існує єдина послідовність ядер
$F^{(n)}\in\Hcal_{ext}^{(n)}$, $n\in\bZ_+$, така, що $F$ розкладається
в ряд~\eqref{e1_5}, який збігається у $(L^2)^\beta_q$, тобто
\begin{equation}\label{e1_8}
\|F\|_{(L^2)^\beta_q}^2
=\sum_{n=0}^\infty (n!)^{1+\beta}2^{qn}|F^{(n)}|_{ext}^2<\infty.
\end{equation}

2. Випадкова величина $F\in (L^2)^\beta$ якщо та лише якщо її можна єдиним чином представити
у вигляді~\eqref{e1_5}, а відповідний ряд~\eqref{e1_8} збігається для кожного $q\in\bZ_+$.

3. Для $F,G\in (L^2)^\beta_q$ скалярний добуток у $(L^2)^\beta_q$ має вигляд
\begin{equation*}
(F,G)_{(L^2)^\beta_q}
=\sum_{n=0}^\infty (n!)^{1+\beta}2^{qn}
(F^{(n)},G^{(n)})_{ext},
\end{equation*}
де $F^{(n)},G^{(n)}\in\Hcal_{ext}^{(n)}$ -- ядра з розкладів~\eqref{e1_5} для
$F$ та $G$ відповідно.
\end{subproposition}
Наступне твердження є тривіальною модифікацією відповідного твердження з
\cite{K13_2}.
\begin{subproposition}
Для довільних $\beta\in (0,1]$ та $q\in\bZ$, так само як і для $\beta=0$ та
$q\in\bZ_+$, простір $(L^2)^\beta_q$ щільно та неперервно вкладено у простір
$(L^2)=(L^2)^0_0$.
\end{subproposition}

Прийнявши до уваги цей результат, побудуємо ланцюжок
\begin{equation}\label{e1_9}
(L^2)^{-\beta}\supset (L^2)^{-\beta}_{-q}\supseteq (L^2)=(L^2)^0_0
\supseteq (L^2)^\beta_q\supset (L^2)^\beta,
\end{equation}
де $(L^2)^{-\beta}_{-q}$ та $(L^2)^{-\beta}
=\mathop{\rm{ind\ lim}}_{q\to+\infty}(L^2)^{-\beta}_{-q}$ -- простори, спряжені
відповідно до $(L^2)^\beta_q$ та $(L^2)^\beta$ відносно $(L^2)$.
\begin{subdefinition}
Ланцюжок~\eqref{e1_9} називається параметризованим регулярним оснащенням простору
$(L^2)$ квадратично інтегровних випадкових величин. Простори $(L^2)^\beta_q$ та
$(L^2)^\beta$ називаються параметризованими просторами типу Кондратьєва регулярних
основних функцій, а простори $(L^2)^{-\beta}_{-q}$ та $(L^2)^{-\beta}$ --
параметризованими просторами типу Кондратьєва регулярних узагальнених функцій.
\end{subdefinition}

Наступне твердження випливає безпосередньо із цього означення та загальної теорії
дуальності.
\begin{subproposition}{\rm (пор. з Теоремою~\ref{t1_2_1}, її наслідком та
Твердженням~\ref{p1_3_1})}

1. Регулярна узагальнена функція (узагальнена випадкова величина) $F\in (L^2)^{-\beta}_{-q}$
якщо та лише якщо існує єдина послідовність ядер $F^{(n)}\in\Hcal_{ext}^{(n)}$,
$n\in\bZ_+$, така, що $F$ розкладається в ряд~\eqref{e1_5}, який збігається у
$(L^2)^{-\beta}_{-q}$, тобто
\begin{equation}\label{e1_10}
\|F\|_{(L^2)^{-\beta}_{-q}}^2
=\sum_{n=0}^\infty (n!)^{1-\beta}2^{-qn}|F^{(n)}|_{ext}^2<\infty.
\end{equation}

2. Регулярна узагальнена функція $F\in (L^2)^{-\beta}$ якщо та лише якщо її можна єдиним
чином представити у вигляді~\eqref{e1_5}, а відповідний ряд~\eqref{e1_10} збігається для
деякого $q\in\bZ_+$.

3. Для $F,G\in (L^2)^{-\beta}_{-q}$ скалярний добуток у $(L^2)^{-\beta}_{-q}$
має вигляд
\begin{equation*}
(F,G)_{(L^2)^{-\beta}_{-q}}
=\sum_{n=0}^\infty (n!)^{1-\beta}2^{-qn}
(F^{(n)},G^{(n)})_{ext},
\end{equation*}
де $F^{(n)},G^{(n)}\in\Hcal_{ext}^{(n)}$ -- ядра з розкладів
\eqref{e1_5} для $F$ та $G$ відповідно.

4. Дуальне спарювання між елементами $F\in (L^2)^{-\beta}_{-q}$ та
$f\in (L^2)^\beta_q$, породжене дійсним (білінійним) скалярним добутком у $(L^2)$,
має вигляд
\begin{equation*}
\langle\!\langle F,f\rangle\!\rangle_{(L^2)}
=\sum_{n=0}^\infty n!(F^{(n)},f^{(n)})_{ext},
\end{equation*}
де $F^{(n)},f^{(n)}\in\Hcal_{ext}^{(n)}$ --
ядра з розкладів~\eqref{e1_5} для $F$ та $f$ відповідно.
\end{subproposition}

Відзначимо, що термін "регулярні"\ у назвах ланцюжка~\eqref{e1_9} та просторів
основних і узагальнених функцій пов'язаний із тим фактом, що ядра з розкладів
\eqref{e1_5} елементів всіх просторів ланцюжка~\eqref{e1_9} належать одним і тим
самим просторам $\Hcal_{ext}^{(n)}$. Більше того, простори $(L^2)^\beta_q$,
$(L^2)=(L^2)^0_0$ та $(L^2)^{-\beta}_{-q}$ мають однакову структуру (пор.
\eqref{e1_8},~\eqref{e1_6} та~\eqref{e1_10}), тому в подальшому ми абстрагуємось від того,
йдеться про регулярні основні, квадратично інтегровні чи регулярні узагальнені
функції, та {\it будемо за умовчанням розглядати $(L^2)^\beta_q$, $\beta\in [-1,1]$,
$q\in\bZ$, з нормою~\eqref{e1_8}}.

\begin{subremark}
Використання ваг $2^{qn}$ саме з числом $2$ та з цілим $q$ у визначенні скалярних добутків
$(\cdot,\cdot)_{q,\beta}$ не є принциповим -- можна використовувати більш загальні
ваги $K^{qn}$ із довільними $K>1$ та $q\in\bR$. Але такі узагальнення не є суттєвим для
кола питань, які розглядаються у статті, тому ми обмежимось розглядом випадку $K=2$ та
$q\in\bZ$ задля спрощення формул та позначень.
\end{subremark}

\subsection{Розширений стохастичний інтеграл}\label{S1_4}
Розклад~\eqref{e1_5} для елементів $(L^2)^\beta_q$ визначає ізометричний ізоморфізм
(узагальнений ізоморфізм Вінера-Іто-Сігала)
$\mathbf I: (L^2)^\beta_q\to\mathop{\oplus}\limits_{n=0}^\infty (n!)^{1+\beta}2^{qn}
\Hcal_{ext}^{(n)}$:
для $F\in (L^2)^\beta_q$ вигляду~\eqref{e1_5}
$\mathbf IF=(F^{(0)},F^{(1)},\dots)
\in\mathop{\oplus}\limits_{n=0}^\infty (n!)^{1+\beta}2^{qn}\Hcal_{ext}^{(n)}$.
Нехай $\mathbf 1$ -- одиничний оператор на $\Hcal_{\bC}$. Тоді оператор
\begin{equation*}
\mathbf I\otimes\mathbf 1:
(L^2)^\beta_q\otimes\Hcal_{\bC}
\to\big(\mathop{\oplus}\limits_{n=0}^\infty (n!)^{1+\beta}2^{qn}
\Hcal_{ext}^{(n)}\big)\otimes\Hcal_{\bC}\cong
\mathop{\oplus}\limits_{n=0}^\infty (n!)^{1+\beta}2^{qn}
(\Hcal_{ext}^{(n)}\otimes\Hcal_{\bC})
\end{equation*}
є ізометричним ізоморфізмом між гільбертовими просторами
$(L^2)^\beta_q\otimes\Hcal_{\bC}$ та
$\mathop{\oplus}\limits_{n=0}^\infty (n!)^{1+\beta}2^{qn}
(\Hcal_{ext}^{(n)}\otimes\Hcal_{\bC})$. Зрозуміло, що для довільних
$m\in\bZ_+$ та $F^{(m)}_\cdot
\in\Hcal_{ext}^{(m)}\otimes\Hcal_{\bC}$ вектор
$(\underset{m}{\underbrace{0,\dots,0}},
F^{(m)}_\cdot,0,\dots)$ належить простору
$\mathop{\oplus}\limits_{n=0}^\infty (n!)^{1+\beta}2^{qn}
(\Hcal_{ext}^{(n)}\otimes\Hcal_{\bC})$. Покладемо
\begin{equation*}
{:\!\langle}\circ^{\otimes m},F^{(m)}_\cdot{\rangle\!:}
\overset{def}=(\mathbf I\otimes\mathbf 1)^{-1}(\underset{m}
{\underbrace{0,\dots,0}},F^{(m)}_\cdot,0,\dots)\in (L^2)^\beta_q
\otimes\Hcal_{\bC}.
\end{equation*}
За побудовою елементи ${:\!\langle}\circ^{\otimes n},
F^{(n)}_\cdot{\rangle\!:}$, $n\in\bZ_+$, формують ортогональний базис у
просторі $(L^2)^\beta_q\otimes\Hcal_{\bC}$ у тому сенсі, що
{\it $F$ належить $(L^2)^\beta_q\otimes\Hcal_{\bC}$ якщо та лише якщо $F$ можна
єдиним чином представити у вигляді ряду
\begin{equation}\label{e1_11}
F(\cdot)=\sum_{n=0}^\infty{:\!\langle}\circ^{\otimes n},
F^{(n)}_\cdot{\rangle\!:},
\quad F^{(n)}_\cdot\in\Hcal_{ext}^{(n)}\otimes\Hcal_{\bC},
\end{equation}
який збігається у $(L^2)^\beta_q\otimes\Hcal_{\bC}$, тобто
\begin{equation}\label{e1_12}
\begin{gathered}
\|F\|_{(L^2)^\beta_q\otimes\Hcal_{\bC}}^2
=\|(\mathbf I\otimes\mathbf 1)F\|_{\mathop{\oplus}\limits_{n=0}^\infty (n!)^{1+\beta}2^{qn}
(\Hcal_{ext}^{(n)}\otimes\Hcal_{\bC})}^2=\\
=\sum_{n=0}^\infty (n!)^{1+\beta}2^{qn}
|F^{(n)}_\cdot|_{\Hcal_{ext}^{(n)}\otimes\Hcal_{\bC}}^2<\infty.
\end{gathered}
\end{equation}
}

Опишемо конструкцію розширеного стохастичного інтеграла за процесом Леві, яка
базується на розкладі~\eqref{e1_11} (зацікавлений читач може знайти більш детальний виклад у
\cite{K13,K13_2}).

Нехай спочатку $F\in (L^2)^\beta_q\otimes\Hcal_{\bC}$ є таким,
що ядра $F^{(n)}_\cdot$ належать просторам
$\Hcal_{\bC}^{\widehat\otimes n}\otimes\Hcal_{\bC}
\subset\Hcal_{ext}^{(n)}\otimes\Hcal_{\bC}$ (див. Підрозділ~\ref{S1_2}). Тоді
розклад~\eqref{e1_11} еквівалентний представленню
\begin{equation}\label{e1_13}
F(\cdot)=F^{(0}_\cdot+\sum_{n=1}^\infty n!\int_0^\infty\int_0^{t_n}\cdots\int_0^{t_2}
F^{(n)}_\cdot (t_1,\dots,t_n)dL_{t_1}\cdots dL_{t_n}
\end{equation}
\cite{K13} (див. також~\cite{K07}), де ряд складається з повторних стохастичних інтегралів Іто;
а розширений стохастичний інтеграл можна визначити за класичною схемою як
\begin{equation}\label{e1_14}
\begin{gathered}
\int F(t)\widehat dL_t:=\\
:=\sum_{n=0}^\infty (n+1)!\int_0^\infty\int_0^{t_{n+1}}\cdots\int_0^{t_2}
\widehat F^{(n)}(t_1,\dots,t_{n+1})dL_{t_1}\cdots dL_{t_{n+1}}\cong\\
\cong\sum_{n=0}^\infty{:\!\langle}\circ^{\otimes n+1},
\widehat F^{(n)}{\rangle\!:}
\in (L^2)^\beta_{q-1},
\end{gathered}
\end{equation}
де $\widehat F^{(n)}\in\Hcal_{\bC}^{\widehat\otimes n+1}
\subset\Hcal_{ext}^{(n+1)}$, $n\in\bZ_+$, -- симетризації ядер $F^{(n)}_\cdot$ за
всіма аргументами (точніше, проекції
$F^{(n)}_\cdot\in\Hcal_{\bC}^{\widehat\otimes n}\otimes\Hcal_{\bC}$
на $\Hcal_{\bC}^{\widehat\otimes n+1}$).

У загальному випадку представлення~\eqref{e1_13} не має місця (адже в аналізі білого шуму
Леві ВХР немає), а ядра $\widehat F^{(n)}$ є невизначеними, оскільки, взагалі кажучи,
неможливо проектувати елементи з просторів
$\Hcal_{ext}^{(n)}\otimes\Hcal_{\bC}$ на простори $\Hcal_{ext}^{(n+1)}$
(див. Зауваження~\ref{r1_5_1} нижче). Тим не менш, можна зробити наступне природне узагальнення.
Нехай $F^{(n)}_\cdot\in\Hcal_{ext}^{(n)}\otimes\Hcal_{\bC}$, $n\in\bN$.
У класі еквівалентності $F^{(n)}_\cdot$ оберемо представника (функцію)
$\dot f^{(n)}_\cdot\in F^{(n)}_\cdot$ такого, що
\begin{equation}\label{e1_15}
\forall t,t_1,\dots,t_n\in\bR_+
\ \big\{\exists k\in\{1,\dots,n\} : t=t_k\big\}
\Rightarrow \dot f^{(n)}_t(t_1,\dots,t_n)=0
\end{equation}
(тобто $\dot f^{(n)}_t(t_1,\dots,t_n)=0$, якщо аргумент $t$ співпадає хоча б з одним
з аргументів $t_1,\dots,t_n$).
Нехай $\widehat f^{(n)}$ -- симетризація функції
$\dot f^{(n)}_\cdot$ за $n+1$ змінною. Визначимо
$\widehat F^{(n)}\in\Hcal_{ext}^{(n+1)}$ як клас еквівалентності в
$\Hcal_{ext}^{(n+1)}$, породжений функцією $\widehat f^{(n)}$ (тобто
$\widehat f^{(n)}\in\widehat F^{(n)}$).
Наступне твердження є тривіальною модифікацією відповідного результату з~\cite{K13}.
\begin{sublemma}
Для довільних $n\in\bN$ та
$F^{(n)}_\cdot\in\Hcal_{ext}^{(n)}\otimes\Hcal_{\bC}$ елемент
$\widehat F^{(n)}\in\Hcal_{ext}^{(n+1)}$ визначений коректно (зокрема,
$\widehat F^{(n)}$ не залежить від вибору представника
$\dot f^{(n)}_\cdot\in F^{(n)}_\cdot$, який задовольняє умову~\eqref{e1_15}), та
\begin{equation}\label{e1_16}
|\widehat F^{(n)}|_{\Hcal_{ext}^{(n+1)}}
\leq |F^{(n)}_\cdot|_{\Hcal_{ext}^{(n)}\otimes\Hcal_{\bC}}.
\end{equation}
\end{sublemma}
\begin{subremark}
Легко бачити, що якщо
$F^{(n)}_\cdot\in\Hcal_{\bC}^{\widehat\otimes n}\otimes\Hcal_{\bC}
\subset\Hcal_{ext}^{(n)}\otimes\Hcal_{\bC}$, то щойно побудоване ядро
$\widehat F^{(n)}$ є згаданою вище проекцією $F^{(n)}_\cdot$ на
$\Hcal_{\bC}^{\widehat\otimes n+1}\subset\Hcal_{ext}^{(n+1)}$.
\end{subremark}
\begin{subdefinition}
Для $F\in (L^2)^\beta_q\otimes\Hcal_{\bC}$
визначимо розширений стохастичний інтеграл за процесом Леві
$\int F(t)\widehat dL_t\in (L^2)^\beta_{q-1}$, поклавши
\begin{equation}\label{e1_17}
\int F(t)\widehat d L_t
:=\sum_{n=0}^\infty{:\!\langle}\circ^{\otimes n+1},
\widehat F^{(n)}{\rangle\!:}
\end{equation}
(пор. з~\eqref{e1_14}), де $\widehat F^{(0)}:=F^{(0)}_\cdot
\in\Hcal_{\bC}=\Hcal_{ext}^{(1)}$ та
$\widehat F^{(n)}\in\Hcal_{ext}^{(n+1)}$, $n\in\bN$, побудовані за ядрами
$F^{(n)}_\cdot\in\Hcal_{ext}^{(n)}\otimes\Hcal_{\bC}$ з розкладу
\eqref{e1_11} для $F$.
\end{subdefinition}
Оскільки (див.~\eqref{e1_17},~\eqref{e1_8},~\eqref{e1_16} та~\eqref{e1_12})
\begin{equation}\label{e1_18}
\begin{gathered}
\Big\|\int F(t)\widehat d L_t\Big\|_{(L^2)^\beta_{q-1}}^2
=\sum_{n=0}^\infty ((n+1)!)^{1+\beta}2^{(q-1)(n+1)}
|\widehat F^{(n)}|_{\Hcal_{ext}^{(n+1)}}^2\leq\\
\leq\sum_{n=0}^\infty (n!)^{1+\beta}2^{qn}[(n+1)^{1+\beta}2^{-n+q-1}]
|F^{(n)}_\cdot|_{\Hcal_{ext}^{(n)}\otimes\Hcal_{\bC}}^2\leq\\
\leq\max\limits_{n\in\bZ_+}[(n+1)^{1+\beta}2^{-n+q-1}]
\|F\|_{(L^2)^\beta_q\otimes\Hcal_{\bC}}^2,
\end{gathered}
\end{equation}
це визначення є коректним, а інтеграл
\begin{equation}\label{e1_19}
\int\circ (t)\widehat d L_t:
(L^2)^\beta_q\otimes\Hcal_{\bC}\to (L^2)^\beta_{q-1}
\end{equation}
є лінійним {\it обмеженим}, а тому і {\it неперервним} оператором.

Відзначимо, що стохастичний інтеграл~\eqref{e1_19} називається {\it розширеним}, оскільки у
випадках, коли $(L^2)^\beta_q\otimes\Hcal_{\bC}$ є простором регулярних
{\it узагальнених} або квадратично інтегровних функцій (тобто коли $\beta<0$ або
$\beta=0$ і $q\leq 0$), він є узагальненням стохастичного інтеграла Іто~\cite{K13}.

Легко бачити, що розширений стохастичний інтеграл можна визначити формулою
\eqref{e1_17} як лінійний неперервний оператор, що діє з простору
$(L^2)^\beta\otimes\Hcal_{\bC}
:=\mathop{\rm{pr\ lim}}_{q\to+\infty}(L^2)^\beta_q\otimes\Hcal_{\bC}$
в простір $(L^2)^\beta$, або з простору $(L^2)^{-\beta}\otimes\Hcal_{\bC}
:=\mathop{\rm{ind\ lim}}_{q\to+\infty}(L^2)^{-\beta}_{-q}
\otimes\Hcal_{\bC}$ в простір $(L^2)^{-\beta}$, тут $\beta\in [0,1]$.
До того ж за аналогією з~\eqref{e1_18} можна показати, що у випадку $\beta=-1$ розширений
стохастичний інтеграл є лінійним неперервним оператором, що діє з простору
$(L^2)^{-1}_q\otimes\Hcal_{\bC}$ в простір $(L^2)^{-1}_q$; а у випадку
$\beta\in (-1,1]$, як випливає з результатів~\cite{K13_2}, цей інтеграл можна інтерпретувати
як лінійний необмежений замкнений оператор, що діє з простору
$(L^2)^\beta_q\otimes\Hcal_{\bC}$ в простір $(L^2)^\beta_q$.

\begin{subremark}\label{r1_4_1}
В цій роботи нам не знадобляться стохастичні інтеграли за вимірними множинами, що
відрізняються від $\bR_+$, але такі інтеграли часто виникають у застосуваннях. Визначення
згаданих інтегралів можна дати у класичний спосіб: для довільного
$\Delta\in\mathcal B(\bR_+)$ покладемо
$\int_\Delta\circ (t)\widehat dL_t:=\int\circ (t)1_\Delta (t)\widehat dL_t$. Зацікавлений
читач може знайти детальну інформацію про такі інтеграли, зокрема, у~\cite{K13,K13_2,K21}.
\end{subremark}

\subsection{Стохастична похідна Хіди}\label{S1_5}
Опишемо конструкцію стохастичної похідної Хіди на просторах $(L^2)^\beta_q$, яка базується
на розкладі~\eqref{e1_5} (детальніше цей матеріал викладено у~\cite{K13,K13_2}).
Нехай $G^{(n)}\in\Hcal_{ext}^{(n)}$,
$n\in\bN$, $\dot g^{(n)}\in G^{(n)}$ -- представник
$G^{(n)}$. Розглянемо $\dot g^{(n)}(\cdot)$, тобто відділимо один аргумент
$\dot g^{(n)}$, та визначимо $G^{(n)}(\cdot)\in\Hcal_{ext}^{(n-1)}
\otimes\Hcal_{\bC}$ як клас еквівалентності у
$\Hcal_{ext}^{(n-1)}\otimes\Hcal_{\bC}$, породжений функцією
$\dot g^{(n)}(\cdot)$ (тобто $\dot g^{(n)}(\cdot)\in G^{(n)}(\cdot)$).
\begin{sublemma}
Для кожного $G^{(n)}\in\Hcal_{ext}^{(n)}$, $n\in\bN$, елемент
$G^{(n)}(\cdot)\in\Hcal_{ext}^{(n-1)}\otimes\Hcal_{\bC}$
визначений коректно (зокрема, $G^{(n)}(\cdot)$ не залежить від вибору представника
$\dot g^{(n)}\in G^{(n)}$) та
\begin{equation}\label{e1_20}
|G^{(n)}(\cdot)|_{\Hcal_{ext}^{(n-1)}\otimes\Hcal_{\bC}}
\leq |G^{(n)}|_{\Hcal_{ext}^{(n)}}.
\end{equation}
\end{sublemma}
Доведення цього твердження співпадає з точністю до очевидних модифікацій із
доведенням відповідного результату у~\cite{K13}.
\begin{subremark}\label{r1_5_1}
Варто відзначити, що, не зважаючи на оцінку~\eqref{e1_20}, простір
$\Hcal_{ext}^{(n)}$, $n\in\bN\backslash\{1\}$, не є підпростором простору
$\Hcal_{ext}^{(n-1)}\otimes\Hcal_{\bC}$, оскільки різні елементи
$\Hcal_{ext}^{(n)}$ можуть співпадати у
$\Hcal_{ext}^{(n-1)}\otimes\Hcal_{\bC}$, тобто представники різних
класів еквівалентності у $\Hcal_{ext}^{(n)}$ можуть потрапляти у один і той
самий клас еквівалентності у $\Hcal_{ext}^{(n-1)}\otimes\Hcal_{\bC}$
(а тому, зокрема, неможливо проектувати елементи
$\Hcal_{ext}^{(n-1)}\otimes\Hcal_{\bC}$ на $\Hcal_{ext}^{(n)}$ та
будувати ядра розкладу~\eqref{e1_17} розширеного стохастичного інтеграла за класичною схемою).
\end{subremark}
\begin{subdefinition}
Для $G\in (L^2)^\beta_{q+1}$ визначимо стохастичну похідну Хіди
$\partial_\cdot G\in (L^2)^\beta_q\otimes\Hcal_{\bC}$, поклавши
\begin{equation}\label{e1_21}
\partial_\cdot G:=\sum_{n=1}^\infty n{:\!\langle}\circ^{\otimes n-1},G^{(n)}(\cdot){\rangle\!:},
\end{equation}
де $G^{(n)}\in\Hcal_{ext}^{(n)}$ -- ядра з розкладу~\eqref{e1_5} для $G$, які розуміються
як елементи $\Hcal_{ext}^{(n-1)}\otimes\Hcal_{\bC}$ (в описаному вище сенсі).
\end{subdefinition}
Оскільки (див.~\eqref{e1_21},~\eqref{e1_12},~\eqref{e1_20} та~\eqref{e1_8})
\begin{equation}\label{e1_22}
\begin{gathered}
\|\partial_\cdot G\|_{(L^2)^\beta_q\otimes\Hcal_{\bC}}^2
=\sum_{n=1}^\infty ((n-1)!)^{1+\beta}n^22^{q(n-1)}|G^{(n)}(\cdot)
|_{\Hcal_{ext}^{(n-1)}\otimes\Hcal_{\bC}}^2\leq\\
\leq\sum_{n=1}^\infty (n!)^{1+\beta}2^{(q+1)n}[n^{1-\beta}2^{-(n+q)}]
|G^{(n)}|_{\Hcal_{ext}^{(n)}}^2\leq\\
\leq \max\limits_{n\in\bN}[n^{1-\beta}2^{-(n+q)}]\|G\|_{(L^2)^\beta_{q+1}}^2,
\end{gathered}
\end{equation}
це визначення є коректним, а похідна
\begin{equation}\label{e1_23}
\partial_\cdot : (L^2)^\beta_{q+1}\to (L^2)^\beta_q\otimes\Hcal_{\bC}
\end{equation}
є лінійним {\it обмеженим}, а тому і {\it неперервним} оператором. Зрозуміло, що, як і
розширений стохастичний інтеграл, стохастичну похідну Хіди можна визначити формулою~\eqref{e1_21}
як лінійний неперервний оператор, що діє з простору $(L^2)^\beta$ в простір
$(L^2)^\beta\otimes\Hcal_{\bC}$ ($\beta\in [-1,1]$).
До того ж за аналогією з~\eqref{e1_22} можна показати, що у випадку $\beta=1$ стохастична
похідна Хіди є лінійним неперервним оператором, що діє з простору $(L^2)^1_q$ в простір
$(L^2)^1_q\otimes\Hcal_{\bC}$; а у випадку $\beta\in [-1,1)$, як випливає з
результатів~\cite{K13_2}, цю похідну можна інтерпретувати як лінійний необмежений замкнений
оператор, що діє з простору $(L^2)^\beta_q$ в простір
$(L^2)^\beta_q\otimes\Hcal_{\bC}$.

В наступному твердженні описано зв'язок між розширеним стохастичним інтегралом та стохастичною
похідною Хіди.
\begin{subtheorem}\label{t1_5_1}
Розширений стохастичний інтеграл
$\int\circ\widehat d L:
(L^2)^{-\beta}_{-q}\otimes\Hcal_{\bC}\to (L^2)^{-\beta}_{-q-1}$
та стохастична похідна Хіди~\eqref{e1_23} є взаємно спряженими операторами:
\begin{equation}\label{e1_24}
\int\circ\widehat d L=\big(\partial_\cdot\big)^*,
\quad
\partial_\cdot=\big(\int\circ\widehat d L\big)^*,
\end{equation}
тобто для довільних $F\in (L^2)^{-\beta}_{-q}\otimes\Hcal_{\bC}$ та
$G\in (L^2)^\beta_{q+1}$
\begin{equation}\label{e1_25}
\langle\!\langle\int F(t)\widehat d L_t,G\rangle\!\rangle_{(L^2)}
=\langle\!\langle F(\cdot),\partial_\cdot G\rangle\!\rangle_{(L^2)\otimes\Hcal_{\bC}}.
\end{equation}
\end{subtheorem}
Доведення зводиться до встановлення рівності~\eqref{e1_25}, яке проводиться так само,
як і для інтеграла та похідної на просторах $(L^2)\otimes\Hcal_{\bC}$ та
$(L^2)$ відповідно, див.~\cite{K13}.

Відзначимо, що результат Теореми~\ref{t1_5_1} тривіальним чином розповсюджується на випадок
граничних просторів, тобто коли стохастичні інтеграл та похідна визначені відповідно на
$(L^2)^{-\beta}\otimes\Hcal_{\bC}$ та $(L^2)^\beta$ ($\beta\in [-1,1]$).
Ясно також, що рівності~\eqref{e1_24} можуть використовуватись як альтернативні визначення
розширеного стохастичного інтеграла та стохастичної похідної Хіди.

\begin{subremark}
Результат Теореми~\ref{t1_5_1} залишається справедливим для розширеного стохастичного інтеграла
$\int\circ (t)\widehat d L_t:
(L^2)^{-\beta}_{-q}\otimes\Hcal_{\bC}\to (L^2)^{-\beta}_{-q}$ та стохастичної
похідної Хіди $\partial_\cdot : (L^2)^\beta_q\to (L^2)^\beta_q\otimes\Hcal_{\bC}$
\cite{K13_2} (див. також~\cite{K13}). З цього результату випливає, зокрема, замкненість
згаданих операторів.
\end{subremark}

Насамкінець зауважимо, що у випадках, коли замість розширеного стохастичного інтеграла
$\int\circ (t)\widehat d L_t$ розглядається інтеграл $\int_\Delta\circ (t)\widehat d L_t
=\int\circ (t)1_\Delta (t)\widehat dL_t$, $\Delta\in\mathcal B(\bR_+)$ (див.
Зауваження~\ref{r1_4_1}), відповідним спряженим оператором є стохастична похідна Хіди
$1_\Delta (\cdot)\partial_\cdot$, це тривіальним чином випливає з рівності~\eqref{e1_25}.


\section{Формули типу Кларка"=Окона та суміжні питання}

Нехай $\mathcal F_t$, $t\in\bR_+$, -- поповнення відносно міри білого шуму Леві
\mbox{$\sigma$"=алгебри} $\sigma (L_u:u\leq t)$, породженої процесом Леві $L$ до моменту
часу $t$. Для $F\in (L^2)^\beta_q$, $\beta\in [-1,0)$ та $q\in\bZ$, або $\beta=0$
та $-q\in\bN$ (тобто для узагальнених випадкових величин), визначимо математичне
сподівання $\bfE$ та умовне математичне сподівання
$\bfE\big(\circ|_{\mathcal F_t}\big)$, поклавши
\begin{gather*}
\bfE{F}:=\langle\!\langle F,1\rangle\!\rangle_{(L^2)}=F^{(0)}\in\bC,\\
\bfE\big(F|_{\mathcal F_t}\big):=F^{(0)}
+\sum_{n=1}^\infty{:\!\langle}\circ^{\otimes n},F^{(n)}1_{[0,t)^n}{\rangle\!:}
\in (L^2)^\beta_q,
\end{gather*}
де $F^{(n)}\in\Hcal_{ext}^{(n)}$ -- ядра з розкладу~\eqref{e1_5} для $F$.
Якщо $F\in (L^2)\subset (L^2)^\beta_q$, то, як легко бачити, $\bfE{F}$ є звичайним
математичним сподіванням $F$; і цілком аналогічно доведенню Теореми~4.2 в~\cite{KT09} можна
показати, що $\bfE\big(F|_{\mathcal F_t}\big)$ є умовним математичним сподіванням $F$
відносно $\mathcal F_t$. Спираючись на це визначення, для
$G\in (L^2)^\beta_q\otimes\Hcal_{\bC}$ природно покласти
\begin{equation}\label{e2_1}
\bfE\big(G(\cdot)|_{\mathcal F_\cdot}\big):=G^{(0)}_\cdot
+\sum_{n=1}^\infty{:\!\langle}\circ^{\otimes n},G^{(n)}_\cdot 1_{[0,\cdot)^n}{\rangle\!:}
\in (L^2)^\beta_q\otimes\Hcal_{\bC},
\end{equation}
де $G^{(n)}_\cdot\in\Hcal_{ext}^{(n)}\otimes\Hcal_{\bC}$ -- ядра
з розкладу~\eqref{e1_11} для $G$. Зрозуміло, що $G^{(n)}_\cdot 1_{[0,\cdot)^n}
\in\Hcal_{ext}^{(n)}\otimes\Hcal_{\bC}$ та
$|G^{(n)}_\cdot 1_{[0,\cdot)^n}|_{\Hcal_{ext}^{(n)}\otimes\Hcal_{\bC}}
\leq|G^{(n)}_\cdot|_{\Hcal_{ext}^{(n)}\otimes\Hcal_{\bC}}$, а тому
${\bfE\big(\circ (\cdot)|_{\mathcal F_\cdot}\big)}$ є лінійним неперервним оператором в
$(L^2)^\beta_q\otimes\Hcal_{\bC}$.

\subsection{Формула Кларка"=Окона в найпростішому частинному випадку}\label{S2_1}
Як і при описі конструкції розширеного стохастичного інтеграла, розглянемо спочатку найпростіший
частинний випадок, в якому формула Кларка"=Окона приймає класичний вигляд.
\begin{subproposition}\label{p2_1_1}
Нехай $F\in (L^2)^\beta_q$ є таким, що ядра $F^{(n)}$, $n\in\bZ_+$, з розкладу
\eqref{e1_5} належать просторам
$\Hcal_{\bC}^{\widehat\otimes n}\subset\Hcal_{ext}^{(n)}$
(див. Підрозділ~\ref{S1_2}). Тоді
\begin{equation}\label{e2_2}
F=\bfE{F}+\int\bfE\big(\partial_t F|_{\mathcal F_t}\big)\widehat dL_t
\end{equation}
(пор. з~\eqref{e0_2}).
\end{subproposition}
\begin{proof}
Використовуючи~\eqref{e1_21},~\eqref{e2_1} та~\eqref{e1_14}, нескладно прийти до висновку,
що представлення~\eqref{e2_2} справедливе, якщо для кожного
$n\in\bN\backslash\{1\}$ і для кожного
$F^{(n)}\in\Hcal_{\bC}^{\widehat\otimes n}$
\begin{equation*}
nPr\big(F^{(n)}(\cdot_1,\dots,\cdot_{n-1},\cdot_n)
1_{[0,\cdot_n)^{n-1}}(\cdot_1,\dots,\cdot_{n-1})\big)=F^{(n)}
\end{equation*}
у просторі $\Hcal_{\bC}^{\widehat\otimes n}$, тут і нижче $Pr$ -- оператор
симетризації за всіма змінними. Але ця рівність виконується у вказаному просторі, оскільки
$F^{(n)}$ є симетричною функцією (точніше, клас еквівалентності $F^{(n)}$ у просторі
$\Hcal_{\bC}^{\widehat\otimes n}$ містить симетричну функцію-представника),
для {\it різних} $t_1,\dots,t_n$ $Pr1_{[0,t_n)^{n-1}}(t_1,\dots,t_{n-1})=\frac{1}{n}$, а
іншими випадками можна знехтувати через неатомарність міри Лебега.
\end{proof}

\begin{subremark}
Нехай $F\in (L^2)^\beta_q$ задовольняє умову Твердження~\ref{p2_1_1}, а також є регулярною
основною ($\beta>0$ або $\beta=0$ та $q\in\bN$), або квадратично інтегровною
($\beta=q=0$, $F\in (L^2)$) та диференційовною за Хідою
($\partial_\cdot F\in (L^2)\otimes\Hcal_{\bC}$) функцією. Тоді
$\bfE\big(\partial_\cdot F|_{\mathcal F_\cdot}\big)$ є інтегровним за Іто випадковим
процесом, а тому у представленні~\eqref{e2_2} можна використовувати стохастичний інтеграл Іто
(який співпадає в зазначених випадках з розширеним стохастичним інтегралом).
\end{subremark}

У загальному випадку представлення~\eqref{e2_2} не може бути справедливим хоча б тому, що
не кожне $F\in (L^2)^\beta_q$ можна представити у вигляді
\begin{equation}\label{e2_3}
F=\bfE{F}+\int G(t)\widehat dL_t,
\end{equation}
де $G$ -- бодай формальний ряд вигляду~\eqref{e1_11} (див. Теорему~\ref{t2_2_1} нижче).
Але навіть якщо $F$ є таким, що його можна подати у вигляді~\eqref{e2_3}, рівність~\eqref{e2_2}
однаково може не виконуватись. Нехай, наприклад,
$F={:\!\langle}\circ^{\otimes 3},F^{(3)}{\rangle\!:}$,
$F^{(3)}\in\Hcal_{ext}^{(3)}$. Тоді $\bfE{F}=0$ та, як неважко підрахувати за допомогою
\eqref{e1_21},~\eqref{e2_1} та~\eqref{e1_17},
\begin{gather*}
\int\bfE\big(\partial_t F|_{\mathcal F_t}\big)\widehat dL_t=\\
={:\!\langle}\circ^{\otimes 3},F^{(3)}(\cdot_1,\cdot_2,\cdot_3)
\big(1_{[0,\cdot_3)^2}(\cdot_1,\cdot_2)+1_{[0,\cdot_2)^2}(\cdot_3,\cdot_1)
+1_{[0,\cdot_1)^2}(\cdot_2,\cdot_3)\big){\rangle\!:};
\end{gather*}
а тому, використовуючи~\eqref{e1_8} та~\eqref{e1_7}, отримуємо
\begin{gather*}
\big\|F-\int\bfE\big(\partial_t F|_{\mathcal F_t}\big)\widehat dL_t\big\|_{(L^2)^\beta_q}^2
=6^{1+\beta}\cdot 8^q\times\\
\times\big|F^{(3)}(\cdot_1,\cdot_2,\cdot_3)
\big(1-[1_{[0,\cdot_3)^2}(\cdot_1,\cdot_2)+1_{[0,\cdot_2)^2}(\cdot_3,\cdot_1)
+1_{[0,\cdot_1)^2}(\cdot_2,\cdot_3)]\big)\big|_{ext}^2=\\
=9\cdot 6^\beta\cdot 8^q\|p_2\|_\nu^2
\int_{\bR_+^2}|F^{(3)}(t_1,t_1,t_2)|^21_{\{t_1\geq t_2\}}dt_1dt_2+\\
+6^\beta\cdot 8^q\|p_3\|_\nu^2
\int_{\bR_+}|F^{(3)}(t_1,t_1,t_1)|^2dt_1.
\end{gather*}
Якщо $F^{(3)}$ є таким, що $\int_{\bR_+}|F^{(3)}(t_1,t_1,t_1)|^2dt_1=0$, то
${:\!\langle}\circ^{\otimes 3},F^{(3)}{\rangle\!:}$ можна представити у вигляді~\eqref{e2_3}
(див. Теорему~\ref{t2_2_1} нижче); але якщо при цьому
$\int_{\bR_+^2}|F^{(3)}(t_1,t_1,t_2)|^21_{\{t_1\geq t_2\}}dt_1dt_2\not=0$, то
$F\not=\int\bfE\big(\partial_t F|_{\mathcal F_t}\big)\widehat dL_t$, тобто
${:\!\langle}\circ^{\otimes 3},F^{(3)}{\rangle\!:}$ не можна представити у вигляді~\eqref{e2_2}.

У наступних підрозділах ми встановимо необхідну і достатню умову, за якої $F\in (L^2)^\beta_q$
можна представити у вигляді~\eqref{e2_3}; отримаємо формули типу Кларка"=Окона для $F$;
а також з'ясуємо необхідну і достатню умову, за якої $F$ допускає представлення
у вигляді~\eqref{e2_2}.

\subsection{Належність випадкових величин області значень розширеного стохастичного інтеграла}
Розглянемо простий приклад. Нехай $F={:\!\langle}\circ^{\otimes 2},F^{(2)}{\rangle\!:}$,
$F^{(2)}\in\Hcal_{ext}^{(2)}$. Зрозуміло, що якщо таке $F$ можна представити у вигляді
\eqref{e2_3}, то $G(\cdot)={:\!\langle}\circ,G^{(1)}_\cdot{\rangle\!:}$,
$G^{(1)}_\cdot\in\Hcal_{ext}^{(1)}\otimes\Hcal_{\bC}$,
та $F^{(2)}=\widehat G^{(1)}$ (див. Підрозділ~\ref{S1_4}). Оскільки за побудовою
$\widehat G^{(1)}$ містить представника $\widehat g^{(1)}$ такого, що для кожного
$t\in\bR_+$ $\widehat g^{(1)}(t,t)=0$, маємо {\it необхідну} умову, за якої
${:\!\langle}\circ^{\otimes 2},F^{(2)}{\rangle\!:}$ можна представити у вигляді~\eqref{e2_3}:
{\it $F^{(2)}$ має містити представника $f^{(2)}$ такого, що для кожного
$t\in\bR_+$ $f^{(2)}(t,t)=0$}. Більше того, легко бачити, що ця умова є й {\it достатньою}:
покладемо $G^{(1)}_\cdot:=F^{(2)}(\cdot)\in\Hcal_{ext}^{(1)}\otimes\Hcal_{\bC}$
(тобто $G^{(1)}_\cdot$ -- це ядро $F^{(2)}$, яке розуміється як елемент
простору $\Hcal_{ext}^{(1)}\otimes\Hcal_{\bC}$, див. Підрозділ~\ref{S1_5}),
тоді за виконання згаданої умови маємо $\widehat G^{(1)}=F^{(2)}$ (в якості представника
$G^{(1)}_\cdot$, що задовольніє умову~\eqref{e1_15}, можна взяти згадану вище функцію $f^{(2)}$),
а тому згідно з~\eqref{e1_17} $\int{:\!\langle}\circ,G^{(1)}_t{\rangle\!:}\widehat dL_t
={:\!\langle}\circ^{\otimes 2},F^{(2)}{\rangle\!:}=F$.

У загальному випадку ситуація, звісно, подібна: $F\in (L^2)^\beta_q$ можна представити у
вигляді~\eqref{e2_3} (таке представлення, взагалі кажучи, не є єдиним) якщо та лише якщо ядра
з розкладу~\eqref{e1_5} для $F$ мають властивості, притаманні ядрам з розкладу~\eqref{e1_17} для
розширених стохастичних інтегралів. Точніше, справедливе таке твердження.
\begin{subtheorem}\label{t2_2_1}
Нехай $F\in (L^2)^\beta_q$. Наступні твердження еквівалентні:
\begin{itemize}
\item[(1)] $F$ можна представити у вигляді~\eqref{e2_3}, де
$G\in (L^2)^\beta_q\otimes\Hcal_{\bC}$ у випадку $\beta\geq 0$, та
$G\in (L^2)^\beta_{q-1}\otimes\Hcal_{\bC}$, якщо $\beta<0$;
\item[(2)] для кожного $n\in\bN\backslash\{1\}$ ядро
$F^{(n)}\in\Hcal_{ext}^{(n)}$ з розкладу~\eqref{e1_5} для $F$ містить представника
$f^{(n)}$ такого, що $f^{(n)}(t_1,\dots,t_n)=0$, якщо для кожного $i\in\{1,\dots,n\}$
існує $j\in\{1,\dots,n\}\backslash\{i\}$ таке, що $t_i=t_j$.
\end{itemize}
\end{subtheorem}
\begin{proof}
Спочатку доведемо теорему для $F={:\!\langle}\circ^{\otimes n},F^{(n)}{\rangle\!:}$,
$F^{(n)}\in\Hcal_{ext}^{(n)}$, $n\in\bN\backslash\{1\}$ (випадки $n=0$ та $n=1$ є
тривіальними).

1) ("(1)"$\Rightarrow$"(2)") Нехай
${:\!\langle}\circ^{\otimes n},F^{(n)}{\rangle\!:}=\int G(t)\widehat dL_t$. Зрозуміло, що
зараз $G(\cdot)={:\!\langle}\circ^{\otimes n-1},G^{(n-1)}_\cdot{\rangle\!:}$,
$G^{(n-1)}_\cdot\in\Hcal_{ext}^{(n-1)}\otimes\Hcal_{\bC}$,
та $F^{(n)}=\widehat G^{(n-1)}$ (див.~\eqref{e1_17}), але елемент
$\widehat G^{(n-1)}\in\Hcal_{ext}^{(n)}$ задовольняє умову твердження (2) за побудовою
(в якості $f^{(n)}$ можна обрати функцію $\widehat g^{(n-1)}\in \widehat G^{(n-1)}$, що є
симетризацією представника $\dot g^{(n-1)}_\cdot\in G^{(n-1)}_\cdot$, який задовольняє умову
\eqref{e1_15}, див. Підрозділ~\ref{S1_4}).

2) ("(2)"$\Rightarrow$"(1)") Нехай $f^{(n)}\in F^{(n)}$ -- описаний у твердженні (2)
представник $F^{(n)}$. Не втрачаючи загальності можна вважати, що $f^{(n)}$ є симетричною
функцією. Покладемо
\begin{equation}\label{e2_4}
\begin{gathered}
h_n(t_1,\dots,t_n)
:=Pr 1_{\{t_1\not=t_n,t_2\not=t_n,\dots,t_{n-1}\not=t_n\}}=\\
=\frac{1}{n}
\big[1_{\{t_1\not=t_n,t_2\not=t_n,\dots,t_{n-1}\not=t_n\}}
+1_{\{t_n\not=t_{n-1},t_1\not=t_{n-1},\dots,t_{n-2}\not=t_{n-1}\}}+\\
+\cdots+1_{\{t_2\not=t_1,t_3\not=t_1,\dots,t_n\not=t_1\}}\big],
\end{gathered}
\end{equation}
\begin{equation}\label{e2_5}
g^{(n-1)}_t(t_1,\dots,t_{n-1}):=
\begin{cases}
\frac{f^{(n)}(t_1,\dots,t_{n-1},t)}{h_n(t_1,\dots,t_{n-1},t)},
{\operatorname{\ \text{якщо}\ }} h_n(t_1,\dots,t_{n-1},t)\not=0
\\
0,\ \ \qquad\qquad\quad
{\operatorname{\ \text{якщо}\ }} h_n(t_1,\dots,t_{n-1},t)=0
\end{cases}
\end{equation}
(зауважимо, що якщо $h_n(t_1,\dots,t_{n-1},t)=0$, то $f^{(n)}(t_1,\dots,t_{n-1},t)=0$ за
умовою твердження (2), а тому можна сказати, що $g^{(n-1)}_\cdot$ "зберігає всю інформацію"\ про
функцію $f^{(n)}$). Використовуючи~\eqref{e1_7}, неатомарність міри Лебега, та рівність
\begin{equation}\label{e2_6}
h_n(\underset{l_1}{\underbrace{t_1,\dots,t_1}},\dots,
\underset{l_k}{\underbrace{t_{s_1+\dots+s_k},\dots,t_{s_1+\dots+s_k}}},t)
=\frac{1}{n}1_{\{l_k>1\}}+\frac{s_k+1}{n}1_{\{l_k=1\}}
\end{equation}
для {\it різних} $t_1,\dots,t_{s_1+\dots+s_k},t$ (тут $k,l_\cdot,s_\cdot\in\bN$,
$l_1>\cdots>l_k$, $l_1s_1+\cdots+l_ks_k=n-1$), яка випливає безпосередньо з~\eqref{e2_4},
отримаємо
\begin{equation}\label{e2_7}
\begin{gathered}
|g^{(n-1)}_\cdot|_{\Hcal_{ext}^{(n-1)}\otimes\Hcal_{\bC}}^2=\\
=\sum_{\underset{l_1s_1+\cdots+l_ks_k=n-1}{k,l_j,s_j\in\bN:
\ j=1,\dots,k,\ l_1>\cdots>l_k,}}
\frac{(n-1)!}{s_1!\cdots s_k!}\Big(\frac{\|p_{l_1}\|_\nu}{l_1!}\Big)^{2s_1}
\cdots\Big(\frac{\|p_{l_k}\|_\nu}{l_k!}\Big)^{2s_k}\times\\
\times\int_{\bR_+^{s_1+\cdots+s_k+1}}|g^{(n-1)}_t(
\underset{l_1}{\underbrace{t_1,\dots,t_1}},\dots,
\underset{l_k}{\underbrace{t_{s_1+\cdots+s_k},\dots,t_{s_1+\cdots+s_k}}})|^2\times\\
\times dt_1\cdots dt_{s_1+\cdots+s_k}dt=\\
=n\sum_{\underset{l_1s_1+\cdots+l_ks_k+1=n}{k,l_j,s_j\in\bN:
\ j=1,\dots,k,\ l_1>\cdots>l_k>1,}}
\frac{n!}{s_1!\cdots s_k!}\Big(\frac{\|p_{l_1}\|_\nu}{l_1!}\Big)^{2s_1}
\cdots\Big(\frac{\|p_{l_k}\|_\nu}{l_k!}\Big)^{2s_k}\times\\
\times\int_{\bR_+^{s_1+\cdots+s_k+1}}|f^{(n)}(
\underset{l_1}{\underbrace{t_1,\dots,t_1}},\dots,
\underset{l_k}{\underbrace{t_{s_1+\cdots+s_k},\dots,
t_{s_1+\cdots+s_k}}},t)|^2\times\\
\times dt_1\cdots dt_{s_1+\cdots+s_k}dt+\\
+n\sum_{\underset{l_1s_1+\cdots+l_{k-1}s_{k-1}+s_k+1=n}{k,l_j,s_j\in\bN:
\ j=1,\dots,k,\ l_1>\cdots>l_k=1,}}\frac{n!}{s_1!\cdots (s_k+1)!(s_k+1)}\times\\
\times\Big(\frac{\|p_{l_1}\|_\nu}{l_1!}\Big)^{2s_1}
\cdots\Big(\frac{\|p_{l_{k-1}}\|_\nu}{l_{k-1}!}\Big)^{2s_{k-1}}\times\\
\times\int_{\bR_+^{s_1+\cdots+s_k+1}}|f^{(n)}(
\underset{l_1}{\underbrace{t_1,\dots,t_1}},\dots,
t_{s_1+\cdots+s_{k-1}+1},\dots,t_{s_1+\cdots+s_k},t)|^2\times\\
\times dt_1\cdots dt_{s_1+\cdots+s_k}dt
\leq n|F^{(n)}|_{\Hcal_{ext}^{(n)}}^2<\infty.
\end{gathered}
\end{equation}
Отже, функція $g^{(n-1)}_\cdot$ породжує елемент (клас еквівалентності)
$G^{(n-1)}_\cdot\in\Hcal_{ext}^{(n-1)}\otimes\Hcal_{\bC}$.
Покладемо
\begin{equation*}
\dot g^{(n-1)}_{\cdot_n}(\cdot_1,\dots,\cdot_{n-1})
:=g^{(n-1)}_{\cdot_n}(\cdot_1,\dots,\cdot_{n-1})
1_{\{\cdot_1\not=\cdot_n,\cdot_2\not=\cdot_n,\dots,\cdot_{n-1}\not=\cdot_n\}}
\in G^{(n-1)}_\cdot.
\end{equation*}
Зрозуміло, що ця функція задовольняє умову~\eqref{e1_15}. Враховуючи~\eqref{e2_4}
та~\eqref{e2_5}, отримуємо
\begin{gather*}
\widehat g^{(n-1)}(\cdot_1,\dots,\cdot_n)
=Pr\dot g^{(n-1)}_{\cdot_n}(\cdot_1,\dots,\cdot_{n-1})=\\
=g^{(n-1)}_{\cdot_n}(\cdot_1,\dots,\cdot_{n-1})h_n(\cdot_1,\dots,\cdot_n)
=f^{(n)}(\cdot_1,\dots,\cdot_n)\in F^{(n)},
\end{gather*}
оскільки $f^{(n)}$ -- симетрична функція, що задовольняє умову твердження (2)
(зауважимо, що якби функція $f^{(n)}$ не задовольняла б умову твердження (2), остання рівність
не мала б місця). З іншого боку, функція $\widehat g^{(n-1)}$ породжує ядро
$\widehat G^{(n-1)}\in\Hcal_{ext}^{(n)}$, яке використовується при визначенні розширеного
стохастичного інтеграла:
$\int{:\!\langle}\circ^{\otimes n-1},G^{(n-1)}_t{\rangle\!:}\widehat dL_t
={:\!\langle}\circ^{\otimes n},\widehat G^{(n-1)}{\rangle\!:}$
(див. Підрозділ~\ref{S1_4}). Отже, $\widehat G^{(n-1)}=F^{(n)}$ в
$\Hcal_{ext}^{(n)}$, а тому, поклавши
$G(\cdot):={:\!\langle}\circ^{\otimes n-1},G^{(n-1)}_\cdot{\rangle\!:}$, маємо
$F=\int G(t)\widehat dL_t$, тобто умова твердження (1) виконана.

У загальному випадку імплікація "(1)"$\Rightarrow$"(2)"\ тривіальним чином випливає
з~\eqref{e1_17} та відповідної імплікації у щойно розглянутому частинному випадку. Доведемо
імплікацію "(2)"$\Rightarrow$"(1)". Нехай
\begin{equation*}
G(\cdot):=\sum_{n=1}^\infty{:\!\langle}\circ^{\otimes n-1},G^{(n-1)}_\cdot{\rangle\!:},
\end{equation*}
де $G^{(n-1)}_\cdot\in\Hcal_{ext}^{(n-1)}\otimes\Hcal_{\bC}$ -- ядра,
побудовані вище для кожного $n\in\bN\backslash\{1\}$,
$G^{(0)}_\cdot:=F^{(1)}(\cdot)\in\Hcal_{ext}^{(1)}=\Hcal_{\bC}$.
Достатньо довести, що $G\in (L^2)^\beta_q\otimes\Hcal_{\bC}$ у випадку $\beta\geq 0$,
та $G\in (L^2)^\beta_{q-1}\otimes\Hcal_{\bC}$, якщо $\beta<0$, тоді рівність
\eqref{e2_3} випливатиме з~\eqref{e1_17} та відповідної імплікації у щойно розглянутому
частинному випадку. Використовуючи~\eqref{e1_12}, оцінку
\begin{equation*}
|G^{(n-1)}_\cdot|_{\Hcal_{ext}^{(n-1)}\otimes\Hcal_{\bC}}^2
\leq n|F^{(n)}|_{\Hcal_{ext}^{(n)}}^2
\end{equation*}
(див.~\eqref{e2_7}; випадок $n=1$ є тривіальним), та~\eqref{e1_8}, отримуємо для $\beta\geq 0$
\begin{gather*}
\|G\|_{(L^2)^\beta_q\otimes\Hcal_{\bC}}^2
=\sum_{n=1}^\infty ((n-1)!)^{1+\beta}2^{q(n-1)}
|G^{(n-1)}_\cdot|_{\Hcal_{ext}^{(n-1)}\otimes\Hcal_{\bC}}^2\leq\\
\leq 2^{-q}\sum_{n=1}^\infty (n!)^{1+\beta}2^{qn}n^{-\beta}
|F^{(n)}|_{\Hcal_{ext}^{(n)}}^2
\leq 2^{-q}\|F\|_{(L^2)^\beta_q}^2<\infty,
\end{gather*}
та для $\beta<0$
\begin{gather*}
\|G\|_{(L^2)^\beta_{q-1}\otimes\Hcal_{\bC}}^2
=\sum_{n=1}^\infty ((n-1)!)^{1+\beta}2^{(q-1)(n-1)}
|G^{(n-1)}_\cdot|_{\Hcal_{ext}^{(n-1)}\otimes\Hcal_{\bC}}^2\leq\\
\leq 2^{1-q}\sum_{n=1}^\infty (n!)^{1+\beta}2^{qn}[2^{-n}n^{-\beta}]
|F^{(n)}|_{\Hcal_{ext}^{(n)}}^2\leq\\
\leq 2^{1-q}\max_{n\in\bN}[2^{-n}n^{-\beta}]\|F\|_{(L^2)^\beta_q}^2<\infty,
\end{gather*}
звідки й випливає потрібне.
\end{proof}

\begin{subremark}\label{r2_2_1}
Нехай випадкову величину $F\in (L^2)^\beta_q$ можна {\it формально}
представити у вигляді $F=\bfE{F}+\int\mathcal G(t)\widehat dL_t$ (пор. з~\eqref{e2_3}), де
\begin{equation*}
\mathcal G(\cdot):=\sum_{n=1}^\infty{:\!\langle}\circ^{\otimes n-1},
\mathcal G^{(n-1)}_\cdot{\rangle\!:},
\end{equation*}
$\mathcal G^{(n-1)}_\cdot\in\Hcal_{ext}^{(n-1)}\otimes\Hcal_{\bC}$, --
{\it формальний} ряд, та
\begin{equation*}
\int\mathcal G(t)\widehat dL_t
:=\sum_{n=1}^\infty{:\!\langle}\circ^{\otimes n},\widehat{\mathcal G}^{(n-1)}{\rangle\!:}
\end{equation*}
(пор. з~\eqref{e1_17}) -- {\it формальний} стохастичний інтеграл. Тоді для кожного ядра
$F^{(n)}\in\Hcal_{ext}^{(n)}$ ($n\in\bN$) з розкладу~\eqref{e1_5} для $F$ маємо
$F^{(n)}=\widehat{\mathcal G}^{(n-1)}$, а тому $F$ задовольняє умову твердження (2)
Теореми~\ref{t2_2_1}, відтак $F$ можна представити у вигляді~\eqref{e2_3} з
$G\in (L^2)^\beta_q\otimes\Hcal_{\bC}$ ($\beta\geq 0$) або
$G\in (L^2)^\beta_{q-1}\otimes\Hcal_{\bC}$ ($\beta<0$).
\end{subremark}

Насамкінець відзначимо, що якщо $F={:\!\langle}\circ^{\otimes n},F^{(n)}{\rangle\!:}$,
$F^{(n)}\in\Hcal_{ext}^{(n)}$, $n\in\bN\backslash\{1\}$, неможливо представити
у вигляді~\eqref{e2_3}, однаково можна побудувати ядра
$G^{(n-1)}_\cdot\in\Hcal_{ext}^{(n-1)}\otimes\Hcal_{\bC}$ за функціями
\eqref{e2_5}. Але в цьому випадку $\widehat G^{(n-1)}\not=F^{(n)}$ в $\Hcal_{ext}^{(n)}$
(зараз $|\widehat G^{(n-1)}-F^{(n)}|_{ext}$ містить інтеграли за сім'ями аргументів, для яких
функція $h_n$, визначена у~\eqref{e2_4}, дорівнює нулю) та
$|\widehat G^{(n-1)}|_{ext}<|F^{(n)}|_{ext}$ (доведення цього факту залишимо зацікавленому
читачу).

\subsection{Найпростіша формула типу Кларка"=Окона в загальному випадку}
Почнемо з певної підготовки. Для $n\in\bN\backslash\{1\}$ та
$t_1,\dots,t_n\in\bR_+$ визначимо
\begin{equation}\label{e2_8}
\hbar_n(t_1,\dots,t_n):=nh_n(t_1,\dots,t_n),
\end{equation}
де функції $h_n$ визначені формулою~\eqref{e2_4}; покладемо також $\hbar_1\equiv 1$.
Далі, для $G^{(n)}_\cdot\in\Hcal_{ext}^{(n)}\otimes\Hcal_{\bC}$,
$n\in\bZ_+$, покладемо
\begin{equation}\label{e2_9}
\widetilde G^{(n)}_\cdot(\cdot_1,\dots,\cdot_n)
:=\begin{cases}
\frac{G^{(n)}_\cdot(\cdot_1,\dots,\cdot_n)}
{\hbar_{n+1}(\cdot_1,\dots,\cdot_n,\cdot)},{\operatorname{\ \text{якщо}\ }}
\hbar_{n+1}(\cdot_1,\dots,\cdot_n,\cdot)\not=0
\\
0,\qquad\qquad{\operatorname{\ \ \ \text{якщо}\ }}
\hbar_{n+1}(\cdot_1,\dots,\cdot_n,\cdot)=0
\end{cases}.
\end{equation}
Легко бачити, що $\widetilde G^{(n)}_\cdot
\in\Hcal_{ext}^{(n)}\otimes\Hcal_{\bC}$ та
\begin{equation}\label{e2_10}
|\widetilde G^{(n)}_\cdot|_{\Hcal_{ext}^{(n)}\otimes\Hcal_{\bC}}
\leq |G^{(n)}_\cdot|_{\Hcal_{ext}^{(n)}\otimes\Hcal_{\bC}}.
\end{equation}

Для $G\in (L^2)^\beta_q\otimes\Hcal_{\bC}$ визначимо
\begin{equation}\label{e2_11}
(AG)(\cdot):=\sum_{n=0}^\infty
{:\!\langle}\circ^{\otimes n},\widetilde G^{(n)}_\cdot{\rangle\!:},
\end{equation}
де ядра $\widetilde G^{(n)}_\cdot$ побудовані по ядрам $G^{(n)}_\cdot$ з розкладу~\eqref{e1_11}
для $G$. З оцінки~\eqref{e2_10} випливає, що  $A$ є лінійним {\it неперервним} оператором
в $(L^2)^\beta_q\otimes\Hcal_{\bC}$.
\begin{subproposition}\label{p2_3_1}
Нехай $F\in (L^2)^\beta_q$. Тоді для $\beta\geq 0$
$A\partial_\cdot F\in (L^2)^\beta_q\otimes\Hcal_{\bC}$, та для
$\beta<0$ $A\partial_\cdot F\in (L^2)^\beta_{q-1}\otimes\Hcal_{\bC}$,
тут $\partial_\cdot$ -- похідна Хіди~\eqref{e1_23}.
\end{subproposition}
\begin{proof}
У випадку $\beta<0$ результат твердження випливає безпосередньо з властивостей операторів
$\partial_\cdot$ та $A$. Розглянемо випадок $\beta\geq 0$. З~\eqref{e1_21} та~\eqref{e2_11}
випливає, що
\begin{equation*}
A\partial_\cdot F=\sum_{n=1}^\infty n{:\!\langle}\circ^{\otimes n-1},
\widetilde F^{(n)}(\cdot){\rangle\!:},
\end{equation*}
де ядра $\widetilde F^{(n)}(\cdot)\in\Hcal_{ext}^{(n-1)}\otimes\Hcal_{\bC}$
побудовані по ядрам $F^{(n)}$ з розкладу~\eqref{e1_5} для $F$, які розуміються як елементи
просторів $\Hcal_{ext}^{(n-1)}\otimes\Hcal_{\bC}$ (див. Підрозділ~\ref{S1_5}).
Використовуючи~\eqref{e1_7}, неатомарність міри Лебега,~\eqref{e2_6} та~\eqref{e2_8}, подібно
до викладки~\eqref{e2_7} можна встановити, що
\begin{equation}\label{e2_19}
|n\widetilde F^{(n)}(\cdot)|_{\Hcal_{ext}^{(n-1)}\otimes\Hcal_{\bC}}^2
\leq n|F^{(n)}|_{\Hcal_{ext}^{(n)}}^2.
\end{equation}
Скориставшись~\eqref{e1_12}, цією оцінкою та~\eqref{e1_8}, отримуємо
\begin{gather*}
\|A\partial_\cdot F\|_{(L^2)^\beta_q\otimes\Hcal_{\bC}}^2
=\sum_{n=1}^\infty ((n-1)!)^{1+\beta}2^{q(n-1)}
|n\widetilde F^{(n)}(\cdot)
|_{\Hcal_{ext}^{(n-1)}\otimes\Hcal_{\bC}}^2\leq\\
\leq 2^{-q}\sum_{n=1}^\infty (n!)^{1+\beta}2^{qn}n^{-\beta}
|F^{(n)}|_{\Hcal_{ext}^{(n)}}^2
\leq 2^{-q}\|F\|_{(L^2)^\beta_q}^2<\infty,
\end{gather*}
звідки й випливає потрібне.
\end{proof}

Сформулюємо й доведемо основний результат підрозділу.
\begin{subtheorem}\label{t2_3_1}
Нехай випадкова величина $F\in (L^2)^\beta_q$ може бути представлена у вигляді~\eqref{e2_3}
(див. Теорему~\ref{t2_2_1}). Тоді має місце представлення (формула типу Кларка"=Окона)
\begin{equation}\label{e2_12}
F=\bfE{F}+\int A\partial_t F\widehat dL_t
\end{equation}
(пор. з~\eqref{e2_2}).
\end{subtheorem}
\begin{proof}
Спочатку доведемо теорему для $F={:\!\langle}\circ^{\otimes n},F^{(n)}{\rangle\!:}$,
$F^{(n)}\in\Hcal_{ext}^{(n)}$, $n\in\bN\backslash\{1\}$ (випадки $n=0$ та $n=1$ є
тривіальними). Задля спрощення позначень приймемо за визначенням $\frac{0}{0}:=0$.
Використовуючи~\eqref{e1_21},~\eqref{e2_11},~\eqref{e2_9} та~\eqref{e2_8}, отримуємо
\begin{gather*}
A\partial_\cdot{:\!\langle}\circ^{\otimes n},F^{(n)}{\rangle\!:}
=n{:\!\langle}\circ^{\otimes n-1},\widetilde F^{(n)}(\cdot){\rangle\!:}=\\
=n{:\!\langle}\circ^{\otimes n-1},\frac{f^{(n)}(\cdot_1,\dots,\cdot_{n-1},\cdot)}
{\hbar_n(\cdot_1,\dots,\cdot_{n-1},\cdot)}{\rangle\!:}
={:\!\langle}\circ^{\otimes n-1},\frac{f^{(n)}(\cdot_1,\dots,\cdot_{n-1},\cdot)}
{h_n(\cdot_1,\dots,\cdot_{n-1},\cdot)}{\rangle\!:},
\end{gather*}
де $f^{(n)}\in F^{(n)}\in\Hcal_{ext}^{(n)}$ -- симетрична функція, описана у
твердженні (2) Теореми~\ref{t2_2_1} (нагадаємо, що якщо для деякого набору аргументів
$t_1,\dots,t_{n-1},t\in\bR_+$ $h_n(t_1,\dots,t_{n-1},t)=\hbar_n(t_1,\dots,t_{n-1},t)=0$,
то $f^{(n)}(t_1,\dots,t_{n-1},t)=0$). Але за побудовою ядер розширеного стохастичного
інтеграла (див. Підрозділ~\ref{S1_4}) $\widehat{\frac{f^{(n)}}{h_n}}=f^{(n)}\in F^{(n)}$,
звідки маємо
$
\int (A\partial_\cdot{:\!\langle}\circ^{\otimes n},F^{(n)}{\rangle\!:})(t)\widehat dL_t
\equiv\int A\partial_t{:\!\langle}\circ^{\otimes n},F^{(n)}{\rangle\!:}\widehat dL_t
={:\!\langle}\circ^{\otimes n},F^{(n)}{\rangle\!:}
$,
що і треба було довести.

Твердження теореми в загальному випадку випливає з Твердження~\ref{p2_3_1},~\eqref{e1_17}
та щойно доведеного результату.
\end{proof}

\begin{subremark}\label{r2_3_1}
Нехай $F\in (L^2)^\beta_q$ можна представити у вигляді~\eqref{e2_3}.
З Зауваження~\ref{r2_2_1}, Твердження~\ref{p2_3_1} та представлення~\eqref{e2_12} випливає, що в
якості підінтегральної функції $G(\cdot)$ можна обрати $A\partial_\cdot F$ (насправді саме у
вигляді $A\partial_\cdot F$, хоч і в інших позначеннях, $G(\cdot)$ було побудовано при доведенні
Теореми~\ref{t2_2_1}).
\end{subremark}

\subsection{Прямий аналог формули Кларка"=Окона в загальному випадку}
Конструкція підінтегральної функції у формулі типу Кларка"=Окона~\eqref{e2_12} є відносно
простою, але ця формула не є безпосереднім узагальненням класичної формули Кларка"=Окона
\eqref{e2_2}. Справді, нехай $F\in (L^2)^\beta_q$ задовольняє умову Твердження~\ref{p2_1_1}.
Використовуючи~\eqref{e1_21},~\eqref{e2_11},~\eqref{e2_9},~\eqref{e2_8},~\eqref{e2_4} та
\eqref{e2_1}, неважко показати, що в цьому випадку
\begin{equation*}
A\partial_\cdot F=\sum_{n=1}^\infty{:\!\langle}\circ^{\otimes n-1},F^{(n)}(\cdot){\rangle\!:},
\end{equation*}
в той час як
\begin{equation*}
\bfE\big(\partial_\cdot F|_{\mathcal F_\cdot}\big)
=\sum_{n=1}^\infty n{:\!\langle}\circ^{\otimes n-1},F^{(n)}(\cdot)1_{[0,\cdot)^{n-1}}{\rangle\!:}
\not=A\partial_\cdot F,
\end{equation*}
взагалі кажучи (тут $F^{(n)}$, $n\in\bN$, -- ядра з розкладу~\eqref{e1_5} для $F$, які
розуміються як елементи просторів $\Hcal_{ext}^{(n-1)}\otimes\Hcal_{\bC}$,
див. Підрозділ~\ref{S1_5}). Подібна ситуація має місце і в гауссівському аналізі,
див.~\cite{K11c}.

Отримаємо прямий аналог, тобто безпосереднє узагальнення формули Кларка"=Окона~\eqref{e2_2}
на випадок, коли випадкову величину $F$ можна представити у вигляді~\eqref{e2_3}, але
умова Твердження~\ref{p2_1_1} не виконана. Для $n\in\bN$ та
$t_1,\dots,t_n,t\in\bR_+$ покладемо
\begin{equation}\label{e2_13}
\begin{gathered}
\chi_{n,t}(t_1,\dots,t_n):=\\
:=\begin{cases}
0,{\operatorname{\ \text{якщо}\ }}
\exists i\in\{1,\dots,n\}:t_i\geq t
{\operatorname{\ \text{та}\ }} \forall j\in\{1,\dots,n\}
\backslash\{i\}\ t_i\not=t_j
\\
1, {\operatorname{\ \text{в}\ \text{інших}\ \text{випадках}}}
\end{cases},
\end{gathered}
\end{equation}
тобто $\chi_{n,t}(t_1,\dots,t_n)=0$, якщо існує $t_i$ кратності $1$ (себто $t_i$ не дорівнює
жодному іншому $t_j$, $j\not=i$), більше за $t$ або рівне $t$. Наприклад,
$\chi_{3,4}(7,7,2)=1$ ($2<4$, $7$ має кратність $2$),
$\chi_{3,4}(5,5,5)=\chi_{3,4}(2,2,2)=1$ (відсутні аргументи кратності $1$),
$\chi_{3,4}(7,2,2)=0$ ($7>4$, $7$ має кратність $1$), $\chi_{3,4}(4,2,2)=0$ (є однократний
аргумент, що дорівнює $4$). Покладемо також $\chi_{0,\cdot}\equiv 1$. Для
$G\in (L^2)^\beta_q\otimes\Hcal_{\bC}$ визначимо
\begin{equation}\label{e2_14}
(\bfE_\cdot G)(\cdot):=\sum_{n=0}^\infty
{:\!\langle}\circ^{\otimes n},G^{(n)}_\cdot\chi_{n,\cdot}{\rangle\!:}
\equiv G^{(0)}_\cdot+\sum_{n=1}^\infty
{:\!\langle}\circ^{\otimes n},G^{(n)}_\cdot\chi_{n,\cdot}{\rangle\!:}
\in (L^2)^\beta_q\otimes\Hcal_{\bC}
\end{equation}
(пор. з~\eqref{e2_1}), де
$G^{(n)}_\cdot\in\Hcal_{ext}^{(n)}\otimes\Hcal_{\bC}$ -- ядра з розкладу
\eqref{e1_11} для $G$. Зрозуміло, що
$G^{(n)}_\cdot\chi_{n,\cdot}\in\Hcal_{ext}^{(n)}\otimes\Hcal_{\bC}$,
та
$|G^{(n)}_\cdot\chi_{n,\cdot}|_{\Hcal_{ext}^{(n)}\otimes\Hcal_{\bC}}
\leq |G^{(n)}_\cdot|_{\Hcal_{ext}^{(n)}\otimes\Hcal_{\bC}}$, отже
$\bfE_\cdot$ є лінійним неперервним оператором в
$(L^2)^\beta_q\otimes\Hcal_{\bC}$.
\begin{subproposition}{\rm (пор. з Твердженням~\ref{p2_3_1})}\label{p2_4_1}
Нехай $F\in (L^2)^\beta_q$. Тоді для $\beta\geq 0$
$\bfE_\cdot\partial_\cdot F\in (L^2)^\beta_q\otimes\Hcal_{\bC}$, та для
$\beta<0$ $\bfE_\cdot\partial_\cdot F\in (L^2)^\beta_{q-1}\otimes\Hcal_{\bC}$,
тут $\partial_\cdot$ -- похідна Хіди~\eqref{e1_23}.
\end{subproposition}
\begin{proof}
У випадку $\beta<0$ результат твердження випливає безпосередньо з властивостей операторів
$\partial_\cdot$ та $\bfE_\cdot$. Розглянемо випадок $\beta\geq 0$. З~\eqref{e1_21} та
\eqref{e2_14} випливає, що
\begin{equation}\label{e2_17}
\bfE_\cdot\partial_\cdot F=\sum_{n=0}^\infty (n+1){:\!\langle}\circ^{\otimes n},
F^{(n+1)}(\cdot)\chi_{n,\cdot}{\rangle\!:},
\end{equation}
де $F^{(n+1)}$ -- ядра з розкладу~\eqref{e1_5} для $F$, які розуміються як елементи
просторів $\Hcal_{ext}^{(n)}\otimes\Hcal_{\bC}$ (див. Підрозділ~\ref{S1_5}).
Для того, щоб оцінити норму $\bfE_\cdot\partial_\cdot F$ в просторі
$(L^2)^\beta_q\otimes\Hcal_{\bC}$, потрібен такий технічний результат.
\begin{sublemma}
Для довільних $n\in\bZ_+$ та $F^{(n+1)}\in\Hcal_{ext}^{(n+1)}$
\begin{equation}\label{e2_15}
(n+1)|F^{(n+1)}(\cdot)\chi_{n,\cdot}|_{\Hcal_{ext}^{(n)}\otimes\Hcal_{\bC}}^2
\leq |F^{(n+1)}|_{\Hcal_{ext}^{(n+1)}}^2
\end{equation}
(пор. з~\eqref{e2_19}).
\end{sublemma}
\begin{proof}
Використовуючи~\eqref{e1_7},~\eqref{e2_13}, неатомарність міри Лебега та той добре відомий факт,
що для {\it симетричної} інтегровної за Лебегом функції $f:\bR_+^m\to\bR_+$,
$m\in\bN$,
\begin{equation*}
\int_{\bR_+^m}f(t_1,\dots,t_m)dt_1\dots dt_m
=m\int_{\bR_+}dt_1\int_{[0,t_1)^{m-1}}f(t_1,\dots,t_m)dt_2\dots dt_m,
\end{equation*}
отримуємо
\begin{gather*}
(n+1)|F^{(n+1)}(\cdot)\chi_{n,\cdot}
|_{\Hcal_{ext}^{(n)}\otimes\Hcal_{\bC}}^2=\\
=\sum_{\underset{l_1s_1+\cdots+l_ks_k=n}{k,l_j,s_j\in\bN:
\ j=1,\dots,k,\ l_1>\cdots>l_k,}}
\frac{(n+1)!}{s_1!\cdots s_k!}
\Big(\frac{\|p_{l_1}\|_\nu}{l_1!}\Big)^{2s_1}
\cdots\Big(\frac{\|p_{l_k}\|_\nu}{l_k!}\Big)^{2s_k}\times\\
\times\int_{\bR_+^{s_1+\cdots+s_k+1}}|F^{(n+1)}(
\underset{l_1}{\underbrace{t_1,\dots,t_1}},\dots,
\underset{l_k}{\underbrace{t_{s_1+\cdots+s_k},\dots,
t_{s_1+\cdots+s_k}}},t)\times\\
\times\chi_{n,t}(\underset{l_1}{\underbrace{t_1,\dots,t_1}},\dots,
\underset{l_k}{\underbrace{t_{s_1+\cdots+s_k},
\dots,t_{s_1+\cdots+s_k}}})|^2
dt_1\cdots dt_{s_1+\cdots+s_k}dt=\\
=\sum_{\underset{l_1s_1+\cdots+l_ks_k=n}{k,l_j,s_j\in\bN:
\ j=1,\dots,k,\ l_1>\cdots>l_k>1,}}
\frac{(n+1)!}{s_1!\cdots s_k!}
\Big(\frac{\|p_{l_1}\|_\nu}{l_1!}\Big)^{2s_1}
\cdots\Big(\frac{\|p_{l_k}\|_\nu}{l_k!}\Big)^{2s_k}\times\\
\times\int_{\bR_+^{s_1+\cdots+s_k+1}}|F^{(n+1)}(
\underset{l_1}{\underbrace{t_1,\dots,t_1}},\dots,
\underset{l_k}{\underbrace{t_{s_1+\cdots+s_k},\dots,
t_{s_1+\cdots+s_k}}},t)|^2\times\\
\times dt_1\cdots dt_{s_1+\cdots+s_k}dt+\\
+\sum_{\underset{l_1s_1+\cdots+l_{k-1}s_{k-1}+s_k=n}{k,l_j,s_j\in\bN:
\ j=1,\dots,k,\ l_1>\cdots>l_k=1,}}
\frac{(n+1)!}{s_1!\cdots s_k!}
\Big(\frac{\|p_{l_1}\|_\nu}{l_1!}\Big)^{2s_1}
\cdots\Big(\frac{\|p_{l_{k-1}}\|_\nu}{l_{k-1}!}\Big)^{2s_{k-1}}\times\\
\times\int_{\bR_+^{s_1+\cdots+s_k+1}}|F^{(n+1)}(
\underset{l_1}{\underbrace{t_1,\dots,t_1}},\dots,
t_{s_1+\cdots+s_{k-1}+1},\dots,t_{s_1+\cdots+s_k},t)\times\\
\times\chi_{n,t}(\underset{l_1}{\underbrace{t_1,\dots,t_1}},\dots,
t_{s_1+\cdots+s_{k-1}+1},\dots,t_{s_1+\cdots+s_k})|^2
dt_1\cdots dt_{s_1+\cdots+s_k}dt=\\
=\sum_{\underset{l_1s_1+\cdots+l_ks_k=n}{k,l_j,s_j\in\bN:
\ j=1,\dots,k,\ l_1>\cdots>l_k>1,}}
\frac{(n+1)!}{s_1!\cdots s_k!}
\Big(\frac{\|p_{l_1}\|_\nu}{l_1!}\Big)^{2s_1}
\cdots\Big(\frac{\|p_{l_k}\|_\nu}{l_k!}\Big)^{2s_k}\times\\
\times\int_{\bR_+^{s_1+\cdots+s_k+1}}|F^{(n+1)}(
\underset{l_1}{\underbrace{t_1,\dots,t_1}},\dots,
\underset{l_k}{\underbrace{t_{s_1+\cdots+s_k},\dots,
t_{s_1+\cdots+s_k}}},t)|^2\times\\
\times dt_1\cdots dt_{s_1+\cdots+s_k}dt+\\
+\sum_{\underset{l_1s_1+\cdots+l_{k-1}s_{k-1}+s_k=n}{k,l_j,s_j\in\bN:
\ j=1,\dots,k,\ l_1>\cdots>l_k=1,}}
\frac{(n+1)!}{s_1!\cdots s_k!}
\Big(\frac{\|p_{l_1}\|_\nu}{l_1!}\Big)^{2s_1}
\cdots\Big(\frac{\|p_{l_{k-1}}\|_\nu}{l_{k-1}!}\Big)^{2s_{k-1}}\times\\
\times\int_{\bR_+^{s_1+\cdots+s_{k-1}+1}}
dtdt_1\cdots dt_{s_1+\cdots+s_{k-1}}\times\\
\times\int_{[0,t)^{s_k}}|F^{(n+1)}(
\underset{l_1}{\underbrace{t_1,\dots,t_1}},\dots,
t_{s_1+\cdots+s_{k-1}+1},\dots,t_{s_1+\cdots+s_k},t)|^2\times\\
\times dt_{s_1+\cdots+s_{k-1}+1}\cdots dt_{s_1+\dots+s_k}=\\
=\sum_{\underset{l_1s_1+\cdots+l_ks_k+1=n+1}{k,l_j,s_j\in\bN:
\ j=1,\dots,k,\ l_1>\cdots>l_k>1,}}
\frac{(n+1)!}{s_1!\cdots s_k!}
\Big(\frac{\|p_{l_1}\|_\nu}{l_1!}\Big)^{2s_1}
\cdots\Big(\frac{\|p_{l_k}\|_\nu}{l_k!}\Big)^{2s_k}\times\\
\times\int_{\bR_+^{s_1+\cdots+s_k+1}}|F^{(n+1)}(
\underset{l_1}{\underbrace{t_1,\dots,t_1}},\dots,
\underset{l_k}{\underbrace{t_{s_1+\cdots+s_k},\dots,
t_{s_1+\cdots+s_k}}},t)|^2\times\\
\times dt_1\cdots dt_{s_1+\cdots+s_k}dt+\\
+\sum_{\underset{l_1s_1+\cdots+l_{k-1}s_{k-1}+s_k+1=n+1}{k,l_j,s_j\in\bN:
j=1,\dots,k,\ l_1>\cdots>1,}}
\frac{(n+1)!}{s_1!\cdots (s_k+1)!}
\Big(\frac{\|p_{l_1}\|_\nu}{l_1!}\Big)^{2s_1}
\cdots\Big(\frac{\|p_{l_{k-1}}\|_\nu}{l_{k-1}!}\Big)^{2s_{k-1}}\times\\
\times\int_{\bR_+^{s_1+\cdots+s_k+1}}|F^{(n+1)}(
\underset{l_1}{\underbrace{t_1,\dots,t_1}},\dots,
t_{s_1+\cdots+s_{k-1}+1},\dots,t_{s_1+\cdots+s_k},t)|^2\times\\
\times dt_1\cdots dt_{s_1+\cdots+s_k}dt
\leq |F^{(n+1)}|_{\Hcal_{ext}^{(n+1)}}^2.
\end{gather*}
(Зауважимо, що якщо  $F^{(n+1)}$ задовольняє умову, накладену на ядра у твердженні (2)
Теореми~\ref{t2_2_1}, то на останньому кроці маємо рівність, тобто в такому випадку
$(n+1)|F^{(n+1)}(\cdot)\chi_{n,\cdot}|_{\Hcal_{ext}^{(n)}\otimes\Hcal_{\bC}}^2
=|F^{(n+1)}|_{\Hcal_{ext}^{(n)}}^2$, довести це пропонується зацікавленому читачу.)
\end{proof}
Повернемось до доведення твердження. Скориставшись~\eqref{e2_17},~\eqref{e1_12},~\eqref{e2_15}
та~\eqref{e1_8}, отримуємо
\begin{gather*}
\|\bfE_\cdot\partial_\cdot F\|_{(L^2)^\beta_q\otimes\Hcal_{\bC}}^2
=\sum_{n=0}^\infty (n!)^{1+\beta}2^{qn}(n+1)^2
|F^{(n)}(\cdot)\chi_{n,\cdot}|_{\Hcal_{ext}^{(n)}\otimes\Hcal_{\bC}}^2\leq\\
\leq 2^{-q}\sum_{n=0}^\infty ((n+1)!)^{1+\beta}2^{q(n+1)}(n+1)^{-\beta}
|F^{(n+1)}|_{\Hcal_{ext}^{(n+1)}}^2
\leq 2^{-q}\|F\|_{(L^2)^\beta_q}^2<\infty,
\end{gather*}
звідки й випливає потрібне.
\end{proof}

Сформулюємо й доведемо основний результат підрозділу.
\begin{subtheorem}{\rm (пор. з Теоремою~\ref{t2_3_1})}
Нехай випадкова величина $F\in (L^2)^\beta_q$ може бути представлена у вигляді~\eqref{e2_3}
(див. Теорему~\ref{t2_2_1}). Тоді має місце представлення (формула типу Кларка"=Окона)
\begin{equation}\label{e2_16}
F=\bfE{F}+\int\bfE_t\partial_t F\widehat dL_t
\end{equation}
(пор. з~\eqref{e2_2},~\eqref{e2_12}).
\end{subtheorem}
\begin{proof}
Спочатку доведемо теорему для $F={:\!\langle}\circ^{\otimes n},F^{(n)}{\rangle\!:}$,
$F^{(n)}\in\Hcal_{ext}^{(n)}$, $n\in\bN\backslash\{1\}$ (випадки $n=0$ та $n=1$ є
тривіальними). Використовуючи~\eqref{e1_21},~\eqref{e2_14} та~\eqref{e1_17}, отримуємо
\begin{equation*}
\int\bfE_t\partial_t{:\!\langle}\circ^{\otimes n},F^{(n)}{\rangle\!:}\widehat dL_t
=n{:\!\langle}\circ^{\otimes n},\widehat{F^{(n)}(\cdot)\chi_{n-1,\cdot}}{\rangle\!:},
\end{equation*}
отже, треба довести, що $n\widehat{F^{(n)}(\cdot)\chi_{n-1,\cdot}}=F^{(n)}$ в
$\Hcal_{ext}^{(n)}$. Нехай $f^{(n)}\in F^{(n)}$ -- симетрична функція, описана у
твердженні (2) Теореми~\ref{t2_2_1}. Нагадаємо, що ядро
$\widehat{F^{(n)}(\cdot)\chi_{n-1,\cdot}}$ породжене симетризацією функції
\begin{equation*}
f^{(n)}(\cdot_1,\dots,\cdot_{n-1},\cdot)\chi_{n-1,\cdot}(\cdot_1,\dots,\cdot_{n-1})
1_{\{\cdot_1\not=\cdot,\dots,\cdot_{n-1}\not=\cdot\}}
\end{equation*}
за всіма змінними, див. Підрозділ~\ref{S1_4}. Використовуючи~\eqref{e1_7}, властивості функції
$f^{(n)}$, щойно згадану конструкцію ядра $\widehat{F^{(n)}(\cdot)\chi_{n-1,\cdot}}$ та
неатомарність міри Лебега, отримуємо
\begin{gather*}
|F^{(n)}-n\widehat{F^{(n)}(\cdot)\chi_{n-1,\cdot}}|_{ext}^2
=|f^{(n)}-n\widehat{f^{(n)}(\cdot)\chi_{n-1,\cdot}}|_{ext}^2=\\
=\sum_{\underset{l_1s_1+\cdots+l_{k-1}s_{k-1}+s_k=n}
{k,l_j,s_j\in\bN:
\ j=1,\dots,k,\ l_1>\cdots>l_k=1,}}
\frac{n!}{s_1!\dots s_k!}
\Big(\frac{\|p_{l_1}\|_\nu}{l_1!}\Big)^{2s_1}
\cdots\Big(\frac{\|p_{l_{k-1}}\|_\nu}{l_{k-1}!}\Big)^{2s_{k-1}}\times\\
\times\int_{\bR_+^{s_1+\cdots+s_k}}\big|f^{(n)}(
\underset{l_1}{\underbrace{t_1,\dots,t_1}},\dots,
t_{s_1+\cdots+s_{k-1}+1},\dots,t_{s_1+\cdots+s_k})[1-\\
-1_{\{t_{s_1+\cdots+s_{k-1}+1}<t_{s_1+\cdots+s_k},
t_{s_1+\cdots+s_{k-1}+2}<t_{s_1+\cdots+s_k},\dots,
t_{s_1+\cdots+s_k-1}<t_{s_1+\cdots+s_k}\}}-\\
-1_{\{t_{s_1+\cdots+s_k}<t_{s_1+\cdots+s_k-1},
t_{s_1+\cdots+s_{k-1}+1}<t_{s_1+\cdots+s_k-1},\dots,
t_{s_1+\cdots+s_k-2}<t_{s_1+\cdots+s_k-1}\}}-\\
-\cdots -1_{\{t_{s_1+\cdots+s_{k-1}+2}<t_{s_1+\cdots+s_{k-1}+1},\dots,
t_{s_1+\cdots+s_k}<t_{s_1+\cdots+s_{k-1}+1}\}}]\big|^2\times\\
\times dt_1\cdots dt_{s_1+\cdots+s_k}=0
\end{gather*}
(для {\it різних} $t_{s_1+\cdots+s_{k-1}+1},\dots,t_{s_1+\cdots+s_k}$ один і тільки один
індикатор в цій викладці дорівнює одиниці; інші випадки можна ігнорувати через неатомарність
міри Лебега).

Твердження теореми в загальному випадку випливає з Твердження~\ref{p2_4_1},~\eqref{e1_17} та
щойно доведеного результату.
\end{proof}

Відзначимо, що якщо $F\in (L^2)^\beta_q$ задовольняє умову Твердження~\ref{p2_1_1} (тобто
якщо ядра $F^{(n)}$, $n\in\bZ_+$, з розкладу~\eqref{e1_5} для $F$ належать просторам
$\Hcal_{\bC}^{\widehat\otimes n}\subset\Hcal_{ext}^{(n)}$), формула~\eqref{e2_16}
редукується до~\eqref{e2_2} (довести це пропонується зацікавленому читачу).

\begin{subremark}\label{r2_4_1}{\rm (пор. з Зауваженням~\ref{r2_3_1})}
Нехай $F\in (L^2)^\beta_q$ можна представити у вигляді~\eqref{e2_3}.
З Зауваження~\ref{r2_2_1}, Твердження~\ref{p2_4_1} та представлення~\eqref{e2_16} випливає, що в
якості підінтегральної функції $G(\cdot)$ можна обрати $\bfE_\cdot\partial_\cdot F$.
\end{subremark}

\subsection{Класична формула Кларка"=Окона}
У Підрозділі~\ref{S2_1} ми встановили доволі обтяжливу {\it достатню} умову на випадкову
величину $F$, за якої формула Кларка"=Окона для $F$ приймає класичний вигляд~\eqref{e2_2}
(див. Твердження~\ref{p2_1_1}). На щастя, ця умова не є необхідною, і клас випадкових величин,
для яких справедливе представлення~\eqref{e2_2}, є доволі широким. Розглянемо це питання
докладно.

Нехай $F={:\!\langle}\circ^{\otimes 3},F^{(3)}{\rangle\!:}$, $F^{(3)}\in\Hcal_{ext}^{(3)}$.
Умовою, за якої це $F$ можна представити у вигляді~\eqref{e2_3}, є рівність
$\int_{\bR_+}|F^{(3)}(t,t,t)|^2dt=0$, див. Підрозділ~\ref{S2_1}. Але, як ми бачили у
згаданому підрозділі, для представлення $F$ у вигляді~\eqref{e2_2} цього недостатньо: потрібна
ще рівність $\int_{\bR_+^2}|F^{(3)}(t_1,t_1,t_2)|^21_{\{t_1\geq t_2\}}dt_1dt_2=0$, яка
виконується, якщо {\it $F^{(3)}$ містить представника, який дорівнює нулю, коли кратність
найбільшого аргументу більша за одиницю} (грубо кажучи, якщо $F^{(3)}(t_1,t_1,t_2)=0$, коли
$t_1\geq t_2$). Виявляється, що для $F\in (L^2)^\beta_q$ подібна умова на ядра з розкладу
\eqref{e1_5} і є необхідною та достатньою для представлення $F$ у вигляді~\eqref{e2_2}.
Сформулюємо та доведемо відповідне твердження; але спочатку пояснимо, чому виникає саме така
умова. Нехай $F\in (L^2)^\beta_q$ можна представити у вигляді~\eqref{e2_3} (насправді за
виконання згаданої вище умови ця вимога виконана автоматично, див. Зауваження~\ref{r2_5_1}
нижче). Представлення~\eqref{e2_16}, яке є однією з конкретизацій представлення
\eqref{e2_3} (див. Зауваження~\ref{r2_4_1}), відрізняється від представлення~\eqref{e2_2}
використанням функцій $\chi_{n,\cdot}$ (див.~\eqref{e2_13}) замість індикаторів
$1_{[0,\cdot)^n}$ у підінтегральному виразі (пор.~\eqref{e2_14} та~\eqref{e2_1}). Але, на відміну
від індикаторів, функції $\chi_{n,\cdot}$ "не реагують"\ на "поведінку"\ тих аргументів,
кратність яких є більшою за одиницю; отже, для того, щоб вирази у правих частинах~\eqref{e2_16}
та~\eqref{e2_2} співпали, "реагувати"\ мають ядра з розкладу~\eqref{e1_5} для $F$.
\begin{subtheorem}{\rm (пор. з Теоремою~\ref{t2_2_1})}\label{t2_5_1}
Нехай $F\in (L^2)^\beta_q$. Наступні твердження еквівалентні:
\begin{itemize}
\item[(1)] $F$ можна представити у вигляді~\eqref{e2_2}, де
$\bfE\big(\partial_\cdot F|_{\mathcal F_\cdot}\big)
\in (L^2)^\beta_q\otimes\Hcal_{\bC}$ у випадку $\beta\geq 0$, та
$\bfE\big(\partial_\cdot F|_{\mathcal F_\cdot}\big)
\in (L^2)^\beta_{q-1}\otimes\Hcal_{\bC}$, якщо $\beta<0$;
\item[(2)] для кожного $n\in\bN\backslash\{1\}$ ядро
$F^{(n)}\in\Hcal_{ext}^{(n)}$ з розкладу~\eqref{e1_5} для $F$ містить представника
$f^{(n)}$ такого, що $f^{(n)}(t_1,\dots,t_n)=0$, якщо існують $i,j\in\{1,\dots,n\}$,
$i\not=j$, такі, що $\max\{t_1,\dots,t_n\}=t_i=t_j$ (тобто якщо кратність найбільшого
$t_\cdot\in\{t_1,\dots,t_n\}$ більша за одиницю).
\end{itemize}
\end{subtheorem}
\begin{proof}
Спочатку доведемо теорему для $F={:\!\langle}\circ^{\otimes n},F^{(n)}{\rangle\!:}$,
$F^{(n)}\in\Hcal_{ext}^{(n)}$, $n\in\bN\backslash\{1\}$ (випадки $n=0$ та $n=1$ є
тривіальними). Нехай $f^{(n)}\in F^{(n)}$ -- симетричний представник класу еквівалентності
$F^{(n)}$ в просторі $\Hcal_{ext}^{(n)}$. Використовуючи~\eqref{e1_21},~\eqref{e2_1} та
конструкцію ядер розширеного стохастичного інтеграла (див. Підрозділ~\ref{S1_4}), отримуємо
\begin{gather*}
\int\bfE\big(\partial_t F|_{\mathcal F_t}\big)\widehat dL_t
=\int\bfE\big(\partial_t{:\!\langle}\circ^{\otimes n},F^{(n)}{\rangle\!:}
|_{\mathcal F_t}\big)\widehat dL_t=\\
=\int n{:\!\langle}\circ^{\otimes n-1},f^{(n)}(t)1_{[0,t)^{n-1}}{\rangle\!:}\widehat dL_t=\\
={:\!\langle}\circ^{\otimes n},
f^{(n)}nPr1_{[0,\cdot_n)^{n-1}}(\cdot_1,\dots,\cdot_{n-1}){\rangle\!:},
\end{gather*}
де, як і раніше, $Pr$  -- оператор симетризації за всіма змінними.
Отже, випадкову величину $F={:\!\langle}\circ^{\otimes n},F^{(n)}{\rangle\!:}$ можна представити
у вигляді~\eqref{e2_2} якщо та лише якщо функції $f^{(n)}$ та
$f^{(n)}nPr1_{[0,\cdot_n)^{n-1}}(\cdot_1,\dots,\cdot_{n-1})$ належать одному
і тому самому класу еквівалентності $F^{(n)}$ в $\Hcal_{ext}^{(n)}$.

1) ("(1)"$\Rightarrow$"(2)") Якщо $F$ можна представити у вигляді~\eqref{e2_2}, то, як щойно
встановлено, $f^{(n)}nPr1_{[0,\cdot_n)^{n-1}}(\cdot_1,\dots,\cdot_{n-1})\in F^{(n)}$.
Легко перевірити, що ця функція задовольняє умову твердження (2) теореми.

2) ("(2)"$\Rightarrow$"(1)") Нехай $f^{(n)}\in F^{(n)}$ -- описаний у твердженні (2)
представник $F^{(n)}$. Не втрачаючи загальності можна вважати, що $f^{(n)}$ є симетричною
функцією. Легко показати, що зараз справедлива рівність
$f^{(n)}=f^{(n)}nPr1_{[0,\cdot_n)^{n-1}}(\cdot_1,\dots,\cdot_{n-1})$, а тому
$F$ можна представити у вигляді~\eqref{e2_2}.

У загальному випадку імплікація "(1)"$\Rightarrow$"(2)"\ тривіальним чином випливає з
\eqref{e1_17} та відповідної імплікації у щойно розглянутому частинному випадку;
імплікація "(2)"$\Rightarrow$"(1)"\ -- з того факту, що за виконання умови твердження
(2) $\bfE\big(\partial_\cdot F|_{\mathcal F_\cdot}\big)
=\bfE_\cdot\partial_\cdot F$ (див. доведення Твердження~\ref{p2_5_1} нижче),
Твердження~\ref{p2_4_1},~\eqref{e1_17} та щойно доведеної відповідної імплікації у частинному
випадку.
\end{proof}
\begin{subremark}\label{r2_5_1}
Відзначимо, що якщо деяке $F\in (L^2)^\beta_q$ задовольняє умову твердження (2)
Теореми~\ref{t2_5_1}, то це $F$ задовольняє й умову твердження (2) Теореми~\ref{t2_2_1},
оскільки представлення~\eqref{e2_2} для $F$ є однією з конкретизацій представлення~\eqref{e2_3}.
Довести це можна й безпосередньо, що пропонується зробити в якості вправи зацікавленому читачу.
\end{subremark}

Як вже відзначалось, формула типу Кларка"=Окона~\eqref{e2_16} є безпосереднім узагальненням
класичної формули Кларка"=Окона~\eqref{e2_2}. Точніше, справедливе таке твердження.
\begin{subproposition}\label{p2_5_1}
Якщо $F\in (L^2)^\beta_q$ можна представити у вигляді~\eqref{e2_2}, то
\begin{equation}\label{e2_18}
\bfE\big(\partial_\cdot F|_{\mathcal F_\cdot}\big)=\bfE_\cdot\partial_\cdot F
\end{equation}
в $(L^2)^\beta_q\otimes\Hcal_{\bC}$ у випадку $\beta\geq 0$ та в
$(L^2)^\beta_{q-1}\otimes\Hcal_{\bC}$ у випадку $\beta<0$.
\end{subproposition}
\begin{proof}
Спочатку доведемо твердження для $F={:\!\langle}\circ^{\otimes n},F^{(n)}{\rangle\!:}$,
$F^{(n)}\in\Hcal_{ext}^{(n)}$, $n\in\bN\backslash\{1\}$ (випадки $n=0$ та $n=1$ є
тривіальними). Нехай $f^{(n)}\in F^{(n)}$ -- описаний в умові твердження (2)
Теореми~\ref{t2_5_1} представник $F^{(n)}$. Згідно з~\eqref{e1_21},~\eqref{e2_1} та~\eqref{e2_14}
достатньо показати, що $f^{(n)}(\cdot)1_{[0,\cdot)^{n-1}}=f^{(n)}(\cdot)\chi_{n-1,\cdot}$ в
$\Hcal_{ext}^{(n-1)}\otimes\Hcal_{\bC}$. Легко перевірити, що якщо для деяких
$t_1,\dots,t_{n-1},t$ $1_{[0,t)^{n-1}}(t_1,\dots,t_{n-1})\not=\chi_{n-1,t}(t_1,\dots,t_{n-1})$,
то кратність $\max\{t_1,\dots,t_{n-1},t\}$ є більшою за одиницю; а у такому випадку
$f^{(n)}(t_1,\dots,t_{n-1},t)=0$. Отже, для будь-якого набору аргументів
\begin{equation*}
f^{(n)}(t_1,\dots,t_{n-1},t)[1_{[0,t)^{n-1}}(t_1,\dots,t_{n-1})-\chi_{n-1,t}(t_1,\dots,t_{n-1})]
=0
\end{equation*}
і тому 
\[
    |f^{(n)}(\cdot)[1_{[0,\cdot)^{n-1}}-\chi_{n-1,\cdot}]|_{\Hcal_{ext}^{(n-1)}\otimes\Hcal_{\bC}}=0,
\]
що і треба було довести.

У загальному випадку рівність~\eqref{e2_18} в просторі $(L^2)^\beta_{q-1}\otimes\Hcal_{\bC}$ є наслідком щойно доведеного результату та неперервності операторів $\partial_\cdot$, $\bfE\big(\circ (\cdot)|_{\mathcal F_\cdot}\big)$ і $\bfE_\cdot$; а якщо $\beta\geq 0$, то згідно з твердженням~\ref{p2_4_1} ця рівність є справедливою в просторі $(L^2)^\beta_q\otimes\Hcal_{\bC} \subset (L^2)^\beta_{q-1}\otimes\Hcal_{\bC}$.
\end{proof}

Відзначимо, що результати цієї статті залишаються справедливими для випадкових величин $F\in (L^2)^\beta$, $\beta\in [-1,1]$, в цьому випадку підінтегральні функції належать відповідним просторам $(L^2)^\beta\otimes\Hcal_{\bC}$.

Насамкінець зауважимо, що крім просторів з регулярного оснащення простору $(L^2)$~\eqref{e1_9}, в аналізі білого шуму Леві уводяться та вивчаються так звані простори {\it нерегулярних} основних і узагальнених функцій~\cite{K13_2,K21}, а також визначаються та досліджуються різноманітні оператори й операції на таких просторах. 
Варто зазначити, що деякі властивості згаданих просторів суттєво відрізняються від властивостей просторів $(L^2)^\beta_q$. 
Побудові й дослідженню формул типу Кларка"=Окона на просторах нерегулярних основних і узагальнених функцій будуть присвячені інші роботи автора.

Я щиро вдячний професору В.~І.~Герасименку за пропозицію написати цю статтю та всебічну підтримку.

