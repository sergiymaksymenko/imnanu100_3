

% !TeX encoding = UTF-8
\documentclass[11pt, reqno]{amsart}

\usepackage[utf8]{inputenc}
\usepackage[T2A]{fontenc}
\usepackage[english,ukrainian,russian]{babel}

\usepackage{imnanu}  % main style file

\usepackage{amssymb, enumitem}
\usepackage{graphicx}
\usepackage{xcolor}
\usepackage[all]{xy}


%% if you need to change parameters
% \SET{\Year}{2021}
% \SET{\Volume}{12}
% \SET{\Number}{5}

\sloppy

\begin{document}
%%%%%%%%%%%%%%%%%%%%%%%%%%%%%%%%%%%%%%%%%%%%%%%%%
%!!!!!! do not change this line - it will print the correct number of the first page
\label{first_page:\thearticlesnum}
%%%%%%%%%%%%%%%%%%%%%%%%%%%%%%%%%%%%%%%%%%%%%%%%%



%%%%%%%%%% PRINT INFORMATION ABOUT AUTHORS

%------ uncomment paper language
\selectlanguage{ukrainian}
% \selectlanguage{russian}
% \selectlanguage{english}


%------ information about each author
\author{С.~А.~Плакса}
\address{Інститут математики НАН України, м.~Київ}
\email{plaksa@imath.kiev.ua, plaksa62@gmail.com}
% \orcid{}

\author{В.~С.~Шпаківський}
\address{Інститут математики НАН України, м.~Київ}
\email{shpakivskyi86@gmail.com}
% \orcid{}



\title[Інтегральні теореми в скінченновимірній
комутативній алгебрі]{Інтегральні теореми в скінченновимірній
комутативній алгебрі}

%------ abstracts
\abstract{english}{For monogenic (continuous and differentiable in the sense of G\^ateaux)
functions given in special real subspaces of an arbitrary finite-dimensional
commutative associative algebra over the complex field and taking values in this algebra,
we establish basic properties analogous to properties of holomorphic functions of a complex variable.
Methods for proving results are based on a representation of monogenic functions via holomorphic functions of complex
 variables that allows to establish analogues of Cauchy--Riemann conditions and the continuity
  of G\^ateaux derivatives of all orders for monogenic functions. In such a way, analogues of a number of classical theorems
   of complex analysis (the Cauchy integral theorem for a curvilinear integral, the Cauchy integral formula, the Morera
theorem, the Taylor theorem) are proved and different equivalent definitions
for the mentioned monogenic functions are established.
An analogue of the Cauchy theorem for an integral over non piecewise smooth surfaces is proved.}



\abstract{ukrainian}{Для моногенних (неперервних і диференційовних за Гато) функцій, що
визначені у спеціальних дійсних підпросторах довільної скінченновимірної комутативної асоціативної алгебри над
комплексним полем і примають значення в цій алгебрі, встановлено основні властивості,
аналогічні властивостям голоморфних функцій комплексної змінної.
Методи дослідження базуються на представленні моногенних функцій через голоморфні функції
комплексних змінних, що дає змогу встановити аналоги умов Коші--Рімана та
неперервність похідних Гато всіх порядків для моногенних функцій.
У такий спосіб доведено аналоги ряду класичних теорем комплексного аналізу
(інтегральна теорема Коші для криволінійного інтеграла, інтегральна формула Коші,
теорема Морера, теорема Тейлора) та
встановлено різні еквівалентні означення моногенних функцій.
Доведено аналог теореми Коші для поверзневого інтеграла по не кусково гладких поверхнях.}

% \abstract{russian}{
% В данной работе мы ...
% }


\keywords{комутативна банахова алгебра, моногенна функція, аналітична функція, гіперголоморфна функція, умови Коші--Рімана,
теорема Коші, інтегральна формула Коші, теорема Морера, теорема Тейлора}
\udc{517.54, 519.95}
\msc{30G35, 35J05, 31A30}
%% DOI of the current paper
% \doi{}

\maketitle


%--------------- parer text
\section{Вступ}\label{1sec1}


З середини 70-х років минулого століття в Інституті математики НАН
України, починаючи з роботи І.П.~Мельниченка \cite{Mel'nichenko75},
систематично і послідовно розробляється
алгебраїчно-аналітичний підхід до основних еліптичних рівнянь
математичної фізики, який пов'язаний з використанням комутативних
алгебр.

Ідея такого підходу полягає у знаходженні комутативних
банахових алгебр таких, щоб диференційовні за Гато
функції зі значеннями в цих алгебрах мали компоненти, які є
розв'язками заданих рівнянь з частинними похідними.

Такі алгебри були знайдені І.П.~Мельниченком для
тривимірного рівняння Лапласа (див.
\cite{Mel'nichenko75,Mel'nichenko03,Mel-Plaksa}) і еліптичних
рівнянь з виродженням на осі, що описують осесиметричні
потенціальні поля (див. \cite{M2,Mel-Plaksa}), В.Ф.~Ковальовим і
І.П.~Мельниченком для двовимірного бігармонічного рівняння (див.
\cite{Kov-Mel,Mel'nichenko86}) і узагальненого бігармонічного
рівняння (див. \cite{Kov-Mel-pr91}).
Згодом, в роботах \cite{Pl-Zb-2013,Pla-Shp-Bull-2012} показано, що для опису всіх просторових гармонічних функцій
у формі компонент диференційовних за Гато гіперкомплексних функцій
відповідні нескінченновимірні комутативні банахові
алгебри необхідно включити у топологiчнi векторнi простори з тим же базисом і топологією покоординатної збіжності.


Інтерес до дослідження функцій в комутативних алгебрах
гіперкомплексних чисел останнім часом зростає у зв'язку з
поєднанням зручностей властивості комутативності з широкими
можливостями застосувань (див., наприклад, монографії
Г.Б,~Прайса \cite{Price}, Д.~Бокалетті та ін. \cite{Boc-Cat}, М.Е.~Луна-Елізаррарас та ін. \cite{Luna-Shapiro-Struppa}
і роботи В.В.~Кісіля \cite{Kisil12,Kisil2-12}, А.~Погоруя, М.Н.~Родрігеса-Даніно і М.~Шапіро \cite{Pogor-Ramon-Shap},
в яких вивчаються різноманітні алгебраїчні, геометричні і
аналітичні аспекти теорії гіперкомплексних чисел).



Зрозуміло, що для успішної реалізації вказаного вище підходу до основних еліптичних рівнянь
математичної фізики, необхідно
розповсюдити класичні методи теорії голоморфних
функцій комплексної змінної %та функціонального аналізу стосовно конктретних
на диференційовні за Гато функції, задані в банахових алгебрах, асоційованих з рівняннями математичної
фізики.
Ускладненість ситуації при цьому полягає в обмеженості можливостей переносу класичних теорем комплексного
аналізу в аналіз на банахових алгебрах.

У роботі Е.Р.~Лорха \cite{Lorch} доведено інтегральну теорему Коші та інтегральну формулу Коші,
теореми Тейлора та Морера для функцій, диференційовних у сенсі
Лорха в довільній опуклій області комутативної банахової алгебри. Умову опуклості області в цих результатах
було знято Е.К.~Блюмом \cite{Blum}.


У той же час, застосування диференціовних функцій зі значеннями в комутативних банахових алгебрах
до побудови розв'язків рівнянь математичної фізики
вимагає дослідження таких функцій у спеціальних дійсних підпросторах вказаних алгебр (див. цитовані роботи
\cite{Mel'nichenko75,Mel'nichenko03,Mel-Plaksa,Kov-Mel,Mel'nichenko86,Kov-Mel-pr91,Pl-Zb-2013,Pla-Shp-Bull-2012}
та роботи П.В.~Кетчума \cite{Ketchum-28, Ketchum-29}, Л.~Собреро \cite{Sobrero}) і М.Н.~Рошкулеця \cite{Rosculet}.

У цій роботі розглядаються моногенні (тобто неперервні та диференційовні за Гато) функції
$\Phi : \Omega\rightarrow\mathbb{A}$, що визначені в області $\Omega$ певного дійсного підпростору $E_{3}$
скінченновимірної комутативної асоціативної алгебри $\mathbb{A}$ з одиницею над полем комплексних чисел
і приймають значення в цій алгебрі.

Як буде показано, основні властивості таких моногенних функцій
аналогічні властивостям голоморфних функцій комплексної змінної.
Методи дослідження базуються на представленні моногенних функцій через голоморфні функції
комплексних змінних, що дає змогу встановити аналоги умов Коші--Рімана та
неперервність похідних Гато всіх порядків для моногенних функцій.
У такий спосіб доведено аналоги ряду класичних теорем комплексного аналізу
(інтегральна теорема Коші для криволінійного інтеграла, інтегральна формула Коші,
теорема Морера, теорема Тейлора) та
встановлено різні еквівалентні означення моногенних функцій.

Встановлені результати узагальнюють відповідні результати робіт
\cite{Pl-Shp2,Pl-Shp3,Plaksa12,Pl-Shp-Al,Pukh-5} для моногенних функцій в конкретних
скін\-чен\-но\-ви\-мір\-них комутативних асоціативних алгебрах.
Згадаємо також роботи П.В.~Кетчума \cite{Ketchum-28,Ketchum-29},
В.~Гончарова \cite{Goncharow} і М.Н.~Рошкулеця \cite{Rosculet-54,Rosculet-55},
в яких аналоги теореми Коші та інтегральної формули Коші для криволінійного інтеграла
встановлено в інших конкретних комутативних алгебрах.

Вкажемо роботи, в яких деякі інтегральні теореми доведено в некомутативних алгебрах.
Так, ряд гіперкомплексних аналогів інтегральної теореми Коші для криволінійного інтеграла
встановлено в роботах А.~Садбері \cite{Sudbery},  Ф.~Коломбо, І.~Сабадіні і Д.~Струппи \cite{Colombo}.
В роботах А.~Садбері \cite{Sudbery}, Ф.~Брекса, Р.~Деланга і Ф.~Соммена
\cite{Brakx}, В.В.~Кравченка і М.В.~Шапіро
\cite{Krav-Shap}, С.~Бернштейн \cite{Bernstein}, О.Ф.~Геруса \cite{Gerus-2011}
подібні теореми доведено для поверхневого інтеграла.


\vskip 1mm


%%%%%%%%%%%%%%%%%%%%%%%%%%%%%%
\section{Моногенні функції в скінченновимірній комутативній асоціативній алгебрі}

Нехай $\mathbb A$ --- $n$-вимірна комутативна асоціативна банахова
алгебра з одиницею $1$ над полем дійсних чисел $\mathbb R$ або над
полем комплексних чисел $\mathbb C$,\,\, $3\le n\le\infty$\,.

Розглянемо в алгебрі $\mathbb{A}$ вектори $e_1=1$, $e_2, e_3$,
лінійно незалежні над полем $\mathbb{R}$. Це означає, що рівність
$$\alpha_1e_1+\alpha_2e_2+\alpha_3e_3=0,\qquad \alpha_1,\alpha_2,
\alpha_3\in\mathbb{R},$$
виконується тоді і тільки тоді, коли $\alpha_1=\alpha_2=
\alpha_3=0$.

Нехай $E_3:=\{\zeta :=xe_1+ye_2+ze_3\,: \,\,x,y,z\in\mathbb R\}$ ---
лінійна оболонка векторів $e_1,e_2,e_3$ над полем $\mathbb R$.

Будемо використовувати однакове позначення $\Omega$ для області
$\Omega\subset{\mathbb R}^3$ і для області в $E_3$, яка є
конгруентною  до області $\Omega$.


%%%%%%%%%%%%%
\subsection{Диференційовність за Лорхом і за Гато. Моногенні і аналітичні функції}


Розглянемо функцію $\Phi \colon \Omega\rightarrow\mathbb{A}$,
визначену в області $\Omega\subset E_3$, і поняття
диференційовності цієї функції за Лорхом і за Гато.


Функція $\Phi \colon \Omega \rightarrow \mathbb{A}$
називається {\em диференційовною за Лорхом} (див. \cite{Lorch}) в
області $\Omega\subset E_3$, якщо для кожної точки
$\zeta\in\Omega$ існує елемент $\Phi_L'(\zeta)\in\mathbb{A}$
такий, що для кожного $\varepsilon>0$ існує $\delta>0$ таке, що
для всіх $h\in E_3$\,,\, для яких\, $\|h\| <\delta$\,, виконується
нерівність
\begin{equation}\label{difLorch}
\left\|\Phi(\zeta+h)-\Phi(\zeta)-h\Phi_L'(\zeta)\right\|\leq\|h\|\,\varepsilon\,.
\end{equation}
Очевидно, що в нерівності (\ref{difLorch})
 {\em похідна Лорха} $\Phi_L'(\zeta)$ є функцією
змінної $\zeta$, тобто\, $\Phi_L' \colon
\Omega\rightarrow\mathbb{A}$\,.


Зазначимо, що динференційовними за Лорхом функціями в комутативних
асоціативних банахових алгебрах над полем $\mathbb{C}$ є, зокрема,
головні продовження (див., наприклад, монографію Е.~Хілле і Р.~Філліпса \cite[с.~182]{Hil_Filips})
голоморфних функцій комплексної змінної. Якщо комплексна функція
$F$ є голоморфною в області $D\subset \mathbb{C}$, то для усіх
$\zeta\in \mathbb{A}$, спектр яких міститься в $D$, головне
продовження функції $F$ виражається рівністю
\begin{equation}\label{pr-ext}
\frac{1}{2\pi i}\int\limits_{\Gamma}
F(t)\,(t-\zeta)^{-1}\,dt\,,
\end{equation}
де $\Gamma$ --- довільна замкнена спрямлювана жорданова
крива в $D$, яка охоплює спектр елемента $\zeta$\,.


Використовуючи диференціал Гато,  І.П.~Мельниченко
\cite{Mel'nichenko75} розглянув похідну Гато функції $\Phi \colon \Omega\rightarrow\mathbb{A}$
також як функцію точки $\zeta\in\Omega$.


Ми кажемо, що функція $\Phi \colon
\Omega\rightarrow\mathbb{A}$ є {\em диференційовною за Гато} в
області $\Omega\subset E_3$, якщо для кожної точки
$\zeta\in\Omega$ існує елемент $\Phi_G'(\zeta)\in\mathbb{A}$
такий, що
\begin{equation}\label{Gprz}
\lim\limits_{\delta\rightarrow 0+0} \left(\Phi(\zeta+\delta
h)-\Phi(\zeta)\right)\delta^{-1}= h\Phi_G'(\zeta)\quad\forall h\in
E_{3}.
\end{equation}
Очевидно, що
{\em похідна Гато} $\Phi_G'(\zeta)$ для кожного вектора $h\in E_{3}$
є узагальненням класичної похідної за напрямком.

Зауважимо, що обидва означення: похідної Лорха (\ref{difLorch}) і
похідної Гато (\ref{Gprz}), --- враховують існування необоротних
елементів $h$ в алгебрі $\mathbb{A}$, оскільки в цих означеннях не
використовується ділення на елементи алгебри на відміну від
класичного означення похідної функції комплексної змінної.

Очевидно, що функція $\Phi$, диференційовна за Лорхом в
області $\Omega$, є також диференційовною за Гато і
$\Phi_L'(\zeta)=\Phi_G'(\zeta)$ для всіх $\zeta\in\Omega$\,.
Обернене твердження не є істинним подібно до того, як
існування класичних похідних у точці за усіма напрямками не гарантує сильної
диференційовності (і навіть неперервності) функції у цій точці.

Для  функції
$\Phi\colon\Omega\rightarrow\mathbb{A}$ розглянемо поняття моногенності та аналітичності.

Ми говоримо, що функція  $\Phi \colon \Omega \rightarrow
\mathbb{A}$  \textit{моногенна} в області $\Omega\subset E_{3}$,
якщо $\Phi$ неперервна і диференційовна за Гато  в кожній точці
області $\Omega$.

Ми використовуємо поняття моногенної функції у сенсі існування для
неї похідних чисел (див. монографії Е.~Гурса \cite{Goursat} і Ю.Ю.~Трохимчука \cite{Trokhimchuk}) у поєднанні з
неперервністю цієї функції.

У науковій літературі назва моногенна функція
вико\-ри\-сто\-ву\-єть\-ся також для функцій, які задані у
некомутативних алгебрах і задовольняють деякі умови, подібні до
класичних умов Коші--Рімана (див., наприклад, роботу Дж.~Райана \cite{Ryan}). Такі
функції називають також регулярними (див., наприклад, роботу А.~Садбері
\cite{Sudbery}) або гіперголоморфними (див., наприклад, монографію В.В.~Кравченка і М.В.~Шапіро
\cite{Krav-Shap}).


Функцію $\Phi : \Omega\rightarrow \mathbb A$ називають {\it
аналітичною\/} в області $\Omega\subset E_3$, якщо в деякому околі
кожної точки $\zeta_0\in \Omega$ вона може бути представлена у
вигляді суми збіжного степеневого ряду з коефіцієнтами, що
належать алгебрі $\mathbb A$.


Очевидно, що кожна аналітична функція $\Phi : \Omega\rightarrow \mathbb A$ є моногенною
в області $\Omega\subset E_3$ і її похідна Гато $\Phi_G'(\zeta)$ також є моногенною в цій області.
Далі будуть вказані достатні умови, за яких моногенна функція $\Phi : \Omega\rightarrow \mathbb A$ є аналітичною в
області $\Omega\subset E_3$.




%%%%%%%%%%%%%%%
\subsection{Представлення моногенних функцій через голоморфні функції комплексних змінних}

Нехай тепер $\mathbb{A}$ --- довільна $n$-вимірна комутативна
асоціативна алгебра з одиницею над полем комплексних чисел.
Е.~Картан у роботі \cite{Cartan} довів, що в алгебрі $\mathbb{A}$
існує базис $\{I_k\}_{k=1}^{n}$ і існують структурні константи
$\Upsilon_{r,k}^{s}$ такі, що виконуються наступні правила множення:

1)  \,\,  $\forall\, r,s\in[1,m]\cap\mathbb{N}\,:$ \quad
$I_rI_s=\left\{
\begin{array}{rcl}
0 &\mbox{при} & r\neq s, \\[1mm]
I_r &\mbox{при} & r=s;\\
\end{array}
\right.$

2) \,\,  $\forall\, r,s\in[m+1,n]\cap\mathbb{N}\,:$ \quad $I_rI_s=
\sum\limits_{k=\max\{r,s\}+1}^n \Upsilon_{r,k}^{s}I_k$\,; \vskip 1mm


3)\,\, $\forall\, s\in[m+1,n]\cap\mathbb{N} \,\,\, \exists\,!\;
 u_s\in[1,m]\cap\mathbb{N}$ \; $\forall\,
 r\in[1,m]\cap\mathbb{N}\,:$\\
$$I_rI_s=\left\{
\begin{array}{lll}
0 & \mbox{при} &  r\neq u_s\,,\\[1mm]
I_s & \mbox{при} &  r= u_s\,,
\end{array}
\right.
$$
де $\mathbb{N}$ --- множина натуральних чисел.  Очевидно, що перші
$m$ базисних векторів $\{I_u\}_{u=1}^m$ є ідемпотентами і
породжують напівпросту підалгебру $\mathcal{S}$ алгебри $\mathbb{A}$, а
вектори $\{I_r\}_{r=m+1}^n$ породжують нільпотентну підалгебру $\mathcal{N}$
цієї алгебри. Надалі алгебру $\mathbb{A}$ з базисом Картана
позначатимемо $\mathbb{A}_n^m$. Одиницею алгебри $\mathbb{A}_n^m$
є елемент $1=\sum\limits_{u=1}^mI_u$.

Норма елемента $v=\sum\limits_{r=1}^nv_rI_r$ алгебри $\mathbb{A}_n^m$ визначається
рівністю
$$\|v\|:=\sqrt{\sum\limits_{r=1}^n|v_r|^2}.$$


Алгебра $\mathbb{A}_n^m$ містить $m$ максимальних ідеалів
$$\mathcal{I}_u:=\Biggr\{\sum\limits_{k=1,\,k\neq u}^n\lambda_kI_k:\lambda_k\in
\mathbb{C}\Biggr\}, \quad  u=1,2,\ldots,m.
$$

Визначимо $m$ лінійних неперервних мультиплікативних функціоналів
$f_u:\mathbb{A}_n^m\rightarrow\mathbb{C}$ рівностями
$$f_u(I_u)=1,\quad f_u(\omega)=0 \qquad\forall\,
\omega\in\mathcal{I}_u,\quad u=1,2,\ldots,m.$$

Нехай
\begin{equation}\label{e_1_e_2_e_3}
e_1=1,\quad e_2=\sum\limits_{r=1}^na_rI_r,\quad e_3=\sum\limits_{r=1}^nb_rI_r
\end{equation}
при $a_r,b_r\in\mathbb{C}$ --- трійка векторів в алгебрі
$\mathbb{A}_n^m$, які лінійно незалежні над полем
$\mathbb{R}$.

Нехай $\zeta:=xe_1+ye_2+ze_3$, де $x,y,z\in\mathbb{R}$.

Очевидно, що  $\xi_u:=f_u(\zeta)=x+ya_u+zb_u$,\, $u=1,2,\ldots,m$.

Накладемо наступне обмеження на вибір лінійної оболонки\, $E_3$:
\begin{equation}\label{3-1:cond-on-E3}
 f_u (E_3):=\{f_u(\zeta) : \zeta\in E_3\} =\mathbb{C}\,,\qquad   u=1,2,\ldots,m\,,
\end{equation}
тобто образом $f_u(E_3)$ множини $E_3$ при кожному відображенні $f_u$ має бути вся комплексна площина
(див. \cite{Pukh-5}).
Очевидно, що це має місце тоді і тільки тоді, коли при кожному
фіксованому $u=1,2, \ldots, m$ хоча б одне з чисел $a_u$ чи $b_u$
належить $\mathbb{C}\setminus\mathbb{R}$.


Для області
$\Omega\subset E_3$ через $D_u$ позначимо область комплексної
площини, на яку $\Omega$ відображається функціоналом $f_u$.

У роботі \cite{Sh-co} доведено наступне представлення резольвенти:
\begin{equation}\label{rozkl-rezol-A_n^mR}
(te_1-\zeta)^{-1}=\sum\limits_{u=1}^m\frac{1}{t-\xi_u}\,I_u+
 \sum\limits_{s=m+1}^{n}\sum\limits_{k=2}^{s-m+1}\frac{Q_{k,s}}
 {\left(t-\xi_{u_{s}}\right)^k}\,I_{s}\,
  \end{equation}
  $$ \forall\,t\in\mathbb{C}:\,
t\neq \xi_u,\quad u=1,2,\ldots,m,$$ де $Q_{k,s}$ визначені такими
рекурентними співвідношеннями:
$$
Q_{2,s}:=T_{s}\,,\quad
Q_{k,s}=\sum\limits_{r=k+m-2}^{s-1}Q_{k-1,r}\,B_{r,\,s}\,,\;
\;\;k=3,4,\ldots,s-m+1,
$$
при
$T_s:=ya_s+zb_s$\,, $B_{r,s}:=\sum\limits_{k=m+1}^{s-1}T_k
C_{r,s}^k$\,, $s=m+2,\ldots,n$,
 а натуральні числа $u_s$  визначені у правилі множення 3 алгебри $\mathbb{A}_n^m$.

Із співвідношень  (\ref{rozkl-rezol-A_n^mR}) випливає, що точки
 $(x,y,z)\in\mathbb{R}^3$, які відповідають необоротним елементам
 $\zeta\in\mathbb{A}_n^m$, лежать на прямих
\begin{equation}\label{3-2:L-uR}
 L_u:\quad\left\{
\begin{array}{r}x+y\,{\rm Re}\,a_u+z\,{\rm Re}\,b_u=0,\vspace*{2mm} \\
y\,{\rm Im}\,a_u+z\,{\rm Im}\,b_u=0. \\
\end{array} \right.
\end{equation}


Наступна теорема містить представлення моногенних функцій, що
приймають значення в алгебрі $\mathbb{A}_n^m$, через голоморфні
функції комплексних змінних.

\vskip 1mm

\begin{theorem}[\cite{Sh-co}] \label{Shpak-costr-op} 
Нехай виконується умова \eqref{3-1:cond-on-E3} і
область
$\Omega\subset E_3$ є опуклою в напрямку прямих $L_u$ при всіх
 $u=1,2,\ldots, m$.
Тоді кожна моногенна функція
$\Phi:\Omega\rightarrow\mathbb{A}_n^m$ подається у вигляді
\begin{multline}
  \label{Teor--1RR}
\Phi(\zeta)=\sum\limits_{u=1}^mI_u\,\frac{1}{2\pi
i}\int\limits_{\Gamma_u} F_u(t)(t-\zeta)^{-1}\,dt+\\
+\sum\limits_{s=m+1}^nI_s\,\frac{1}{2\pi i}\int\limits_
{\Gamma_{u_s}}G_s(t)(t-\zeta)^{-1}\,dt,
\end{multline}
де $F_u$ --- деяка голоморфна функція в області $D_u$ і $G_s$ ---
 деяка голоморфна функція
в області $D_{u_s}$, а $\Gamma_q$ --- замкнена жорданова
спрямлювана крива, яка лежить в області $D_q$, охоплює  точку
$\xi_q$ і не містить точок $\xi_{\ell}$\,, $\ell=1,2,\ldots,
m$,\,$\ell\neq q$. 
\end{theorem}

\vskip 1mm


\begin{remark}\label{3-1:Rem-int-op-sim-pe}
Основна відмінність між інтегральним оператором (\ref{Teor--1RR})
і головним продовженням (\ref{pr-ext}) голоморфних функцій у комутативну банахову алгебру полягає в тому, що крива
$\Gamma_u$ не зобов'язана охоплювати всі точки спектра елемента\, $\zeta$\,.
Тому інтегральний оператор (\ref{Teor--1RR})
застосовний також у випадку, коли деякі точки згаданого спектра не належать області\, $D$\,.
\end{remark}



Зазначимо, що Теорему \ref{Shpak-costr-op} узагальнено в роботі \cite{Sh25} на випадок
моногенних функцій $\Phi : \Omega\rightarrow \mathbb{A}_n^m$, заданних в області
$\Omega\subset E_k$, де
$E_k:=\{\zeta=\sum\limits_{j=1}^k x_{j}e_j:\,\,\,x_j\in\mathbb{R}\}$ ---
лінійна оболонка векторів\,\, $e_1=1,e_2,\dots,e_k$\,, $2\leq k\leq 2n$\,,
лінійно незалежних над полем\, $\mathbb{R}$.
Отримані представлення моногенних функцій
узагальнюють  відповідні результати робіт
 \cite{Pl-Gr-big,Pl-Shp1,PlPu13,Pukh-5,Pl-Shp-Al} та ряд інших результатів про представлення аналітичних
функцій в конкретних скінченновимірних комутативних алгебрах, що
беруть свій початок від роботи Ф.~Рінглеба \cite{Ringleb}, який
отримав аналогічне представлення аналітичних функцій бікомплексної змінної.


Принциповими наслідками рівності (\ref{Teor--1RR}) є твердження, сформульовані в
наступній теоремі, яка справедлива для довільної області\, $\Omega\subset E_3$\,.

\begin{theorem}\label{3-1:teo_pro_naslidky2}
Нехай виконується умова \eqref{3-1:cond-on-E3}
і функція $\Phi : \Omega\rightarrow \mathbb{A}_n^m$ є моногенною в довільній області \, $\Omega\subset E_3$\,.
Тоді:

\emph{1)} функція $\Phi$ є диференціовною за Лорхом в області\, $\Omega$\,;

\emph{2)} похідні Гато $\Phi_G^{(r)}$ є моногенними функціями в $\Omega$ для всіх\, $r$\,.
\end{theorem}

\begin{proof}
Розглянемо довільну точку\, $\zeta_0\in\Omega$\, і кулю\, $\mho\subset\Omega$\, з центром у точці $\zeta_0$.

Оскільки\, $\mho$\, --- опукла множина, то функція\, $\Phi$\, подається у вигляді (\ref{Teor--1RR}) в\, $\mho$.
Звідси випливає, що компоненти $U_k$ розкладу
\begin{equation}\label{3-1:rozklad-Phi-v-bazysi}
\Phi(\zeta)=\sum_{k=1}^n U_k(x,y,z)\,I_k\,.
 \end{equation}
 є $\mathbb{R}$-диференційовними функціями в області\, $\mho$\,, тобто співвідношення
\begin{multline*}
U_k(x+\Delta x,y+\Delta y,z+\Delta z)-U_k(x,y,z)=\\[2mm]
=\displaystyle \frac{\partial U_k(x,y,z)}{\partial x}\,\Delta x+
\frac{\partial U_k(x,y,z)}{\partial y}\,\Delta y+ \frac{\partial
U_k(x,y,z)}{\partial z}\,\Delta z+\\[2mm]
+\,o\left(\sqrt{(\Delta x)^2+(\Delta
y)^2+(\Delta z)^2}\,\right),\qquad (\Delta x)^2+(\Delta
y)^2+(\Delta z)^2\to 0\,,
\end{multline*}
виконуються для всіх\, $(x,y,z)\in\mho$\,.

Тепер диференційовність функції $\Phi$ за Лорхом в області\, $\mho$\, встановлюється
повністю аналогічно до диференційовності функції комплексної змінної за умови,
що її дійсна і уявна частини є диференційовними функціями двох дійсних змінних
(див., наприклад, М.О.~Лаврентьєв і Б.В.~Шабат \cite[с.~21]{L1}).


Використовуючи представлення (\ref{Teor--1RR}) функції $\Phi$ в області\, $\mho$\,,
отримуємо наступний вираз для похідної Гато порядку $r$:
\begin{multline*}
\Phi_G^{(r)}(\zeta)=\sum\limits_{u=1}^mI_u\,\frac{r!}{2\pi i}\int\limits_{\Gamma_u}
F_u(t)\Big((te_1-\zeta)^{-1}\Big)^{r+1}\,dt+\\
+\sum\limits_{s=m+1}^nI_s\,\frac{r!}{2\pi i}\int\limits_
{\Gamma_{u_s}}G_s(t)\Big((te_1-\zeta)^{-1}\Big)^{r+1}\,dt\qquad
\forall\,\zeta\in\mho\,.
\end{multline*}
Більш того, ця похідна є неперервною функцією в області\, $\mho$. Отже, похідна Гато
$\Phi_G^{(r)}$ є моногенною функцією в\, $\mho$\, для будь-якого\, $r$\,.

В силу довільності вибору точки\, $\zeta_0$\, і кулі\, $\mho$\, всі твердження теореми
справделиві в області\, $\Omega$\,.
\end{proof}

Розглянемо питання про аналоги умов Коші--Рімана як необхідних і достатніх умов моногенності
функцій\, $\Phi : \Omega\rightarrow \mathbb{A}_n^m$\,.


\begin{theorem}\label{3-1:teo-CR-An}
Нехай виконується умова \eqref{3-1:cond-on-E3}.
Для того, щоб функція $\Phi : \Omega\rightarrow \mathbb{A}_n^m$ була моногенною в області
$\Omega\subset E_3$, необхідно і достатньо, щоб усі функції
 $U_k:\Omega\rightarrow\mathbb{C}$ в розкладі \eqref{3-1:rozklad-Phi-v-bazysi} були $\mathbb{R}$-диференціовними в
області $\Omega$ і в цій області виконувалися умови
\begin{equation}\label{Umovy_K-R}
\frac{\partial \Phi}{\partial y}=\frac{\partial \Phi}{\partial
x}\,e_{2}\,,\qquad \frac{\partial \Phi}{\partial z}=\frac{\partial
\Phi}{\partial x}\,e_{3}\,.
\end{equation}
\end{theorem}

\begin{proof} {\it Необхідність.} $\mathbb{R}$-диференційовність функцій
$U_k:\Omega\rightarrow\mathbb{C}$ в розкладі (\ref{3-1:rozklad-Phi-v-bazysi}) випливає
з представлення (\ref{Teor--1RR}) моногенної функції $\Phi$ (див. доведення теореми \ref{3-1:teo_pro_naslidky2}).

Вибираючи в рівності (\ref{Gprz}) послідовно елементи $e_{1}=1, e_{2}, e_{3}$ у якості вектора $h$,
отримуємо рівності
\begin{eqnarray*}
 \frac{\partial\Phi}{\partial x}=\Phi'_G(\zeta),\qquad
 \frac{\partial\Phi}{\partial y}=e_2\,\Phi'_G(\zeta),\qquad
\frac{\partial\Phi}{\partial z}=e_3\,\Phi'_G(\zeta),
\end{eqnarray*}
наслідком яких є рівності \eqref{Umovy_K-R}.

{\it Достатність} доводиться повністю аналогічно до того, як це робиться при доведенні
відповідної теореми про диференційовність функцій комплексної змінної
(див., наприклад, М.О.~Лаврентьєв і Б.В.~Шабат \cite[с.~21]{L1}).
\end{proof}

Отже, умови (\ref{Umovy_K-R}) за своєю природою аналогічні до класичних умов Коші--Рімана для
голоморфних функцій комплексної змінної.






%%%%%%%%%%%%%%%%%%%%%%%%%%%%%%
\section{Контурні інтегральні теореми для моногенних функцій в скінченновимірній комутативній асоціативній алгебрі}


Розглянемо лініну оболонку $E_3:=\{\zeta=xe_1+ye_2+ze_3:\,\, x,y,z\in\mathbb{R}\}$,
породжену векторами \eqref{e_1_e_2_e_3}.


Визначимо криволінійний інтеграл в просторі\, $E_3$\,.
Будемо використовувати те саме позначення\, $\gamma$\, для кривої в\, ${\mathbb R}^3$\,
і для конгруентної кривої в\, $E_3$\,.


Для спрямлюваної жорданової кривої $\gamma$ в $\mathbb{R}^{3}$ і
неперервної функції $\Psi:\gamma\rightarrow\mathbb{A}_n^m$, розкладеної за базисом
$\{I_k\}_{k=1}^n$ у вигляді
\begin{equation}\label{3-2:Phi-form}
\Psi(\zeta)=\sum\limits_{k=1}^{n}{U_k(x,y,z)\,I_k}+i\sum
\limits_{k=1}^{n}{V_k(x,y,z)\,I_k},
\end{equation}
 де\, $(x,y,z)\in\gamma$\, і\, $U_k : \gamma\rightarrow\mathbb{R}$\,,\,\, $V_k :
\gamma\rightarrow\mathbb{R}$\,, визначимо інтеграл  вздовж
кривої $\gamma\subset E_3$ рівністю
\begin{multline*}
\int\limits_{\gamma}\Psi(\zeta)\,d\zeta:=\sum\limits_{k=1}^{n}
I_{k}\int\limits_{\gamma}U_{k}(x,y,z)\,dx+
\sum\limits_{k=1}^ne_{2}I_{k}\int\limits_{\gamma}U_{k}(x,y,z)\,dy+\\
+\sum\limits_{k=1}^ne_{3}I_{k}\int\limits_{\gamma}U_{k}(x,y,z)\,dz
+i\sum\limits_{k=1}^nI_{k}\int\limits_{\gamma}V_{k}(x,y,z)\,dx+\\
+i\sum\limits_{k=1}^ne_{2}I_{k}\int\limits_{\gamma}V_{k}(x,y,z)\,dy+
i\sum\limits_{k=1}^ne_{3}I_{k}\int\limits_{\gamma}V_{k}(x,y,z)\,dz\,,
\end{multline*}
де\,\, $d\zeta:=e_1\,dx+e_{2}\,dy+e_{3}\,dz$\,.

Визначимо також поверхневий інтеграл в просторі\, $E_3$\,.
Ми використовуємо те саме позначення\, $\Sigma$\, для поверхні в\, ${\mathbb R}^3$\,
і для конгруентної поверхні в\, $E_3$\,.

Нехай\, $\Sigma$ --- поверхня в\, $\mathbb{R}^{3}$\, з вимірними за Жорданом
проекціями на координатні площини. Для неперервної функції\, $\Psi : \Sigma\rightarrow \mathbb{A}_n^m$,
розкладеної за базисом\, $\{I_k\}_{k=1}^n$\, у вигляді
(\ref{3-2:Phi-form}), де\, $(x,y,z)\in\Sigma$\, і\, $U_k :
\Sigma\rightarrow\mathbb{R}$\,,\,\, $V_k : \Sigma\rightarrow\mathbb{R}$\,,
визначимо інтеграл по поверхні\, $\Sigma$\, з диференціальною формою\,\, $dx\,dy$\,\, рівністю
\[\int\limits_{\Sigma}\Psi(\zeta)\,dx\,dy:=
\sum\limits_{k=1}^nI_{k}\int\limits_{\Sigma}U_{k}(x,y,z)\,dx\,dy+
i\sum\limits_{k=1}^nI_{k}\int\limits_{\Sigma}V_{k}(x,y,z)\,dx\,dy\,.\]
Аналогічно визначаються інтеграли з диференціальними формами
$dydz$ та $dzdx$.


Означення криволінійного і поверхневого інтегралів від
функції гіперкомплексної змінної коректні в тому сенсі, що їх
значення не залежать від вибору допустимих параметризацій
відповідно кривої чи поверхні, оскільки гіперкомплексні інтеграли
визначаються через відповідні дійсні інтеграли.


\subsection{Формула Стокса і теорема Коші для криволінійного інтеграла.}

Наступне твердження містить аналог формули Стокса в алгебрі\, $\mathbb{A}_n^m$\,.


\begin{theorem}\label{3-2:th-Stoksa}
Якщо функція $\Phi:\Omega\rightarrow\mathbb{A}_n^m$ неперервна
разом з частинними похідними першого порядку в області $\Omega\subset E_3$ і $\Sigma$ --- кусково-гладка поверхня в
$\Omega$, край якої  $\gamma$ є кусково-гладкою жордановою кривою,
то справедливий наступний аналог формули Стокса:
\begin{multline*}
\int\limits_{\gamma}\Phi(\zeta)\,d\zeta=\int\limits_{\Sigma}\left(\frac{\partial\Phi}{\partial
x}e_{2}-\frac{\partial\Phi}{\partial y}e_1\right)dx\,dy
+\left(\frac{\partial\Phi}{\partial
y}e_{3}-\frac{\partial\Phi}{\partial z}e_{2}\right)dy\,dz+\\
+\left(\frac{\partial\Phi}{\partial
z}e_1-\frac{\partial\Phi}{\partial x}e_{3}\right)dz\,dx\,.
\end{multline*}
\end{theorem}

Тепер перший крок в доведенні теореми Коші для гіпер\-комп\-лекс\-но\-го криволінійного інтеграла
полягає у використанні неперервності
похідної Гато моногенної функції, яку встановлено в теоремі \ref{3-1:teo_pro_naslidky2}
за умови \eqref{3-1:cond-on-E3}, аналогів умов Коші--Рімана \eqref{Umovy_K-R} і теореми \ref{3-2:th-Stoksa}.
В результаті отримуємо наступне твердження.


\begin{theorem}\label{3-2:teo-int-po-kryv-z-neper-poh}
Нехай виконується умова \eqref{3-1:cond-on-E3}. Нехай
$\Phi : \Omega\rightarrow\mathbb{A}_n^m$ --- моногенна функція в області
$\Omega\subset E_3$ і $\Sigma$ --- кусково-гладка поверхня в $\Omega$,
край якої $\gamma$ є кусково-гладкою жордановою кривою.
Тоді
\begin{equation}\label{3-2:form-Koshi-po-kryv}
\int\limits_{\gamma}\Phi(\zeta)\,d\zeta=0.
\end{equation}
\end{theorem}

\begin{remark}\label{3-2:rem-th-C-M}
Зокрема, рівність (\ref{3-2:form-Koshi-po-kryv}) виконується у випадку, коли\, $\gamma$\, --- межа
$\partial\triangle$ будь-якого трикутника $\triangle$, що міститься в $\Omega$.
Пояснимо, що під трикутником $\triangle$ ми розуміємо плоску фігуру, обмежену
трьома відрізками, що з'єднують три його вершини, а
межа $\partial\triangle$ розглядається у відносній топології площини трикутника.
\end{remark}

Другий крок в доведенні теореми Коші для гіпер\-комп\-лекс\-но\-го криволінійного інтеграла здійснюється
у випадку, коли область $\Omega\subset E_3$ є опуклою і\, $\gamma$\, --- довільна замкнена спрямлювана жорданова
крива в області $\Omega$. У цьому випадку рівність (\ref{3-2:form-Koshi-po-kryv}) для кожної моногенної функції
$\Phi : \Omega\rightarrow\mathbb{A}_n^m$ може бути доведена класичним способом так, як це зробив Е.Р.~Лорх
\cite{Lorch} в опуклій області усієї алгебри.



Через $\gamma[\zeta_1,\zeta_2]$ будемо позначати дугу орієнтованої жорданової кривої $\gamma$,
де $\zeta_{1}$ --- початок цієї дуги і $\zeta_{2}$ --- її кінець.


Нарешті, доведемо наступний гіперкомплексний аналог інтегральної теореми Коші.


\begin{theorem}\label{3-2:teo-int-po-kryv-Blum}
Нехай виконується умова \eqref{3-1:cond-on-E3} і функція\, $\Phi:\Omega\rightarrow\mathbb{A}_n^m$ є моногенною
в області $\Omega\subset E_3$. Тоді для довільної замкненої жорданової спрямлюваної кривої\, $\gamma$\,,
яка гомотопна точці в $\Omega$, справедлива рівніcть \eqref{3-2:form-Koshi-po-kryv}.
\end{theorem}


\begin{proof}
Застосуємо схему доведення теореми 3.2 з роботи Е.~Блю\-ма \cite{Blum}.
Нехай крива $\gamma$ має параметризацію $\zeta=\phi(t)$, $0\leq
t\leq1$, при цьому $\phi(0)=\phi(1)=\zeta_0$, і нехай $\gamma$ --- гомотопна точці $\zeta_0$.
Тоді існує функція $H(s,t)$ двох дійсних змінних\, $s$\, і\, $t$\,, яка  неперервна на квадраті
$Q:=[0,1]\times[0,1]$ і приймає значення в області $\Omega$, така, що
\[H(0,t)=\phi(t),\quad H(1,t)\equiv\zeta_0\qquad\forall\,t\in[0,1],\]
\[H(s,0)=H(s,1)=\zeta_0\qquad\forall\,s\in[0,1].\]

Оскільки функція $H$ --- неперервна на компактній множині $Q$, то образ $K:=\{H(s,t) : (s,t)\in Q\}$
є компактною множиною в $\Omega$.

Позначимо\,\,\, $\rho:=\min\limits_{\zeta'\in K,\,\zeta''\in\partial\Omega}\|\zeta'-\zeta''\|$.

Оскільки функція $H$ --- рівномірно неперервна на множині $Q$, то існує
$\delta>0$ таке, що
\begin{equation}\label{2-3:ner2per}
\|H(s',t')-H(s,t)\|<\rho/2\quad \forall\,(s,t), (s',t') :
|s'-s|<\delta,\,\,|t'-t|<\delta\,.
\end{equation}

Виберемо числа $0=t_0<t_1<\ldots<t_n=1$, що задовольняють нерівності $t_j-t_{j-1}<\delta$,\, $j=1,2,\ldots,n$,
і покладемо
$s_1=t_1$. Позначимо $\zeta_{0,j}:=H(0,t_j)$, $\zeta_{1,j}:=H(s_1,t_j)$ при $j=1,2,\ldots,n-1$\,.
Через $L_j$ позначимо відрізок з початком у точці $\zeta_{0,j}$ і кінцем у точці $\zeta_{1,j}$.

Введемо в розгляд криву $\gamma_1:=\{H(s_1,t) : 0\leq t\leq1\}$.
Не зменшуючи загальності, можемо вважати, що крива $\gamma_1$ є спрямлюваною, оскільки, якщо це необхідно,
$\gamma_1$ може бути замінена гомотопною їй ламаною, яка складається з відрізків, що послідовно сполучають точки
$\zeta_0$, $\zeta_{1,1}, \zeta_{1,2},\dots, \zeta_{1,n}$\,.

В силу нерівності (\ref{2-3:ner2per}) дуги $\gamma[\zeta_0,\zeta_{01}]$,
$\gamma_1[\zeta_0,\zeta_{11}]$ і відрізок $L_1$ містяться в кулі
$B(\zeta_0):=\{\zeta\in E_3 : \|\zeta-\zeta_0\|<\rho\}$. Оскільки $B(\zeta_0)$ --- опукла множина, що міститься в
$\Omega$, то
\begin{equation}\label{2-3:rivn-0K}
\int\limits_{\gamma[\zeta_0,\zeta_{01}]}\Phi(\zeta)d\zeta+
\int\limits_{L_1}\Phi(\zeta)d\zeta=\int\limits_{\gamma_1[\zeta_0,\zeta_{11}]}\Phi(\zeta)d\zeta.
\end{equation}

При $j=1,2,\ldots,n-2$ з нерівності (\ref{2-3:ner2per}) випливають наступні нерівності:
\[\|\zeta-\zeta_{0,j}\|<\rho/2 \qquad \forall\,\zeta\in\gamma[\zeta_{0,j},\zeta_{0,j+1}]\,,\]
\[\|\zeta-\zeta_{1,j}\|<\rho/2 \qquad \forall\,\zeta\in\gamma_1[\zeta_{1,j},\zeta_{1,j+1}]\,, \qquad
\|\zeta_{1,j}-\zeta_{0,j}\|<\rho/2\,,\]
в силу яких дуги $\gamma[\zeta_{0,j},\zeta_{0,j+1}]$, $\gamma_1[\zeta_{1,j},\zeta{1,j+1}]$ і відрізки
$L_j$, $L_{j+1}$ містяться в кулі $B(\zeta_{0,j}):=\{\zeta\in E_3 : \|\zeta-\zeta_{0,j}\|<\rho\}$.
Оскільки $B(\zeta_{0,j})$  --- опукла множина, що міститься в $\Omega$,
то
\begin{multline}\label{2-3:rivn-jK}
-\int\limits_{L_{j}}\Phi(\zeta)d\zeta+
\int\limits_{\gamma[\zeta_{0,j},\zeta_{0,j+1}]}\Phi(\zeta)d\zeta+\int\limits_{L_{j+1}}\Phi(\zeta)d\zeta=\\
=\int\limits_{\gamma_1[\zeta_{1,j},\zeta_{1,j+1}]}\Phi(\zeta)d\zeta,\qquad
j=1,2,\ldots,n-2.
\end{multline}

Нарешті, подібно до рівності (\ref{2-3:rivn-0K}) отримуємо рівність
\begin{equation}\label{2-3:rivn-nK}
-\int\limits_{L_{n-1}}\Phi(\zeta)d\zeta+
\int\limits_{\gamma[\zeta_{0,n-1},\zeta_{0}]}\Phi(\zeta)d\zeta=
\int\limits_{\gamma_1[\zeta_{1,n-1},\zeta_{0}]}\Phi(\zeta)d\zeta.
\end{equation}

Додаючи всі рівності (\ref{2-3:rivn-0K}), (\ref{2-3:rivn-jK}) і (\ref{2-3:rivn-nK}), бачимо, що знищуються
всі інтеграли вздовж відрізків і залишається рівність
\begin{equation}\label{2-3:rivn-int-1}
\int\limits_{\gamma}\Phi(\zeta)d\zeta=\int\limits_{\gamma_1}\Phi(\zeta)d\zeta.
\end{equation}

Далі покладаємо $s_j=t_j$ і вводимо в розгляд криву $\gamma_j:=\{H(s_j,t) : 0\leq t\leq1\}$ при $j=2,3,\ldots,n$.
Подібно до $\gamma_1$, не зменшуючи загальності, можемо вважати, що всі криві $\gamma_j$ є спрямлюваними.

Тепер подібно до рівності (\ref{2-3:rivn-int-1}) отримуємо рівності
\[\int\limits_{\gamma_1}\Phi(\zeta)d\zeta=\int\limits_{\gamma_2}\Phi(\zeta)d\zeta=\ldots=\int\limits_{\gamma_n}\Phi(\zeta)d\zeta.\]

Таким чином, ми отримали рівність
\[\int\limits_{\gamma}\Phi(\zeta)d\zeta=\int\limits_{\gamma_n}\Phi(\zeta)d\zeta,\]
в якій крива $\gamma_n$ вироджується в точку, оскільки  $H(1,t)\equiv\zeta_0$.
Отже,
\[\int\limits_{\gamma_n}\Phi(\zeta)d\zeta=0\,,\]
і рівність (\ref{3-2:form-Koshi-po-kryv}) доведено.
\end{proof}


\subsection{Теорема Морера.}

Через $s[\zeta_1,\zeta_2]$ позначимо відрізок з початком $\zeta_1$ і кінцем $\zeta_2$.

Справедливий наступний аналог теореми Морера для функцій зі значеннями в алгебрі $\mathbb{A}_n^m$.

\begin{theorem}\label{3-2:teo_Morera}
Якщо функція $\Phi:\Omega\rightarrow\mathbb{A}_n^m$ неперервна в області $\Omega\subset E_3$ і задовольняє рівність
\begin{equation} \label{3-2:Morera}
\int\limits_{\partial\triangle}\Phi(\zeta)\,d\zeta=0
\end{equation}
 для кожного трикутника\, $\triangle$\,, що міститься в області\, $\Omega$\,,
 то функція\, $\Phi$\, є моногенною в області\, $\Omega$\,.
\end{theorem}

\begin{proof}
Нехай\, $a$\, --- довільна фіксована точка в\, $\Omega$\,.
Розглянемо функцію
\[\Psi(\zeta):=\int\limits_{s[a,\zeta]}\Phi(\tau)\,d\tau\,.\]
Покажемо, що функція\, $\Psi$\, є моногенною в області\, $\Omega$\, і
\begin{equation}\label{2-3:teo-Morera-f3}
\Psi_G'(\zeta)=\Phi(\zeta).
\end{equation}

Візьмемо $h\in E_3$ і $\delta>0$ такі, що трикутник
$\triangle$ з вершинами $a$, $\zeta$ і $\zeta+\delta h$ міститься в\, $\Omega$\,.

Використовуючи рівність (\ref{3-2:Morera}), перетворюємо різницю
\begin{multline}\label{2-3:teo-Morera-f1}
\Psi(\zeta+\delta h)-\Psi(\zeta)=\int\limits_{s[a, \zeta+\delta h]}\Phi(\tau)\,d\tau\,-
\int\limits_{s[a,\zeta]}\Phi(\tau)\,d\tau=\\[1mm]
=\int\limits_{s[a, \zeta+\delta h]}\Phi(\tau)\,d\tau\,+
\int\limits_{s[\zeta,a]}\Phi(\tau)\,d\tau+\int\limits_{s[\zeta+\delta h,
\zeta]}\Phi(\tau)\,d\tau\,-\int\limits_{s[\zeta+\delta h, \zeta]}\Phi(\tau)\,d\tau=\\[1mm]
=\int\limits_{\triangle}\Phi(\tau)\,d\tau\,+\int\limits_{s[\zeta,
\zeta+\delta h]}\Phi(\tau)\,d\tau=\int\limits_{s[\zeta,
\zeta+\delta h]}\Phi(\tau)\,d\tau\,.
\end{multline}

Використовуючи рівність (\ref{2-3:teo-Morera-f1}), і неперервність функції
$\Phi$ в точці $\zeta$\,, отримуємо співвідношення
\begin{multline}\label{2-3:teo-Morera-f2}
\Biggr\|\frac{\Psi(\zeta+\delta h)-\Psi(\zeta)}{\delta}-\Phi(\zeta)h\Biggr\|=
\frac{1}{\delta}\,\Biggr\|\,\,\int\limits_{s[\zeta,\zeta+\delta h]}\Phi(\tau)\,d\tau-\Phi(\zeta) \delta h\Biggr\|=\\[1.5mm]
=\frac{1}{\delta}\,\Biggr\|\,\,\int\limits_{s[\zeta,
\zeta+\delta
h]}\Big(\Phi(\tau)-\Phi(\zeta)\Big)\,d\tau\Biggr\|\leq\frac{M}{\delta}\,\,\int\limits_{s[\zeta,
\zeta+\delta h]}\|\Phi(\tau)-\Phi(\zeta)\|\,\|d\tau\|\le\\[1.5mm]
\leq\frac{M}{\delta}\,\,\sup\limits_{\tau\in s[\zeta,\zeta+\delta h]}\,\|\Phi(\tau)-\Phi(\zeta)\,\,\|\int\limits_{s[\zeta,
\zeta+\delta h]}\,\|d\tau\|\leq\\[1.5mm]
\leq M\,\|h\|\,\,\sup\limits_{\tau\in s[\zeta,\zeta+\delta h]}\,\|\Phi(\tau)-\Phi(\zeta)\|\rightarrow 0\,,\qquad
\delta\rightarrow 0\,,
\end{multline}
де\, $M$ --- деяка абсолютна стала.

Зі співвідношення (\ref{2-3:teo-Morera-f2}) випливає рівність
\[\lim\limits_{\delta\rightarrow0+0}\frac{\Psi(\zeta+\delta h)-\Psi(\zeta)}{\delta}=\Phi(\zeta)h,\]
наслідком якої є рівність (\ref{2-3:teo-Morera-f3}).

Оскільки в довільному околі точки $\zeta$ функція $\Phi$ є похідною Гато моногенної функції
$\Psi : \Omega\rightarrow\mathbb{A}_{3}$, то в силу теореми \ref{3-1:teo_pro_naslidky2} $\Phi$
є моногенною функцією в області $\Omega$.
\end{proof}




%%%%
\subsection{Інтегральна формула Коші}

Ми використовуємо те саме позначення $L_u$ для множини в $E_3$, що є
конгруентною до прямої (\ref{3-2:L-uR}) в $\mathbb{R}^{3}$.

Нехай $\zeta_0:=x_0e_1+y_0e_2+z_0e_3$ --- довільна точка області
$\Omega\subset E_3$. В околі точки $\zeta_0$, який міститься
в $\Omega$, візьмемо коло $C(\zeta_0)$ з центром в точці
 $\zeta_0$. Через $C_u$ позначимо образ кола $C(\zeta_0)$ при відображенні $f_u$,
$u=1,2,\ldots,m$. Припустимо, що коло $C(\zeta_0)$
\emph{охоплює множину} (див. \cite{Pukh-5})
\begin{equation}\label{3-2:L-zeta_0}
\Big\{\zeta_0+\zeta: \zeta\in\bigcup\limits_{u=1}^m L_u\Big\}.
\end{equation}
 Це означає, що крива
$C_u$ обмежує область $D_u'$ таку, що
$f_u(\zeta_0)\in D_u'$ при\, $u=1,2,\ldots,m$.

Скажемо, що крива\, $\gamma\subset\Omega$\, \emph{один раз
охоплює множину} \eqref{3-2:L-zeta_0} (див. \cite{Pukh-5}), якщо існує коло $C(\zeta_0)$, яке охоплює
вказану множину і гомотопне кривій  $\gamma$ в області
$\Omega\setminus\Big\{\zeta_0+\zeta: \zeta\in\bigcup\limits_{u=1}^m
L_u\Big\}$.


Візьмемо коло\, $C(0)$\, з центром в точці\, $0$\,, яке міститься в\, $E_3$\, і охоплює
множину\, $\bigcup\limits_{u=1}^m L_u$\,.
Оскільки функція\, $\zeta^{-1}$\, --- неперервна на кривій\,
$C(0)$\,, то існує інтеграл
\begin{equation}\label{3-2:lambda}
\lambda:=\int\limits_{C(0)}\tau^{-1}\,d\tau\,.
\end{equation}
В силу теореми \ref{3-2:teo-int-po-kryv-Blum}, значення інтеграла \eqref{3-2:lambda} не залежить від вибору
кола\, $C(0)$\,, яке охоплює множину\, $\bigcup\limits_{u=1}^m L_u$\,.


Наступне твердження містить аналог інтегральної формули Коші для моногенних функцій\,
$\Phi:\Omega\rightarrow\mathbb{A}_{n}^m$\,, заданних в області\, $\Omega\subset E_3$\,.

\vskip 1mm

\begin{theorem}\label{teo-formula-Koshi-3-2}
Нехай виконується умова \eqref{3-1:cond-on-E3} і функція\, $\Phi:\Omega\rightarrow\mathbb{A}_n^m$ є моногенною
в області $\Omega\subset E_3$. Тоді для кожної точки\, $\zeta_{0}\in\Omega$\,
виконується наступна рівність:
\begin{equation}\label{3-2:form-Koshi-3-2}
\lambda\,\Phi(\zeta_{0})=
\int\limits_{\gamma}\Phi(\zeta)\left(\zeta-\zeta_{0}\right)^{-1}\,d\zeta\,,
\end{equation}
де $\gamma$ --- довільна замкнена жорданова спрямлювана крива в\, $\Omega$\,,
яка охоплює один раз множину \eqref{3-2:L-zeta_0}.
\end{theorem}

\begin{proof} 
Для довільного\, $\varepsilon\in(0,1)$\, розглянемо коло
\[C(\zeta_0,\varepsilon):=\{\zeta_0+\varepsilon(\zeta-\zeta_0)\, : \, \zeta\in C(\zeta_0)\}\,,\]
де $C(\zeta_0)$ --- коло, існування якого випливає з того, що крива  $\gamma$
охоплює один раз множину \eqref{3-2:L-zeta_0}.

Оскільки $\gamma$ гомотопна колу $C(\zeta_0,\varepsilon)$ в області\,
$\Omega\setminus\big\{\zeta_0+\zeta: \zeta\in\bigcup\limits_{u=1}^m L_u\big\}$\,, то
з використанням теореми \ref{3-2:teo-int-po-kryv-Blum} отримуємо рівність
   \begin{equation}\label{3-2:form-Koshi-}
\int\limits_{\gamma}\Phi(\zeta)\left(\zeta-\zeta_{0}\right)^{-1}\,d\zeta=
\int\limits_{C(\zeta_0,\varepsilon)}\Phi(\zeta)\left(\zeta-\zeta_{0}\right)^{-1}\,d\zeta\,.
\end{equation}

Далі, представляючи інтеграл в правій частині рівності
(\ref{3-2:form-Koshi-}) у вигляді суми двох інтегралів, отримуємо
\begin{multline}\label{3-2:form-Koshi-sum-int}
\int\limits_{\gamma}\Phi(\zeta)\left(\zeta-\zeta_{0}\right)^{-1}\,d\zeta=
\int\limits_{C(\zeta_0,\varepsilon)}(\Phi(\zeta)-\Phi(\zeta_{0}))\left(\zeta-\zeta_{0}\right)^{-1}\,d\zeta+\\
+\Phi(\zeta_{0})\int\limits_{C(\zeta_0,\varepsilon)}\left(\zeta-\zeta_{0}\right)^{-1}\,d\zeta=
:J_{1}+J_{2}\,.
\end{multline}
Тут\, $J_{2}=\lambda\,\Phi(\zeta_{0})$\, в силу рівності \eqref{3-2:lambda} при\, $\tau=\zeta-\zeta_0$\,.

Підінтегральна функція в інтегралі\, $J_1$\, обмежена константою, яка
не залежить від\, $\varepsilon$. Дійсно, в силу теореми \ref{3-1:teo_pro_naslidky2}
функція\, $\Phi$\, диференційовна за Лорхом в області\, $\Omega$\,
і похідна\, $\Phi_L'=\Phi_G'$\, неперервна в\, $\Omega$\,.
Тому зі співвідношення (\ref{difLorch}) випливає нерівність
\[\|\Phi(\zeta)-\Phi(\zeta_{0})\|\le \,c\varepsilon \qquad \forall\,\zeta\in C(\zeta_0,\varepsilon)\,,\]
де стала\, $c$\, не залежить від\, $\varepsilon$.
Крім того, функція\, $(\zeta-\zeta_{0})^{-1}$\,, яка є неперервною на\, $C(\zeta_0)$\,, є також обмеженою
на\, $C(\zeta_0)$\,. В результаті ми отримуємо оцінку
\[\|(\zeta-\zeta_{0})^{-1}\|\le \, \varepsilon^{-1}\, \max_{\tau\in C(\zeta_0)}\, \|(\tau-\zeta_{0})^{-1}\|
  \le\, c\varepsilon^{-1}  \qquad \forall\,\zeta\in C(\zeta_0,\varepsilon)\,,\]
де стала\, $c$\, не залежить від\, $\varepsilon$. З отриманих оцінок випливає, що функція
$(\Phi(\zeta)-\Phi(\zeta_{0}))(\zeta-\zeta_{0})^{-1}$ обмежена на колі\, $C(\zeta_0,\varepsilon)$\,
константою, яка не залежить від\, $\varepsilon$\,.
Отже, інтеграл\, $J_1$\, прямує до нуля, коли $\varepsilon\rightarrow0$.

Нарешті, переходячи до границі в рівності (\ref{3-2:form-Koshi-sum-int}) при\,
$\varepsilon\rightarrow 0$\,, отримуємо рівність \eqref{3-2:form-Koshi-3-2}\,.
\end{proof}



\vskip 1mm


На відміну від подібних результатів Е.~Лорха \cite{Lorch} і
Е.~Блюма \cite{Blum}, функція
$\Phi:\Omega\rightarrow\mathbb{A}_n^m$ в теоремі \ref{teo-formula-Koshi-3-2}
задана тільки в області
$\Omega$ підпростору $E_{3}$, а не в області з усієї алгебри.
Більш того, зауважимо, що інтегральна формула Коші, встановлена у роботах
\cite{Lorch,Blum}, не застосовна до моногенної функції
$\Phi : \Omega\rightarrow\mathbb {A}_n^m$, оскільки в ній інтегрування здійснюється
вздовж кривої, на якій функція $\Phi$, взагалі кажучи, не визначена.


\vskip 1mm



\begin{theorem}\label{3-2:teo-pro-obr-lambda}
 Стала\, $\lambda$\,, визначена рівністю \eqref{3-2:lambda}, є оборотним елементом в алгебрі\, $\mathbb{A}_n^m$\,.
\end{theorem}

\begin{proof} 
З розкладу (\ref{rozkl-rezol-A_n^mR}) випливає рівність
\begin{equation}\label{3-2:obr-elem}
\zeta^{-1}=\sum\limits_{k=1}^n\widetilde{A}_k\,I_k
\end{equation}
з коефіцієнтами\, $\widetilde{A}_k$\,, що визначаються рівностями
\begin{equation}\label{3-2:A__p}
\begin{array}{ll}
\displaystyle \widetilde{A}_u=\frac{1}{\xi_u}\,,& \qquad u=1,2,\ldots,m\,,\\[4mm]
 \displaystyle
\widetilde{A}_s=\sum\limits_{k=2}^{s-m+1}\frac{\widetilde{Q}_{k,s}}{\xi_{u_s}^k}\,,& \qquad
s=m+1,m+2,\ldots,n\,,
\end{array}
\end{equation}
в яких\, $\widetilde{Q}_{k,s}$\, визначаються рекурентними співвідношеннями:
\begin{multline}\label{3-2:lem_3_2--}
\widetilde{Q}_{2,s}:=-T_{s}\,, \\
\widetilde{Q}_{k,s}=-\sum\limits_{r=k+m-2}^{s-1}\widetilde{Q}_{k-1,r}\,B_{r,\,s}\,,\; \;\;k=3,4,\ldots,s-m+1\,,
\end{multline}
де
\begin{equation}\label{3-2:lem_1_T_p}
T_s:=ya_s+zb_s\,,\qquad s=m+1,m+2,\ldots,n,
\end{equation}
\begin{equation}\label{3-2:lem_1_B_p}
B_{r,s}:=\sum\limits_{p=m+1}^{s-1}T_p \Upsilon_{r,s}^p\,,\qquad s=m+2,m+3,\ldots,n\,,
\end{equation}
а структурні сталі\, $\Upsilon_{r,s}^p$\, і натуральні числа $u_s$ визначені відповідно
в правилах множення 2 і 3 алгебри\, $\mathbb{A}_n^m$\,.


Враховуючи рівність (\ref{3-2:obr-elem}) і співвідношення
\begin{multline*}
d\zeta=dx\,e_1+dy\,e_2+dz\,e_3=\sum\limits_{u=1}^m\Big(dx+a_u\,dy+b_u\,dz\Big)I_u+\\
+\sum\limits_{r=m+1}^n\Big(a_r\,dy+b_r\,dz\Big)I_r
=\sum\limits_{u=1}^md\xi_u\,I_u+\sum\limits_{r=m+1}^n dT_r\,I_r\,,
\end{multline*}
отримуємо наступну рівність:
\begin{multline}\label{3-2:obr-elem-0}
\zeta^{-1}\,d\zeta=\sum\limits_{u=1}^m\widetilde{A}_u\,d\xi_u\,I_u+
\sum\limits_{r=m+1}^n\widetilde{A}_{u_r}\,dT_r\,I_r+\\
+\sum\limits_{s=m+1}^n\widetilde{A}_{s}\,d\xi_{u_s}\,I_s+
\sum\limits_{s=m+1}^n\sum\limits_{r=m+1}^n\widetilde{A}_{s}\,dT_r\,I_sI_r=:
\sum\limits_{k=1}^n\sigma_k\,I_k\,.
\end{multline}

Тепер, враховуючи рівності (\ref{3-2:obr-elem-0}) і (\ref{3-2:A__p}), обчислюємо
$$\int\limits_{C(0)}\sum\limits_{u=1}^m\sigma_u\,I_u=
\sum\limits_{u=1}^mI_u\int\limits_{C_u(0)}\frac{d\xi_u}{\xi_u}=
2\pi i\sum\limits_{u=1}^mI_u=2\pi i\,,
$$
де\, $C_u(0)$\, --- образ кола\, $C(0)$\, при відображенні\, $f_u$\,.

Тому
\begin{equation*}
\lambda=2\pi
i+\sum\limits_{k=m+1}^nI_k\int\limits_{C(0)}\sigma_k\,,
\end{equation*}
і\, $\lambda$\, є оборотним елементом.
\end{proof}

\vskip 1mm

Отже, інтегальна формула Коші (\ref{3-2:form-Koshi-3-2}) може бути переписана у вигляді
\begin{equation}\label{3-2:form-Koshi-3-2-ober-lambda}
\Phi(\zeta_{0})=
\lambda^{-1}\int\limits_{\gamma_{\zeta}}\Phi(\zeta)\left(\zeta-\zeta_{0}\right)^{-1}d\zeta\,.
\end{equation}

В деяких спеціальних випадках (див. \cite{Pl-Shp3,Pl-Shp-Al,Pukh-5}) встановлено,
що
\begin{equation}\label{3-2:lambda-}
\lambda=2\pi i\,,
\end{equation}
як і в комплексній площині.

Очевидно, що рівність (\ref{3-2:lambda-}) виконується тоді і тільки тоді, коли
\begin{equation}\label{3-2:obr-elem-1}
\int\limits_{C(0)}\sigma_k=0\qquad \forall\;k=m+1,\ldots,n\,,
\end{equation}
але умови \eqref{3-2:obr-elem-1} важко перевірити в загальному випадку.


Вкажемо один спеціальний випадок, коли умови \eqref{3-2:obr-elem-1} легко перевіряются і
рівність (\ref{3-2:lambda-}) виконується.
Використаємо представлення алгебри\, $\mathbb{A}_n^m$\, у вигляді напівпрямої суми\,
$\mathbb{A}_n^m=\mathcal{S}\oplus_s \mathcal{N}$\,, де\, $\mathcal{S}$ --- $m$-вимірна напівпроста підалгебра
і\, $\mathcal{N}$ --- $(n-m)$-вимірна нільпотентна підалгебра.


\vskip 1mm

\begin{theorem}\label{3-2:teo-pro-napivprostu-alg-}
Нехай $\mathbb{A}_n^m=\mathcal{S}\oplus_s \mathcal{N}$ і $E_3\subset \mathcal{S}$.
Тоді виконується рівність \eqref{3-2:lambda-}.
 \end{theorem}

\begin{proof} 
З умови\, $E_3\subset \mathcal{S}$\, випливає, що\, $a_k=b_k=0$\, для всіх\, $k=m+1,\ldots,n$\,
в розкладі (\ref{e_1_e_2_e_3}).
Тому для довільного\, $\zeta\in E_3$\,, усі функції\, $T_s$\, і\, $B_{r,s}$\, рівні нулю
у співвідношеннях \eqref{3-2:lem_1_T_p} і \eqref{3-2:lem_1_B_p}.

Тепер, враховуючи співвідношення (\ref{3-2:lem_3_2--}), робимо висновок про те, що в рівностях (\ref{3-2:A__p}),
$\widetilde{A}_s=0$\, при\, $s=m+1,\ldots,n$\,. Звідси миттєво випливають рівності\,
$\sigma_k=0$\, при\, $k=m+1,\ldots,n$\, і усіх\, $\zeta\in E_3$\,.
Отже, умови \eqref{3-2:obr-elem-1} задовольняються і рівність (\ref{3-2:lambda-}) виконується.
\end{proof}

\vskip 1mm

Зазначимо, що у випадку тривимірної лінійної оболонки\, $E_3$\,
теорема \ref{3-2:teo-pro-napivprostu-alg-} узагальнює теорему 6
роботи \cite{Pukh-5}, доведену для напівпростих алгебр, на алгебри\, $\mathbb{A}_n^m$\,, які, взагалі кажучи,
не є напівпростими.


Якщо крива інтерування в рівності \eqref{3-2:lambda} не охоплює множину\, $\bigcup\limits_{u=1}^m L_u$\,,
то інтеграл в \eqref{3-2:lambda} може бути необоротним елементом алгебри\, $\mathbb{A}_n^m$\,.
Це підтверджує наступний приклад.

\begin{example}\label{3-2:ex-ne2pi}
Розглянемо алгебру\, ${\mathbb A_2}$\, (див. \cite{PlPu13}) з базисом $\{\mathcal{I}_1,
\mathcal{I}_2, \rho\}$ і таблицею множення:
\[\mathcal{I}_1^2=\mathcal{I}_1,\,\,
\mathcal{I}_2^2=\mathcal{I}_2,\,\,
\mathcal{I}_1\mathcal{I}_2=0,\,\, \rho^2=0,\,\,
\mathcal{I}_1\rho=0,\,\, \mathcal{I}_2\rho=\rho\,.\]
Тут базис складається з двох ідемпотентних елементів\, $I_1=\mathcal{I}_1$\,, $I_2=\mathcal{I}_2$\, і нільпотентного
елемента\, $I_3=\rho$\,, тобто\, $n=3$\, і\, $m=2$\,.

Розглянемо інший базис в алгебрі\, ${\mathbb A_2}$:
$$e_1=1=\mathcal{I}_1+\mathcal{I}_2, \quad e_2=i\mathcal{I}_1+\rho,\quad e_3=i\mathcal{I}_2\,.$$

Для\, $\zeta=xe_1+ye_2+ze_3$\, маємо
$$\xi_1=x+iy\, \quad \mbox{і} \quad   \xi_2=\xi_{u_3}=x+iz\,.$$

Обернений елемент (\ref{3-2:obr-elem}) має вигляд
$$\zeta^{-1}=\frac{1}{\xi_1}\,\mathcal{I}_1+\frac{1}{\xi_2}\,\mathcal{I}_2-\frac{y}{\xi_2^2}\,\rho\,,$$
так що всі необоротні елементи\, $\zeta\in E_3$ розміщуються на двох координатних прямих\,
$L_1=\{ze_3 : z\in\mathbb{R}\}$\, і\, $L_2=\{ye_2 : y\in\mathbb{R}\}$\,.


Візьмемо коло
\[C^{y,R}:=\{\tau=xe_1+e_2+ze_3 : x^2+z^2=R^2\}\,,\]
яке охоплює пряму\, $L_2$\,, але не охоплює пряму\, $L_1$\,.

Для $\zeta\in C^{y,R}$ маємо
$$d\zeta=dx\,e_1+dz\,e_3=dx\,\mathcal{I}_1+d\xi_2\,\mathcal{I}_2\,,$$
і
\begin{multline*}
\zeta^{-1}\,d\zeta=\biggl(\frac{1}{x+i}\,\mathcal{I}_1+\frac{1}{\xi_2}\,\mathcal{I}_2-\frac{1}{\xi_2^2}\,\rho\biggr) 
(dx\,\mathcal{I}_1+d\xi_2\,\mathcal{I}_2)=\\
=\frac{dx}{x+i}\,\mathcal{I}_1+\frac{d\xi_2}{\xi_2}\,\mathcal{I}_2-\frac{d\xi_2}{\xi_2^2}\,\rho\,.
\end{multline*}

Легко обчислюється інтеграл
$$\int\limits_{C^{y,R}}\zeta^{-1}\,d\zeta= 2\pi i\,\mathcal{I}_2\,,$$
тобто цей інтеграл є необоротним елементом алгебри\, $\mathbb{A}_n^m$\,.
\end{example}



\subsection{Теорема Тейлора}

Виконуючи розклад функції (\ref{3-2:form-Koshi-3-2-ober-lambda}) у степеневий ряд подібно
до розкладу голоморфних функцій, що базується на розкладі в степеневий ряд ядра Коші
(див., наприклад, О.І.~Маркушевич \cite[с.~298]{Markush-v1}), отримуємо наступне твердження.


\begin{theorem}\label{3-2:teo-Taylor-An}
Нехай виконується умова \eqref{3-1:cond-on-E3} і функція\, $\Phi:\Omega\rightarrow\mathbb{A}_n^m$ є моногенною
в області $\Omega\subset E_3$. Тоді\, $\Phi$\, є аналітичною в області\, $\Omega$\,,
тобто в деякому околі кожної точки\, $\zeta_0\in\Omega$\, вона може бути пред\-став\-ле\-на у вигляді
суми збіжного степеневого ряду
\begin{equation}
\label{3-2:st-riad}
 \Phi(\zeta)=\sum\limits_{k=0}^{\infty} c_k \  (\zeta-\zeta_0)^k,
\end{equation}
де
\[c_n=\frac{\Phi_G^{(n)}(\zeta_0)}{n!}=\lambda^{-1}
\int\limits_{\gamma}\Phi(\tau)\Bigl((\tau-\zeta_0)^{-1}\Bigr)^{n+1}\,d\tau\,,\quad n=0,1,\dots,\]
і $\gamma$ --- довільна замкнена жорданова спрямлювана крива в\, $\Omega$\,,
яка охоплює один раз множину \eqref{3-2:L-zeta_0}.
\end{theorem}


\subsection{Еквівалентні означення моногенних функцій.}


Наступна теорема, що містить різні еквівалентні означення моногенної функції,
є аналогом класичної теореми комплексного аналізу про різні еквівалентні означення голоморфних функцій.

\vskip 1mm


\begin{theorem} \label{3-2:eqv-opr-mon-fun}
Нехай виконується умова \eqref{3-1:cond-on-E3}.
Функція\,\, $\Phi : \Omega\rightarrow\mathbb{A}_n^m$\,\,  є моногенною в
довільній області\,\, $\Omega\subset E_3$\, тоді і тільки тоді, коли
виконується одна з наступних умов:

\emph{(I)} усі компоненти $U_k:\Omega\rightarrow\mathbb{C}$ розкладу \eqref{3-1:rozklad-Phi-v-bazysi}
є $\mathbb{R}$-диференційовними функціями в області $\Omega$ і в цій області виконуються умови
\eqref{Umovy_K-R};

\emph{(II)} функція $\Phi$ є аналітичною в області\,
$\Omega$\,, тобто
для кожної точки $\zeta_0\in \Omega$ існує окіл, в якому функція
$\Phi$ представляється   у вигляді суми збіжного степеневого ряду \eqref{3-2:st-riad}
з коефіцієнтами\, $c_k$\,,  що належать алгебрі\,
$\mathbb{A}_n^m$\,;

\emph{(III)} функція  $\Phi$ неперервна в області $\Omega$ і
виконується рівність \eqref{3-2:Morera}
для кожного трикутника $\triangle$, що міститься в області $\Omega$;


\emph{(IV)} Функція $\Phi$ є диференційовною за Лорхом в області\, $\Omega$\,;

\emph{(V)} в кожній кулі\, $\mho\subset\Omega$\,
існують\, $m$\, голоморфних функцій\, $F_u$\, в областях\, $D_u:=\{f_u(\zeta) : \zeta\in\mho\}$\,, $u=1,2,\dots,m$\,,
і\, $n-m$\, голоморфних функцій\, $G_s$\, в областях\, $D_{u_s}$\,, $s=m+1,m+2,\dots,n$\,, таких,
що функція $\Phi$ представляється у вигляді \eqref{Teor--1RR},
де\, $\Gamma_u$ --- замкнена жорданова спрямлювана крива, яка лежить в області $D_u$, охоплює  точку\,
$f_u(\zeta)$\, і не містить точок\, $f_q(\zeta)$\,, $q=1,2,\ldots, m$,\, $q\neq u$.
\end{theorem}

\begin{proof} 
Еквівалентність умови (I) і моногенності функції $\Phi$ доведено в теоремі \ref{3-1:teo-CR-An}.
Еквівалентність умови (II) і моногенності функції $\Phi$ є наслідком теореми
\ref{3-2:teo-Taylor-An} і властивості збіжного степеневого ряду
(\ref{3-2:st-riad}) визначати моногенну функцію в області збіжності.
Еквівалентність умови (III) і моногенності функції $\Phi$ випливає з теорем
\ref{3-2:teo-int-po-kryv-Blum} і \ref{3-2:teo_Morera}. Еквівалентність умови (IV) і моногенності функції
$\Phi$ випливає з першого твердження теореми \ref{3-1:teo_pro_naslidky2}. Нарешті,
еквівалентність умови (V) і моногенності функції $\Phi$ випливає з теореми
\ref{Shpak-costr-op}.
\end{proof}


Теорема \ref{3-2:eqv-opr-mon-fun} узагальнює результати робіт
\cite{Pl-Shp1,Pl-Shp2,Pl-Shp3,Plaksa12,PlPu13,Pl-Shp-Al,Pukh-5},
встановлені для моногенних функцій в конкретних скінченновимірних алгебрах.




%%%%%%%%%%%%%%%%%%%%%%%%%%%%%%
\section{Теорема Коші для поверхневого інтеграла в скінченновимірній комутативній асоціативній алгебрі}


Інтегральна теорема Коші є фундаментальним результатом класичного
комплексного аналізу в комплексній площині $\mathbb{C}$:
якщо межа $\partial D$ області $D\subset\mathbb{C}$ є замкненою спрямлюваною жордановою кривою
і функція $F\colon \overline{D}\longrightarrow \mathbb{C}$ --- неперервна в замиканні $\overline{D}$ області $D$ 
і голоморфна в $D$, то
$$\int\limits_{\partial D}F(z)dz=0\,.$$


Розвиток гіперкомплексного аналізу як в комутативних, так і
некомутативних алгебрах потребує аналогічних загальних аналогів
інтегральної теореми Коші для багатовимірних просторів.


Добре відомо, що у випадку, коли однозв'язна область
має замкнену кусково-гладку межу, просторові аналоги інтегральної теореми Коші
можна отримати за допомогою класичної формули Гаусса--Остроградського  за умови,
що задана функція має неперервні частинні похідні першого порядку, які неперервно продовжуються на
межу області. У такий спосіб аналоги інтегральної теореми Коші
доведено в алгебрі кватерніонів (див., наприклад, монографію В.В.~Кравченка і М.В.~Шапіро
\cite[с.~66]{Krav-Shap}) і в алгебрах Кліффорда (див., наприклад, монографію
Ф.~Брекса, Р.~Деланга і Ф.~Соммена \cite[с.~52]{Brakx}).

Узагальнення інтегральної теореми Коші полягають у
послабленні вимог до межі або до заданої функції.
Як правило, такі узагальнення базуються на узагальненій Гаусса--Остроградського--Гріна--Стокса
формулі (див., наприклад,  Г.~Федерер \cite{Federer-3-3} або Дж.~Харрісон і А.~Нортон \cite{Harrison-Norton-3-3}) 
за умови неперервності частинних похідних заданої функції, але для розширених класів поверхонь інтегрування; див.,
наприклад,, Р.~Абреу Блайя і Х.~Борі Рейєс \cite{Abreu-Bory-99-3-3}, а також
Р.~Абреу Блайя, Д.~Пена Пена і Х.~Борі Рейєс
\cite{Abreu-Bory-Pena-3-3}, де розглядаються спрямлювані або регулярні поверхні. 
В роботах А.~Садбері \cite{Sudbery} і О.Ф.~Геруса \cite{Gerus-2011}, 
неперервність частинних похідних замінено диференційовністю компонент
заданої функції, що приймає значення в алгебрі кватерніонів. Зазначимо, що в роботі \cite{Gerus-2011}
межа області залишається кусково-гладкою.


Далі ми доведемо аналог інтегральної теореми Коші для поверхневого інтеграла від гіперголоморфної
функції, що задана в області тривимірного простору і
приймає значення в довільній $n$-вимірній комутативній асоціативній алгебрі, де\, $3\leq n<\infty$\,. 
Зазначимо, що моногенні функції в гармонічних алгебрах утворюють підмножину гіперголоморфних функцій, які, крім того,
можуть бути заданими в областях, межі яких не є кусково-гладкими.

Аналогічний результат опубліковано в роботі \cite{Pla-Shp-CVEE-2014} для гіперголоморфних функцій за умови,
що задана комутативна алгебра розглядається над полем комплексних чисел. Проте, як випливає з наведеного далі доведення, 
така умова є неістотною. Зазначимо також, що подібний аналог інтегральної теореми Коші доведено 
О.Ф.~Герусом \cite{Gerus-2018} 
для гіперголоморфних функцій, що приймають значення в некомутативній алгебрі кватерніонів.





%%%%%%%%%%%%%%%%%%
\subsection{Поверхневі інтеграли по  квадровних поверхнях}


Роз\-гля\-не\-мо поняття квадровної поверхні в $\mathbb{R}^{3}$.

Множина $\Sigma$ називається \textit{поверхнею} у просторі
$\mathbb{R}^{3}$, якщо $\Sigma$ є гомеоморфним образом квадрата
$G:=[0,1]\times [0,1]$ (див., наприклад, \cite[c.~24]{Rado}).

Через $\Sigma^\varepsilon$ позначимо
$\varepsilon$-окіл поверхні $\Sigma$, тобто
множину 
\begin{multline*}
\Sigma^\varepsilon:=\{(x,y,z)\in\mathbb{R}^3:
\sqrt{(x-x_1)^2+(y-y_1)^2+(z-z_1)^2}\leq\varepsilon,\\
 \quad (x_1,y_1,z_1)\in \Sigma\}.
\end{multline*} 

\textit{Відстанню Фреше} $d(\Sigma,\Lambda)$ між поверхнями $\Sigma$
і $\Lambda$ називається інфімум дійсних чисел $\varepsilon$, для
яких виконуються співвідношення $\Sigma\subset \Lambda^\varepsilon$,
$\Lambda\subset \Sigma^\varepsilon$ (див., наприклад,
\cite{Freshe}). Послідовність багатогранників  $\Lambda_n$
називається \textit{рівномірно збіжною} до поверхні $\Sigma$, якщо
$d(\Lambda_n,\Sigma)\rightarrow0$ при $n\rightarrow\infty$ (див.,
наприклад, \cite[c. 121]{Rado}).

\textit{Площею Лебега} поверхні $\Sigma$ називається величина
$$\mathfrak{L}(\Sigma):=\inf\liminf\limits_{n\rightarrow\infty}
\mathfrak{L}(\Lambda_n),$$ де інфімум береться по усіх
послідовностях $\Lambda_n$, рівномірно збіжних до $\Sigma$ (див.,
наприклад, \cite[c. 468]{Rado}), а $\mathfrak{L}(\Lambda_n)$ ---
площа багатогранника $\Lambda_n$.


Нехай поверхня $\Sigma$ має скінченну площу Лебега. Тоді за
теоремою Л.~Чезарі \cite[с.~7]{Cesari}
 існує параметризація поверхні
$$
\Sigma=\left\{f(u,v):=\big(x(u,v),\,y(u,v),\,z(u,v)\big):(u,v)\in
G\right\}
$$
така, що якобіани
\begin{equation}\label{Jacobiany}
A:=\frac{\partial y}{\partial u}\frac{\partial z}{\partial
v}-\frac{\partial y}{\partial v}\frac{\partial z}{\partial u},
\quad B:=\frac{\partial z}{\partial u}\frac{\partial x}{\partial
v}-\frac{\partial z}{\partial v}\frac{\partial x}{\partial
u},\quad C:=\frac{\partial x}{\partial u}\frac{\partial
y}{\partial v}-\frac{\partial
x}{\partial v}\frac{\partial y}{\partial u} 
\end{equation}
існують майже всюди на квадраті $G:=[0;1]\times [0;1]$ і
\begin{equation}\label{riv-pl-leb}
\mathfrak{L}(\Sigma)=\int\limits_{G}\sqrt{A^2+B^2+C^2}\,dudv.
\end{equation}

У випадку, коли $\mathfrak{L}(\Sigma)<\infty$ і рівність
(\ref{riv-pl-leb}) виконується для заданої параметризації
$\Sigma$, поверхню  $\Sigma$ будемо називати {\it квадровною}.

Сформулюємо деякі достатні умови квадровності поверхні $\Sigma$.

 \begin{itemize}
\item 
Якщо $\Sigma$ --- спрямлювана поверхня (тобто, ліпшицевий образ квадрата),
 то з результатів Т.~Радо \cite[IV.4.28, IV.4.1~(e)]{Rado} випливає, що поверхня $\Sigma$ квадровна.

\item 
Нехай компоненти $x(u,v),\,y(u,v),\,z(u,v)$ відображення
$f$ --- абсолютно неперервні за Тонеллі (див., наприклад,
\cite[с.~169]{Saks}). Нехай, крім того, в якобіанах $A,B,C$
відображення $f$ в кож\-но\-му з добутків $\frac{\partial y}{\partial
u}\frac{\partial z}{\partial v}$, $\frac{\partial y}{\partial
v}\frac{\partial z}{\partial u}$, $\frac{\partial z}{\partial
u}\frac{\partial x}{\partial v}$, $\frac{\partial z}{\partial
v}\frac{\partial x}{\partial u}$, $\frac{\partial x}{\partial
u}\frac{\partial y}{\partial v}$, $\frac{\partial x}{\partial
v}\frac{\partial y}{\partial u}$ одна час\-тин\-на похідна належить
класу інтегровних функцій $L_p(G)$ на $G$, а інша частинна похідна
належить $L_q(G)$, де $\frac{1}{p}+\frac{1}{q}=1$. Тоді поверхня $\Sigma$
квадровна (див. Т.~Радо \cite[V.2.26]{Rado}). Від\-зна\-чи\-мо, що для спрямлюваної
поверхні $\Sigma$ компоненти $x(u,v)$,\, $y(u,v)$,\, $z(u,v)$
відображення $f$ абсолютно неперервні за  Тонеллі (див., наприклад,
\cite[с.~169]{Saks}).

\item 
Якщо дві компоненти відображення $f(u,v)$ є функціями Ліпшиця, а третя компонента ---
абсолютно неперервна за Тонеллі, то поверхня $\Sigma$ квадровна (див. Т.~Радо
\cite[V.2.28]{Rado}).
\end{itemize}

Тепер визначимо поверхневі інтеграли по квадровних поверхнях.

{\it Замкнену поверхню} $\Gamma\subset\mathbb{R}^3$ розуміємо як
образ сфери при гомеоморфному відображенні, яке відображає {\it
хоча б одне коло на спрялювану криву}.

Отже, замкнена
поверхня $\Gamma$ подається як об'єднання двох поверхонь $\Gamma_1$,
$\Gamma_2$, для яких $\Gamma_1\cap\Gamma_2=:\gamma$ є замкненою
жордановою спрямлюваною кривою. 

Нехай поверхні $\Gamma_1$, $\Gamma_2$ задані параметрично:
$$
\Gamma_1=\left\{f_1(u,v):=\big(x_1(u,v),\,y_1(u,v),\,z_1(u,v)\big):(u,v)\in
G\right\},
$$
$$
\Gamma_2=\left\{f_2(u,v):=\big(x_2(u,v),\,y_2(u,v),\,z_2(u,v)\big):(u,v)\in
G\right\}.$$

Замкнена поверхня $\Gamma$ називається {\it квадровною}\,, якщо
поверхні $\Gamma_1$ і $\Gamma_2$ квадровні.

Для замкненої квадровної поверхні $\Gamma$ і неперервної функції
$F:\Gamma\rightarrow\mathbb{R}$ визначимо інтеграли по $\Gamma$
рівностями
$$\int\limits_{\Gamma}F(x,y,z)\,dydz:=\int\limits_{G}F\Big(x_1(u,v),y_1(u,v),z_1(u,v)\Big)A_1
\,dudv-$$
\begin{equation}\label{int-1}
-\int\limits_{G}F\Big(x_2(u,v),y_2(u,v),z_2(u,v)\Big)A_2\,dudv,
\end{equation}
$$\int\limits_{\Gamma}F(x,y,z)\,dzdx:=\int\limits_{G}F\Big(x_1(u,v),y_1(u,v),z_1(u,v)\Big)B_1
\,dudv-$$
\begin{equation}\label{int-2}
-\int\limits_{G}F\Big(x_2(u,v),y_2(u,v),z_2(u,v)\Big)B_2\,dudv,
\end{equation}
$$\int\limits_{\Gamma}F(x,y,z)\,dxdy:=\int\limits_{G}F\Big(x_1(u,v),y_1(u,v),z_1(u,v)\Big)C_1
\,dudv-$$
\begin{equation}\label{int-3}
-\int\limits_{G}F\Big(x_2(u,v),y_2(u,v),z_2(u,v)\Big)C_2\,dudv
\end{equation}
з якобіанами $A_k,B_k,C_k$ відображення $f_k$ вигляду
(\ref{Jacobiany}) при $k=1,2$.

Легко переконатися у коректності означень (\ref{int-1})
--- (\ref{int-3}). Справді, значення правих частин інтегралів в рівностях
 (\ref{int-1}) --- (\ref{int-3}) однакові для усіх параметризацій $f_1$, $f_2$, для яких площі
$\mathfrak{L}(\Gamma_1)$, $\mathfrak{L}(\Gamma_2)$ подаються
рівностями вигляду (\ref{riv-pl-leb}), і значення правих частин
інтегралів в рівностях (\ref{int-1}) --- (\ref{int-3}) не залежать
від вибору спрямлюваної кривої $\gamma$, яка розбиває $\Gamma$ на
дві частини.\vskip 2mm


\begin{lemma}\label{lema-1} 
Якщо $\Gamma$ замкнена квадровна поверхня, то
\begin{equation}\label{lem-3-int-rivni-nulyu}
\int\limits_{\Gamma}dydz=\int\limits_{\Gamma}dzdx=\int\limits_{\Gamma}dxdy=0.
\end{equation}
\end{lemma}


\begin{proof}
 За означенням
\begin{equation}\label{lem-3-1}
\int\limits_{\Gamma}dydz=\int\limits_{G}A_1\,dudv-\int\limits_{G}A_2\,dudv.
\end{equation}
З результатів Т.~Радо \cite[V.2.64~$(iii)$, IV.4.21~$(iii_3)$]{Rado}
випливає, що для поверхонь $\Gamma_1$, $\Gamma_2$ справедливі
наступні рівності:
\begin{equation}\label{lem-3-2}
\int\limits_{G}A_k\,dudv=\int\limits_{\partial G}ydz,\qquad k=1,2,
\end{equation}
де інтеграл в правій частині розуміємо як інтеграл Лебега--Стілтьєса,
який беремо по межі $\partial G$ квадрата $G$ в додатному напрямку.
Тепер з рівностей (\ref{lem-3-1}), (\ref{lem-3-2}) плививає, що перший
інтеграл в рівності (\ref{lem-3-int-rivni-nulyu}) дорівнює нулю. Інші
рівності (\ref{lem-3-int-rivni-nulyu}) доводяться аналогічно.
\end{proof}

\vskip 2mm



\subsection{Гіперголоморфні функції. Допоміжні твердження}


Нехай тепер\, $\mathbb{A}$ --- комутативна асоціативна алгебра над полем дійсних чисел\, $\mathbb{R}$\, з базисом\,
$\{e_{k}\}_{k=1}^{n}$\,,\, $3\leq n<\infty$\,.
Виділимо лінійний підпростір\,
$E_{3}:=\{\zeta=xe_1+ye_{2}+ze_{3}: x, y, z\in\mathbb{R}\}$\,,
породжений векторами\, $e_1,e_2,e_3$\,. Будемо використовати однакове позначення\, $\Omega$\, 
для множини в\, ${\mathbb R}^3$\,  і для конгруентної множини в\, $E_3$\,.

Розглянемо функцію $\Psi:\Omega\rightarrow\mathbb{A}$, розкладену за базисом\, $\{e_{k}\}_{k=1}^{n}$\, у вигляді
\begin{equation}\label{Phi-form}
\Psi(\zeta)=\sum\limits_{k=1}^{n}U_{k}(x,y,z)e_{k}\,,
\end{equation}
де\, $(x,y,z)\in\Omega$\, і\, $U_k :
\Omega\rightarrow\mathbb{R}$\,.

Будемо казати, що функція $\Psi: \Omega\rightarrow\mathbb{A}$
є \textit{гіперголоморфною}\/ в області $\Omega\subset E_3$, якщо
її дійснозначні компоненти розкладу (\ref{Phi-form}) є
диференційовними функціями в $\Omega$ і виконується
наступна умова в кожній точці області $\Omega$:
\begin{equation}\label{GiperGol}
\frac{\partial\Psi}{\partial x}\,e_1+\frac{\partial\Psi}{\partial
y}\,e_{2}+\frac{\partial\Psi}{\partial z}\,e_{3}=0.
\end{equation}




В науковій літературі використовуються різні назви для функцій, які
задовольняють рівняння вигляду (\ref{GiperGol}). Наприклад, в
роботах А.~Садбері \cite{Sudbery}, Ф.~Коломбо, І.~Сабадіні і Д.~Струппи \cite{Colombo},
В.~Шпрьоссіга \cite{Spr} такі функції називаються
регулярними, а в роботах Ф.~Брекса, Р.~Деланга і Ф.~Соммена \cite{Brakx}, С.~Бернштейн \cite{Bernstein},
Дж.~Райана \cite{Ryan} --- моногенними функціями. 
Ми будемо використовувати термінологію робіт В.В.~Кравченка і М.В.~Шапіро \cite{Krav-Shap}
В.~Шпрьоссіга \cite{Spros}, О.Ф.~Геруса \cite{Gerus-2011}.



Нехай $\Omega$ --- обмежена замкнена множина в $\mathbb{R}^{3}$. Для
неперервної функції $\Psi:\Omega_{\zeta}\rightarrow\mathbb{A}$, розкладеної за базисом $\{e_{k}\}_{k=1}^{n}$
у вигляді (\ref{Phi-form}), означимо об'ємний інтеграл рівністю
$$\int\limits_{\Omega}\Psi(\zeta)\,dx\,dy\,dz:=\sum\limits_{k=1}^{n}
e_{k}\int\limits_{\Omega}U_{k}(x,y,z)\,dx\,dy\,dz\,.$$


Нехай $\Gamma$ --- замкнена квадровна поверхня в $\mathbb{R}^{3}$.
Для неперервної функції $\Psi:\Gamma_{\zeta}\rightarrow
\mathbb{A}$, розкладеної за базисом $\{e_{k}\}_{k=1}^{n}$
у вигляді (\ref{Phi-form}), де $(x,y,z)\in\Gamma$ і
$U_k : \Gamma\rightarrow\mathbb{R}$, означимо поверхневий інтеграл по
$\Gamma$ з диференціальною формою\,
$\sigma:=dy\,dz\,e_1+dz\,dx\,e_{2}+dx\,dy\,e_{3}$\, рівністю
\begin{multline*}
 \int\limits_{\Gamma}\Psi(\zeta)\sigma:=
 \sum\limits_{k=1}^{n}e_1e_{k}\int\limits_{\Gamma}U_{k}(x,y,z)\,dy\,dz+
\sum\limits_{k=1}^{n}e_{2}e_{k}\int\limits_{\Gamma}U_{k}(x,y,z)\,dz\,dx+\\
+\sum\limits_{k=1}^{n}e_{3}e_{k}\int\limits_{\Gamma}U_{k}(x,y,z)\,dx\,dy\,,
\end{multline*}
де інтеграли в правій частині рівності визначені рівностями
(\ref{int-1}) --- (\ref{int-3}).


Наступна лема є наслідком леми \ref{lema-1} і означення диференціальної форми
$\sigma$.\vskip 2mm


\begin{lemma}\label{lema-2}
Якщо $\Gamma$ --- замкнена квадровна поверхня, то
\begin{equation}\label{int-sigma}
\int\limits_{\Gamma}\sigma=0\,.
\end{equation}
\end{lemma}
\vskip 2mm

Введемо евклідову норму
\[\|a\|:=\biggl(\sum\limits_{k=1}^n|a_k|^2\biggr)^{1/2}\] 
в алгебрі $\mathbb{A}$, де $a=\sum\limits_{k=1}^na_k e_k$ і
$a_k\in\mathbb{R}$ при $k=\overline{1,n}$.

Нехай $\Gamma$ --- замкнена квадровна поверхня в $\mathbb{R}^{3}$.
Для неперервної функції $U : \Gamma\rightarrow \mathbb{R}$,
означимо поверхневий інтеграл по $\Gamma$ з диференціальною
формою $\|\sigma\|$ рівністю 
\begin{multline*}
\int\limits_{\Gamma}U(xe_1+ye_{2}+ze_{3})\|\sigma\|:=\\
:=\int\limits_{G}U\Big(x_1(u,v)e_1
+y_1(u,v)e_{2}+z_1(u,v)e_{3}\Big) \sqrt{A_1^2+B_1^2+C_1^2}\,du\,dv-\\
-\int\limits_{G}U\Big(x_2(u,v)e_1
+y_2(u,v)e_{2}+z_2(u,v)e_{3}\Big)
\sqrt{A_2^2+B_2^2+C_2^2}\,du\,dv\,.
\end{multline*}


Наступне твердження містить аналог
формули Гаусса--Остро\-град\-сько\-го в алгебрі $\mathbb{A}$, який випливає з
класичної формули Гаусса--Остроградського:

\begin{theorem}\label{3-3:th-Ostr-Gaus}
Нехай $\Omega$ --- однозв'язна область в $E_3$ з кусково-гладкою межею $\partial \Omega$\,. Нехай\, $\Psi :
\Omega\rightarrow\mathbb{A}$ --- неперервна функція, що має неперервні частинні похідні першого порядку в області
 $\Omega$, які неперервно продовжуються на межу $\partial \Omega$\,. Тоді справедливий наступний аналог
формули Гаусса--Остроградського:
\begin{equation}\label{3-3:form-Ostrogradsky}
\int\limits_{\partial
\Omega}\Psi(\zeta)\sigma=\int\limits_{\Omega}\left(\frac{\partial
\Psi}{\partial x}\,e_1+\frac{\partial \Psi}{\partial
y}\,e_{2}+\frac{\partial \Psi}{\partial
z}\,e_{3}\right)dx\,dy\,dz\,.
\end{equation}
\end{theorem}

\vskip 2mm

Доведення наступної теореми подібне до доведення теореми 9 з роботи
А.~Садбері \cite{Sudbery} і теореми 1 з роботи О.Ф.~Геруса \cite{Gerus-2011},
де розглядалися функції, що приймають значення в алгебрі кватерніонів.


\begin{theorem}\label{teo-Koshi-po-pov}
Нехай \,$\partial P$ --- межа замкненого куба\, $P$\,, який міститься в області\, $\Omega\subset E_3$\,, 
і функція\, $\Psi:\Omega\rightarrow\mathbb{A}$ --- гіперголоморфна в
області\, $\Omega$\,. Тоді виконується наступна рівність:
$$\int\limits_{\partial P}\Psi(\zeta)\sigma=0.$$
\end{theorem}

\begin{proof} 
Позначимо
\[K:=\biggl\|\,\int\limits_{\partial P}\Psi(\zeta)\sigma\biggr\|\,.\]
Позначимо також через\, $S$\, площу поверхні\, $\partial P$\,.
Розділимо\, $P$\, на\, $8$\, рівних кубів і позначимо через\, $P^1$\,
такий куб, для якого
\[\biggl\|\,\int\limits_{\partial P^1}\Psi(\zeta)\sigma\biggr\|\geq K/8\,.\] 
Очевидно, що поверхня\, $\partial P^1$\, має площу\,
$S/4$\,.

Продовжуючи цей процес, отримаємо послідовність вкладених кубів
$P^m$ з площами $S/4^m$ поверхонь $\partial P^m$, які задовольняють
нерівності
\begin{equation}\label{3-3:ocinka-1}
\biggr\|\,\int\limits_{\partial P^m}\Psi(\zeta)\sigma\biggr\|\ge
K/8^m.
\end{equation}

За принципом Кантора існує єдина тояка\,
$\zeta_0:=x_0e_1+y_0e_2+z_0e_3$\,, спільна для всіх кубів\, $P^m$\,.
Оскільки функція\, $\Psi$\, має вигляд (\ref{Phi-form}) і дійснозначні компоненти\, $U_k$\, 
диференційовні в\, $\Omega$, то в околі точки\, $\zeta_0$\, маємо розклад
$$\Psi(\zeta)=\Psi(\zeta_0)+\Delta x\frac{\partial
\Psi(\zeta_0)}{\partial x}+\Delta y\frac{\partial
\Psi(\zeta_0)}{\partial y}+\Delta z\frac{\partial
\Psi(\zeta_0)}{\partial z}+\delta(\zeta,\zeta_0)\rho,$$ 
де
$\Delta x:=x-x_0$\,, $\Delta y:=y-y_0$\,, $\Delta z:=z-z_0$\,,
і\, $\delta(\zeta,\zeta_0)$\, --- нескінченно мала функція при\,
$\rho:=\|\zeta-\zeta_0\|\rightarrow0$\,.

Тому для всіх достатньо малих кубів маємо
\begin{multline*}
\int\limits_{\partial
P^m}\Psi(\zeta)\sigma=\Psi(\zeta_0)\int\limits_{\partial
P^m}\sigma+\frac{\partial \Psi(\zeta_0)}{\partial
x}\int\limits_{\partial P^m}\Delta x\,\sigma+\frac{\partial
\Psi(\zeta_0)}{\partial y}\int\limits_{\partial P^m}\Delta
y\,\sigma+\\
+\frac{\partial \Psi(\zeta_0)}{\partial z}\int\limits_{
\partial P^m}\Delta z\,\sigma+\int\limits_{
\partial P^m}\delta(\zeta,\zeta_0)\rho\,\sigma=\sum_{r=1}^5J_r\,.
\end{multline*}

Згідно з формулою (\ref{3-3:form-Ostrogradsky}), $J_1=0$. Використовуючи формулу
(\ref{3-3:form-Ostrogradsky}) і враховуючи рівність (\ref{GiperGol}), отримуємо
$$J_2+J_3+J_4=\frac{\partial \Psi(\zeta_0)}{\partial
x}\,e_1V_m+\frac{\partial \Psi(\zeta_0)}{\partial
y}\,e_2V_m+\frac{\partial \Psi(\zeta_0)}{\partial z}\,e_3V_m=0,$$
де через\, $V_m$\, позначено об'єм куба\, $P^m$\,.

Відзначимо, що для довільного\, $\varepsilon>0$\, існує число\, $m_0$\, таке,
що нерівність\,
$\|\delta(\zeta,\zeta_0)\|<\varepsilon$\, виконується
для всіх кубів\, $P^m$\, при\, $m>m_0$\,. Відзначимо також, що\, $\rho$\,  не
більше, ніж діагональ куба\, $P^m$\,, тобто,\,
$\rho\leq\frac{\sqrt{S}}{2^m\sqrt{2}}$\,. Тому, використовуючи 
згадані вище  нерівності для\,
$\delta(\zeta,\zeta_0)$\, і\, $\rho$\,, отримуємо
\begin{multline}\label{ocinka-4}
\biggr\|\,\int\limits_{\partial
P^m}\Psi(\zeta)\sigma\biggr\|=\|J_5\|\leq
n\,M\int\limits_{\partial
P^m}\rho\,\|\delta(\zeta,\zeta_0)\|\,\|\sigma\|\leq\\
\le n\,M\frac{\sqrt{S}}{2^m\sqrt{2}}\,\frac{S}{4^m}\,\varepsilon\,,
\end{multline}
де $M$ --- деяка абсолютна стала.


Зі співвідношень (\ref{3-3:ocinka-1}) і
(\ref{ocinka-4}) випливає нерівність\, $K\le c\,\varepsilon $, де стала\, $c$\, 
не залежить від\, $\varepsilon$. Перейшовши до границі в останній нерівності при\, $\varepsilon\rightarrow0$\,,
отримаємо рівність\, $K=0$\,.
\end{proof}

\vskip 2mm



\subsection{Аналог теореми Коші для поверхневого інтеграла} 


Вста\-но\-ви\-мо аналог теореми Коші для поверхневого
інтеграла по межі\, $\partial\Omega$\, у випадку, коли функція\,
$\Psi:\overline{\Omega}\rightarrow\mathbb{A}$\, гіперголоморфна в області\, $\Omega$\, 
і неперервна в замиканні\, $\overline{\Omega}$\, цієї області.


Для функції $\Psi:\overline{\Omega}\rightarrow\mathbb{A}$\,,
гіперголоморфної в області\, $\Omega\subset E_3$\, і неперервної в замиканні цієї області,
розглянемо модуль неперервності 
$$\omega_{\Psi,\overline{\Omega}}(\delta):=\sup\limits_{\zeta_1,\zeta_2\in
\overline{\Omega}\,,\,\,\|\zeta_1-\zeta_2\|\leq\delta}\|\Psi(\zeta_1)-\Psi(\zeta_2)\|\,.$$

{\it Верхньою площею Мінковського}  множини $\partial\Omega$ називається величина
$$
\mathcal{M}^*(\partial\Omega):=\limsup\limits_{\varepsilon\rightarrow0}\frac
{V(\partial\Omega^\varepsilon)}{2\varepsilon}
$$
(див., наприклад, П.~Маттіла \cite[с.~79]{Mattila}),
де через $V(\partial\Omega^\varepsilon)$ позначено об'єм множини
\begin{multline*}
\partial\Omega^\varepsilon :=\{(x,y,z)\in\mathbb{R}^3:
\sqrt{(x-x_1)^2+(y-y_1)^2+(z-z_1)^2}\leq\varepsilon,\\
 \, (x_1,y_1,z_1)\in \partial\Omega\}.
\end{multline*}


\vskip 1mm

\begin{theorem}\label{Caushy-surface} 
Нехай межею $\partial\Omega$
однозв'язної області $\Omega\subset E_3$ є замкнена квадровна
поверхня, для якої $\mathcal{M}^*(\partial\Omega)<\infty$, і
$\Omega$ має вимірні за Жорданом перетини з усіма площинами, перпендикулярними до
координатних осей. Крім того, нехай функція
$\Psi:\overline{\Omega}\rightarrow\mathbb{A}$ є
гіперголоморфною в області $\Omega$ і неперервною в замиканні
$\overline{\Omega}$ цієї області. Тоді справедлива рівність 
\[\int\limits_{\partial \Omega}\Psi(\zeta)\sigma=0.\]
\end{theorem}

\vskip 1mm


\begin{proof}
Оскільки  $\mathcal{M}^*(\partial\Omega)<\infty$, то існує $\varepsilon_0>0$ таке, що для всіх
$\varepsilon\in(0,\varepsilon_0)$ виконується нерівність
\begin{equation}\label{3-3:teo2,ner-dlja-pover}
V(\partial\Omega^\varepsilon)\leq c\, \varepsilon,
\end{equation}
де стала\, $c$\, не залежить від\, $\varepsilon$.

Візьмемо\, $\varepsilon<\varepsilon_0/\sqrt{3}$\,. Розіб'ємо простір
$\mathbb{R}^3$ на куби площинами, перпендикулярними до координатних
осей, з ребром, довжина якого менша\, $\varepsilon$\,. Тоді маємо рівність
\begin{equation}\label{3-3:form-Koshi-po-pov-1}
\int\limits_{\partial\Omega}\Psi(\zeta)\,\sigma=\sum\limits_j
\int\limits_{\partial(\Omega\cap
K^j)}\Psi(\zeta)\,\sigma+\sum\limits_k \int\limits_{\partial
K^k}\Psi(\zeta)\,\sigma,
\end{equation}
де перша сума застосовується до кубів\, $K^j$\,, для яких
$\overline{K^j}\cap\partial\Omega\ne\varnothing$\,, а друга сума
застосовується до кубів\, $K^k$\,, для яких\,
$\overline{K^k}\subset\Omega$\,. За теоремою
\ref{teo-Koshi-po-pov}, друга сума
дорівнює нулю.

Для оцінки інтеграла першої суми візьмемо точку\,
$\zeta_j\in \Omega\cap K^j$\,. Зазначимо, що діаметр множини\,
$\Omega\cap K^j$\, не перевищує\, $\varepsilon\sqrt {3}$\,.
Оскільки\, $\Omega$\, має вимірні за Жорданом перетини з площинами, перпендикулярними до
осей координат, то міра Лебега меж згаданих вище перетинів дорівнює\, $0$\,,
і тому множина\, $\partial(\Omega\cap K^j)$\, складається з квадровних поверхонь. 
Отже, беручи до уваги рівність (\ref{int-sigma}), отримуємо
\begin{multline}\label{3-3:teo2_ocinka-1}
\biggr\|\,\int\limits_{\partial(\Omega\cap
K^j)}\Psi(\zeta)\sigma\biggr\|=\biggr\|\,\int\limits_{\partial(\Omega\cap
K^j)}(\Psi(\zeta)-\Psi(\zeta_j))\sigma\biggr\|\le\\
\leq nM\int\limits_{\partial(\Omega\cap
K^j)}\|\Psi(\zeta)-\Psi(\zeta_j)\|\,\|\sigma\| \leq
nM\,\omega_{\Psi,\overline{\Omega}}(\varepsilon\sqrt{3})\int\limits
_{\partial(\Omega\cap K^j)}\|\sigma\|\,,
\end{multline}
де $M$ --- деяка абсолютна стала.

Таким чином, наступна оцінка є результатом рівності
(\ref{3-3:form-Koshi-po-pov-1}) і нерівності (\ref{3-3:teo2_ocinka-1}):
\begin{multline}\label{3-3:teo2_ocinka}
\biggr\|\,\int\limits_{\partial\Omega}\Psi(\zeta)\,\sigma\biggr\|
\leq nM\,\omega_{\Psi,\overline{\Omega}}(\varepsilon\sqrt{3})\,
\sum_j\int\limits_{\partial(\Omega\cap K^j)}\|\sigma\|\le\\
\leq nM\,\omega_{\Psi,\overline{\Omega}}(\varepsilon\sqrt{3})\,\bigg(\,\int\limits_{\partial\Omega}\|\sigma\|+6\sum_j\varepsilon^2\bigg).
\end{multline}

Оскільки\,\,
$\bigcup_{j}K^j\subset\partial\Omega^{\varepsilon\sqrt{3}}$\,,  то,
враховуючи нерівність (\ref{3-3:teo2,ner-dlja-pover}), отримуємо оцінку
$$\sum_j\varepsilon^3\leq V\Big(\partial\Omega^{\varepsilon\sqrt{3}}\Big)\le
c\varepsilon\sqrt{3}$$ з якої випливає нерівність
\begin{equation}\label{3-3:epsilon_kv}
\sum_j\varepsilon^2\leq c\sqrt{3}\,.
\end{equation}

Нарешті, наступна нерівність є результатом оцінок
(\ref{3-3:teo2_ocinka}) і (\ref{3-3:epsilon_kv}):
\begin{equation*}
\biggr\|\,\int\limits_{\partial\Omega_\zeta}\Psi(\zeta)\sigma\biggr\|\leq
c_1\,\omega_{\Psi,\overline{\Omega}}(\varepsilon\sqrt{3})
\end{equation*}
де стала\, $c_1$\, не залежить від\, $\varepsilon$.

Для завершення доведення зазначимо, що\,
$\omega_{\Psi,\overline{\Omega}}(\varepsilon\sqrt{3})\to 0$\, при\,
$\varepsilon\rightarrow0$\, в силу рівномірної
неперервності функції\, $\Psi$\, на\, $\overline{\Omega}$. 
 \end{proof}

\vskip 2mm


Теорема \ref{Caushy-surface}  узагальнює теорему 1 роботи \cite{Pl-Shp3} 
доведену для трьохвимірної комутативної алгебри 
при додаткових припущеннях
про межу\, $\partial\Omega$\, і задану функцію.
Подібний аналог теореми \ref{Caushy-surface}
доведений О.Ф.~Герусом \cite{Gerus-2018} для
гіперголоморфних функцій в некомутативній алгебрі кватерніонів.




Розглянемо деякі умови, за яких виконується нерівність
(\ref{3-3:teo2,ner-dlja-pover}), що є еквівалентною умові\, $\mathcal{M}^*(\partial\Omega)<\infty$\,.

\begin{itemize}
\item 
Зазначимо, що для поверхні $\Sigma$ в $\mathbb{R}^3$
існують додатні сталі $c_1$ і $c_2$ такі, що
\begin{equation}\label{3-3:naslid-1}
c_1\varepsilon^3N_{\Sigma}(\varepsilon)\leq
V(\Sigma^\varepsilon)\leq
c_2\varepsilon^3N_{\Sigma}(\varepsilon)\,,
\end{equation}
де\, $N_{\Sigma}(\varepsilon)$ ---  найменша кількість\,
$\varepsilon$-куль, що покривають\, $\Sigma$\, (див.
Ф.M.~Бородіч і А.Ю.~Воловіков \cite{Borodich}).

Зі співвідношення (\ref{3-3:naslid-1}) випливає, що нерівність
(\ref{3-3:teo2,ner-dlja-pover}) еквівалентна нерівності вигляду
\begin{equation}\label{3-3:naslid-2}
N_{\Sigma}(\varepsilon)\,\varepsilon^2\leq c,
\end{equation}
де стала\, $c$\, не залежить від\, $\varepsilon$.

\item 
Враховуючи, що спрямлювана поверхня $\Sigma$\,  є
ліпшицевим образом квадрата\, $G$\, і нерівність вигляду
(\ref{3-3:naslid-2}) виконується для\, $G$\,,
легко довести нерівність (\ref{3-3:naslid-2}) для такої поверхні\,
$\Sigma$\,.

\item 
Якщо для поверхні\, $\Sigma$\, в\, $\mathbb{R}^3$\,, яка має
скінченну двовимірну міру Хаусдорфа\, $\mathcal{H}^2(\Sigma)$\,, 
існує додатна стала\, $c$\, така, що
\begin{equation}\label{3-3:left-reg}
c \varepsilon^2\leq\mathcal{H}^2\big(\Sigma\cap
B(x,\varepsilon)\big) \quad
\forall\,x\in\Sigma\,\,\,\forall\,\varepsilon\in(0;{\rm
diam}\,\Sigma]\,,
\end{equation}
де\, ${\rm diam}\,\Sigma$ --- діаметр поверхні\, $\Sigma$\,
і через\, $B(x,\varepsilon)$\, позначено відкриту кулю з центром\,
$x$\, і радіусом\, $\varepsilon$, то виконуються нерівності\,
$P_\Sigma(\varepsilon)\varepsilon^2\leq
c_1\mathcal{H}^2(\Sigma)<\infty$\,, де\,
$P_\Sigma(\varepsilon)$ ---
найбільша кількість неперетинних\,
$\varepsilon$-куль з центрами в\, $\Sigma$\, і стала\, $c_1$\, не залежить від\, $\varepsilon$\, (див.
Р.~Абреу Блайя, Х.~Борі Рейєс і Т.~Морено-Гарсіа
\cite[с.~309]{Abreu-Bory-Moreno-3-3}). Тому, враховуючи нерівність\, $N_{\Sigma}(2\varepsilon)\leq
P_\Sigma(\varepsilon)$\, (див. П.~Маттіла
\cite[с.~78]{Mattila}), отримуємо нерівність 
(\ref{3-3:naslid-2}) для поверхні\, $\Sigma$\,, яка задовольняє умову (\ref{3-3:left-reg}).
\end{itemize}


%%%%%%%%%%%%%%%%
\begin{remark}
Має інтерес розгляд комутативних асоціативних алгебр $\mathbb{A}_n^m$, в яких існують
лінійно незалежні над полем $\mathbb R$ елементи $e_1, e_2, e_3$, що задовольняють умову
\begin{equation}
\label{sbel}
     e_1^2+e_2^2+e_3^2=0\,.
\end{equation}
Такі алгебри називаються {\it гармонічними}, оскільки в них кожна моногенна функція $\Phi(\zeta)$ змінної
$\zeta=xe_1+ye_2+ze_3$ задовольняє тривимірне рівняння Лапласа в силу
співвідношень
\[\left(\frac{{\partial}^2}{{\partial x}^2}+
 \frac{{\partial}^2}{{\partial y}^2}+
\frac{{\partial}^2}{{\partial z}^2}\right) \Phi(\zeta)\equiv
\Phi_G''(\zeta) \, (e_1^2+e_2^2+e_3^2)=0\]
(див., наприклад, \cite{Ketchum-28,Mel'nichenko75,Mel-Plaksa,Pl-Shp1,Plaksa12})
подібно до того, як кожна голоморфна функція комплексної змінної задовольняє двовимірне рівняння Лаплаа.


Зазначимо, що у випадку лінійної оболонки $E_3$, для якої
виконуються умови \eqref{3-1:cond-on-E3}, \eqref{sbel} і\, $e_1=1$\,,
моногенні функції вказаної змінної $\zeta$ утворюють підмножину гіперголоморфних функцій. 
Дійсно, кожна моногенна функція $\Phi \colon \Omega\rightarrow\mathbb{A}_n^m$ має
$\mathbb{R}$-диференційовні компоненти $U_k$ в розкладі \eqref{3-1:rozklad-Phi-v-bazysi} в області $\Omega$, 
і в цій області виконуються умови \eqref{Umovy_K-R}, наслідком яких є рівності
\[\frac{\partial\Phi}{\partial x}\,e_1+\frac{\partial\Phi}{\partial
y}\,e_{2}+\frac{\partial\Phi}{\partial z}\,e_{3}=\frac{\partial\Phi}{\partial x}\, (e_1^2+e_2^2+e_3^2)=0\,.\]

У той же час існують гіперголоморфні функції
які не є моногенними. Наприклад, функція
$\Psi(x+ye_2+ze_3)=ze_{2}-ye_{3}$ задовольняє умову
(\ref{GiperGol}), але не задовольняє необхідні умови моногенності, а саме рівності
вигляду \eqref{Umovy_K-R}.

Аналогічно, існують гіперголомофні функції, які не задовольняють тривимірне рівняння Лапласа. Наприклад, функція
\[\Psi(xe_1+ye_2+ze_3)=x^2e_1+i\left(x^2+\frac{z^2}{2}\right)e_2+(xz+x+iy)e_3\]
задовольняє рівняння (\ref{GiperGol}), але компоненти
$x^2$ і $x^2+z^2/2$ не є просторовими гармонічними функціями.
\end{remark}






















%%%%%%%%%%%%



%---------------  bibliography


\bibliographystyle{imnanu}
\bibliography{mybiblio2}



%%%%%%%%%%%%%%%%%%%%%%%%%%%%%%%%%%%%%%%%%%%%%%%%%
%!!!!!! do not change this line - it will print the correct number of the last page
\label{last_page:\thearticlesnum}
%%%%%%%%%%%%%%%%%%%%%%%%%%%%%%%%%%%%%%%%%%%%%%%%%


\end{document}

}
