% !TeX encoding = UTF-8
\documentclass[11pt, reqno]{amsart}

\usepackage[utf8]{inputenc}
\usepackage[T2A]{fontenc}
\usepackage[english,ukrainian]{babel}

\usepackage{imnanu}  % main style file


\usepackage{cite}

\usepackage{amssymb}
\usepackage{graphicx}
\usepackage{xcolor}
\usepackage[all]{xy}
\usepackage{hyperref}
%\hypersetup{hidelinks, hypertexnames=false} %% <-- no colors for links
% \hypersetup{
% 	hypertexnames=false,
% 	colorlinks,
% 	linkcolor={red!50!black},
% 	citecolor={blue!50!black},
% 	urlcolor={blue!80!black}
% }


%% if you need to change parameters
% \SET{\Year}{2021}
% \SET{\Volume}{12}
% \SET{\Number}{5}

\begin{document}
%%%%%%%%%%%%%%%%%%%%%%%%%%%%%%%%%%%%%%%%%%%%%%%%%
%!!!!!! do not change this line - it will print the correct number of the first page
\label{first_page:\thearticlesnum}
%%%%%%%%%%%%%%%%%%%%%%%%%%%%%%%%%%%%%%%%%%%%%%%%%



%%%%%%%%%% PRINT INFORMATION ABOUT AUTHORS

%------ uncomment paper language
\selectlanguage{ukrainian}
% \selectlanguage{english}


%------ information about each author
\author{М.~І.~Портенко}
\address{Інститут математики НАН України, м.~Київ}
\email{portenko@imath.kiev.ua}
\orcid{0000-0003-1425-0628}  %% <-- ORCID of the first author if available



\title[Зародження і розвиток ідей теорії СДР]{Зародження і розвиток ідей теорії стохастичних диференціальних рівнянь\\ в українській школі математики}

%------ abstracts
\abstract{english}{
It is universally recognized that the theory of stochastic differential equations was created in the Japanese mathematical school during the forties of the last century and it is much less-known that the idea of a stochastic differential equation arose independently in the Ukrainian school of mathematics about the same time. This paper is devoted to a description of the initial stages in forming the school of the theory of stochastic differential equations and its subsequent development in Ukraine.
}

\abstract{ukrainian}{
Як зародилось поняття стохастичного диференціального рівняння в рамках української математичної школи та як проходило становлення теорії таких рівнянь в Україні --- це основні питання, що їх висвітлено в статті.
}


\keywords{стохастичні диференціальні рівняння, дифузійні процеси, рівняння Колмогорова}
\udc{519.21}
\msc{60H10, 01A60}
%% DOI of the current paper
% \doi{}

\maketitle

%--------------- parer text
\section*{Вступ}

Одним з найзначніших досягнень математики середини минулого сторіччя слід вважати зародження та розвиток ідей теорії стохастичних диференціальних рівнянь, що зрештою призвело до утворення нового розділу сучасної математики під назвою ``Стохастичний аналіз''. Окреслилися зв'язки нової науки з такими класичними розділами математики, як математичний аналіз, диференціальні рівняння, динамічні системи тощо. Визначилися сфери застосування: фізика, біологія, теорія оптимального керування системами з розподіленими параметрами, фінансова математика та ін.

Цікаво, що і до процесу зародження теорії стохастичних диференціальних рівнянь, і до її подальшого розвитку причетними виявилися науковці української школи математики. В цій статті якраз йтиметься про ті початкові етапи становлення стохастичного аналізу в Україні. Автор не ставив собі за мету дослідження цього процесу у всіх деталях. Натомість підкреслюється роль деяких сильних сторін тогочасної української  математичної школи, які, власне, і призвели до зародження ідеї стохастичного диференціального рівняння та стали потужним фактором подальшого розвитку теорії таких рівнянь. Крім того, наводяться деякі яскраві результати з теорії стохастичних диференціальних рівнянь, отримані представниками української школи теорії ймовірностей. Їх добірка відображає смаки автора цієї статті і аж ніяк не претендує на бодай якусь повноту.

Основою цієї статті стали матеріали доповіді автора на засіданні Київського математичного товариства 6 вересня 2022 року. Це засідання відбувалося в рамках так званих Гравевських читань, що традиційно проводяться в день народження Д.~О.~Ґраве, який став лідером Київської школи математики на початку XX сторіччя і був ним до своєї смерті в 1939 році. Отже, мені, як доповідачеві на цих читаннях, слід було провести лінію від Д.~О.~Ґраве до Й.~І.~Гіхмана, оскільки саме у його публікації 1947 року з'явилося поняття стохастичного диференціального рівняння. Таку лінію провести нескладно: Й.~І.~Гіхман, закінчивши Київський університет в 1939 році, одразу поступив на навчання в аспірантурі при цьому університеті, і керівником йому було призначено М.~М.~Боголюбова, який зі своїх юних літ був активним учасником семінару під керівництвом Д.~О.~Ґраве. Відштовхуючись від робіт саме свого учителя (спільних з М.~М.~Криловим) кінця 1930-х років, Й.~І.~Гіхман зробив вирішальний крок до поняття стохастичного диференціального рівняння. В низці своїх публікацій початку 1950-х років Й.~І.~Гіхман розвинув свої первісні ідеї.

В подальшому розвитку теорії стохастичних диференціальних рівнянь надзвичайно важливим був той факт, що такий титан науки як А.~В.~Скороход долучився (не без впливу Й.~І.~Гіхмана) до дослідження в цій галузі математики, починаючи з 1957 року. І вже в кінці 1961 року у видавництві Київського університету вийшла з друку його книга ``Исследовання по теории случайних процессов'', присвячена (значною мірою) теорії стохастичних диференціальних рівнянь. Низку нових ідей та результатів в цій теорії було запропоновано автором книги. Після цього спільна робота Й.~І.~Гіхмана та А.~В.~Скорохода в теорії стохастичних диференціальних рівнянь увінчалася публікацією книги ``Стохастические дифференциальнне уравнення'' в 1968-у році у видавництві ``Наукова думка'' в Києві. З'явилася ціла когорта учнів цих двох видатних математиків. Так зародилась в Україні школа теорії стохастичних диференціальних рівнянь. Саме про ті далекі тепер події ця стаття.

Слід сказати, що на початку 1940-х років японський математик К.~Іто (K.~It\^{о}) побудував теорію стохастичного інтегрування, яка дозволила йому створити теорію стохастичних диференціальних рівнянь. Як і у  Й.~І.~Гіхмана, основи теорії були сформовані на самому початку 1950-х років. На відміну від К.~Іто, Й.~І.~Гіхман не володів поняттям стохастичного інтеграла, хоч його підхід до поняття стохастичного диференціального рівняння був цілком строгим\footnote{Намагання уникати складних пояснень під час доповіді 6 вересня 2022~року привело тоді автора до цілком неадекватного опису того, чим є стохастичне диференціальне рівняння у  Й.~І.~Гіхмана. Визнаючи цю мою провину і перепрошуючи за неї перед моїми слухачами, постараюся у цій статті усунути ту прикру недбалість, що трапилася у доповіді.}. Підхід К.~Іто виявився вдалим, і тепер теорію стохастичних диференціальних рівнянь викладають, грунтуючись саме на понятті стохастичного інтеграла.

Ще одне ім'я варто згадати тут. Як показують деякі матеріали, подібні ідеї були і у франко-німецького математика В.~Дебліна (W.~D\"{o}blin). Однак його коротке життя трагічно обірвалося вже в перші роки 2-ї світової війни, а зміст його листа до академії наук Франції, в якому згадані ідеї викладені, оприлюднено порівняно недавно. Тому автор не буде тут торкатися цієї сторони справи.

Закінчу цей вступний розділ статті двома коротенькими історіями, що мають стосунок до двох
постатей в математиці, причетних до створення теорії стохастичних диференціальних рівнянь.

Наприкінці літа 1975 року в м.~Ташкенті відбувався Радянсько-Япон\-ський симпозіум з теорії
ймовірностей. К. Іто не брав участі в тому симпозіумі. В один з вечорів у неформальній обстановці
троє молодих тоді математиків з Києва --- В.~В.~Булдигін, А.~Ф.~Турбін та я --- мали нагоду поспілкуватися з молодим тоді японським математиком М.~Фукушімою (M.~Fukushima). На наше запитання: ``Кого він вважає математиком №1 в Японії?'' відповідь була однозначною і без жодних вагань: ``К. Іто!''. Ми продовжили: ``А хто на другому місці?''. Він зробив паузу, ніби щось зважуючи, і зрештою сказав: ``Немає нікого!''. Так було і з місцями 3 та~4. ``А на п’ятому місці --- багато різних!''. Таким був його вердикт.

Друга історія асоціюється із запитанням, чи знав К.~Іто про Й.~І.~Гіхмана і про його підхід до поняття стохастичного диференціального рівняння. Моя відповідь: ``Безумовно, так!''. І ось доведення цього твердження.

Це було в м. Тбілісі влітку 1982 року, де тоді проходив черговий Радянсько-Японський симпозіум з
теорії ймовірностей (до речі, наступний був у Києві в 1991 році, він завершився за тиждень до ``ГКЧП''). Цього разу К. Іто був у складі японської делегації. Одного дня в певний час всі учасники симпозіуму мали зібратися в вестибюлі готелю, де мешкали, мабуть, перед екскурсією, чи бенкетом. Сталося так, що я стояв поруч з К.~Іто. Ми встигли обмінятися парою фраз перед тим, як до нас підійшов Ілля Гіхман --- син Йосипа Ілліча. Він також був учасником симпозіуму і був зовсім ще молодим чоловіком (трохи більше 30-ти років). У нього на бейджику був напис: I.~I.~Gikhman. Правила хорошого тону зобов’язували мене представити його професору, і я це зробив. Професор нахилився до Іллі, щоб прочитати на бейджику ім’я молодого математика. Коли він випростався, на його обличчі був вираз найвищого ступеня подиву. Він не міг повірити в те, що його, так би мовити, конкурент з України у справі створення теорії стохастичних диференціальних рівнянь є такою молодою людиною (різниця у віці між К.~Іто та Й.~І.~Гіхманом складала 3 роки, перший був старшим). Ми з Іллею поспішили заспокоїти професора, пояснивши ситуацію. Цим і завершується доведення.

\textit{Подяка.} Щиро дякую професорові М.~М.~Осипчуку з Прикарпатського національного університету імені Василя Стефаника за допомогу при підготовці цієї статті до друку.


\section{Дифузійні процеси в сенсі А.~Колмогорова.}

Як відомо, процес Маркова $(x(t))_{t\geq 0}$ в евклідовому просторі $\mathbb{R}^d$ характеризується існуванням функції $P(s,x,t,\Gamma)$ аргументів $s\geq0$, $x\in\mathbb{R}^d$, $t>s$ та $\Gamma\in\mathcal{B}$ (через $\mathcal{B}$ позначається $\sigma$-алгебра всіх борельових підмножин $\mathbb{R}^d$), значеннями якої є числа з проміжку $[0,1]$ дійсної осі і яка має такі властивості:
\begin{enumerate}
	\item[а)] вона є $\mathcal{B}$-вимірною функцією аргумента $x\in\mathbb{R}^d$ при фіксованих $s<t$ та $\Gamma\in\mathcal{B}$;
	\item[б)] вона є ймовірнісною мірою по $\Gamma\in\mathcal{B}$ при фіксованих $s<t$, $x\in\mathbb{R}^d$;
	\item[в)] при всіх $s<\tau<t$, $x\in\mathbb{R}^d$ та $\Gamma\in\mathcal{B}$ виконується рівність
	$$
	P(s,x,t,\Gamma)= \int_{\mathbb{R}^d}P(\tau,y,t,\Gamma)P(s,x,\tau,\mathrm{d}y);
	$$
	\item[г)] при всіх $s<t$ та $\Gamma\in\mathcal{B}$ виконується (майже напевно) рівність
	$$\mathbb{P}\left(\{x(t)\in\Gamma\}/\mathcal{M}_{s}\right)=P(s,x(s),t,\Gamma),$$
	де ліворуч записана умовна ймовірність події $\{x(t)\in\Gamma\}$, коли фіксована $\sigma$-алгебра $\mathcal{M}_{s}$, що є найменшою $\sigma$-алгеброю подій, яка містить всі події вигляду  $\{x(r)\in\Lambda\}$ при $r\in[0,s]$ та $\Lambda\in\mathcal{B}$.
\end{enumerate}

Умова г) означає, що при складанні прогнозу майбутньої поведінки процесу (тобто, в момент часу $t$), якщо до уваги береться повна інформація про його поведінку в минулому (тобто, до теперішнього моменту часу $s$, $s\le t$), вся інформація про $x(r)$ при $r<s$ виявляється зайвою: важливою є лише та, що стосується $x(s)$. Саме ця властивість покладена в основу визначення процесу Маркова.

Отже, якщо є процес Маркова $(x(t))_{t\geq 0}$ в фазовому просторі $(\mathbb{R}^d, \mathcal{B})$, то існує функція $P(s,x,t,\Gamma)$, $0\leq s<t$, $x\in\mathbb{R}^d$, $\Gamma\in\mathcal{B}$, яка задовольняє умови  а)~--- г). Навпаки, якщо маємо функцію $P(s,x,t,\Gamma)$, що задовольняє умови а)~--- в), і додатково задана ймовірнісна міра $\left(\mu(\Gamma)\right)_{\Gamma\in\mathcal{B}}$, то існує процес Маркова $(x(t))_{t\geq 0}$ в $(\mathbb{R}^d, \mathcal{B})$, для якого $\mathbb{P}\left(\{x(0)\in\Gamma\}\right)=\mu(\Gamma)$ при всіх $\Gamma\in\mathcal{B}$ і також виконується умова г).

Таким чином, всякий розв'язок рівняння в умові в) (він має задовольняти умови а) та б), бо лише за цих умов і можна писати те рівняння) визначає процес Маркова в $\mathbb{R}^d$ і навіть не один (вони будуть різнитися лише початковими розподілами). В цьому розумінні рівняння в) є джерелом всіх процесів Маркова в $(\mathbb{R}^d, \mathcal{B})$.  Воно носить назву рівняння Колмогорова-Чепмена. Всякий розв'язок цього рівняння, що є ймовірнісною мірою по четвертій змінній, зветься ймовірністю переходу. Її інтерпретація очевидна: $P(s,x,t,\Gamma)$ при $s<t$ задає умовну ймовірність події $\{x(t)\in\Gamma\}$ за умови $x(s)=x$.

Виражаючи досить загальний принцип еволюції систем, що описуються процесами Маркова, рівняння Колмогорова-Чепмена є нелінійним. А.~Н.~Колмогоров на межі 20-х та 30-х років минулого сторіччя запропонував метод лінеаризації таких рівнянь, що ґрунтується на тих чи інших припущеннях щодо локальної поведінки процесу на малих проміжках часу (див. \cite{b1}). Він виділив декілька класів процесів Маркова, один з яких згодом дістав назву дифузійних процесів. Сформулюємо ті умови на ймовірність переходу в $(\mathbb{R}^d, \mathcal{B})$, які визначають дифузійний процес. Через $B_{r}(x)$ позначається відкрита куля в $\mathbb{R}^d$ з центром в точці $x\in\mathbb{R}^d$ радіуса $r>0$, а через $B_{r}(x)^{c}$ її доповнення до $\mathbb{R}^d$.

Нехай ймовірність переходу $P(s,x,t,\Gamma)$, $0\le s\le t$, $x\in\mathbb{R}^d$, $\Gamma\in\mathcal{B}$, задовольняє наступні умови:
\begin{enumerate}
	\item[1)] при всіх $s\geq0$, $x\in\mathbb{R}^d$ та $\varepsilon>0$ виконується співвідношення
	$$\lim_{\Delta s\downarrow 0}\frac{1}{\Delta s}\int_{B_{\varepsilon}(x)^{c}}P(s,x,s+\Delta s,\mathrm{d}y)=0;$$
	\item[2)] при всіх $s\geq0$, $x\in\mathbb{R}^d$ та деякому $\varepsilon>0$ існує границя
	$$\lim_{\Delta s\downarrow 0}\frac{1}{\Delta s}\int_{B_{\varepsilon}(x)}(y-x)P(s,x,s+\Delta s,\mathrm{d}y);$$
	\item[3)] при всіх $s\geq0$, $x\in\mathbb{R}^d$, $\theta\in\mathbb{R}^d$ та деякому $\varepsilon>0$ існує границя
	$$\lim_{\Delta s\downarrow 0}\frac{1}{\Delta s}\int_{B_{\varepsilon}(x)}(y-x,\theta)^{2}P(s,x,s+\Delta s,\mathrm{d}y).$$
\end{enumerate}

Неважко бачити, що за умови 1) існування границь в 2) та 3) при деякому $\varepsilon>0$ означає їх існування при будь-якому $\varepsilon>0$ і незалежність тих границь від $\varepsilon>0$. Тому границя в 2) визначає $\mathbb{R}^d$-значну функцію $(a(s,x))_{s\geq 0, x\in\mathbb{R}^d}$, яка зветься вектором переносу. Границя в 3) визначає операторну функцію $(b(s,x))_{s\geq 0, x\in\mathbb{R}^d}$, яка зветься оператором дифузії і з допомогою якої та границя записується у вигляді квадратичної форми $(b(s,x)\theta,\theta)$. В одновимірному випадку ці функції називаються коефіцієнтом переносу та коефіцієнтом дифузії, відповідно. Це і є локальні характеристики процесу, які описують рух на макроскопічному та мікроскопічному рівнях.

Наступний результат належить А.~Н.~Колмогорову. Припустимо, що задана ймовірність переходу в $\mathbb{R}^d$, яка задовольняє умови 1)~--- 3) з непрервними локальними характеристиками, а задана функція $(\varphi(x))_{x\in\mathbb{R}^d}$ з дійсними значеннями є неперервною обмеженою і такою, що функція ($t>0$ фіксоване)
\begin{equation}
\label{f1}
u(s,x)=\int_{\mathbb{R}^d}\varphi(y)P(s,x,t,\mathrm{d}y), \quad (s,x)\in[0,t)\times\mathbb{R}^d,
\end{equation}
двічі неперервно диференційовна по $x\in\mathbb{R}^d$.
Тоді вона диференційовна і по змінній $s\in[0,t)$ та задовольняє рівняння
\begin{equation}
\label{f2}
u'_s(s,x)+\left(a(s,x),u'_x(s,x)\right)+\frac{1}{2}\mathrm{Tr}\left(b(s,x)u''_{xx}(s,x)\right)=0
\end{equation}
в області $(s,x)\in[0,t)\times\mathbb{R}^d$, а також ``початкову'' умову
\begin{equation}
\label{f3}
u(t-,x)=\varphi(x),\quad x\in\mathbb{R}^d.
\end{equation}

У разі, якщо ймовірність переходу задовольняє умови 1)~--- 3) і має щільність відносно лебегової міри в $\mathbb{R}^d$, тобто,
\[
P(s,x,t,\Gamma)=\int_\Gamma G(s,x,t,y)\,\mathrm{d}y,\quad 0\le s<t,\ x\in\mathbb{R}^d,\ \Gamma\in\mathcal{B},
\]
 А.~Н.~Колмогоров за певних умов вивів рівняння для функції $G(s,x,t,y)$, $t\in(s,+\infty)$, $y\in\mathbb{R}^d$ при фіксованих $s\ge0$ та $x\in\mathbb{R}^d$ . Це рівняння є формально спряженим до рівняння \eqref{f2}. Воно відоме в літературі, ближчій до фізики, під назвою рівняння Фоккера-Планка. Ймовірнісники називають його прямим рівнянням Колмогорова, на відміну від рівняння \eqref{f2}, яке є оберненим.

У частинному випадку, коли $a(s,x)\equiv0$, а $b(s,x)\equiv I$ (через $I$ позначається тотожній оператор в $\mathbb{R}^d$), рівняння \eqref{f2} перетворюється на рівняння теплопровідності
\begin{equation}
\label{f4}
u'_s(s,x)+\frac{1}{2}\Delta u(s,x)=0
\end{equation}
при $(s,x)\in[0,t)\times\mathbb{R}^d$ ($t>0$ фіксоване), де оператор Лапласа $\Delta$ діє на функцію $(u(s,x))_{(s,x)\in[0,t)\times\mathbb{R}^d}$ по змінній $x\in\mathbb{R}^d$, тобто, $\Delta u(s,x)=\mathrm{Tr}\,(u''_{xx}(s,x))$. Розв'язком задачі Коші в цьому випадку буде функція
\[
u(s,x)=\int_{\mathbb{R}^d} g(s,x,t,y)\varphi(y)\,\mathrm{d}y,\quad (s,x)\in[0,t)\times\mathbb{R}^d,
\]
де $g(s,x,t,y)=(2\pi(t-s))^{-d/2}\exp\left\{-|y-x|^2/2(t-s)\right\}$ при $s<t$, $x\in\mathbb{R}^d$ та $y\in\mathbb{R}^d$. Працюючи саме з цією функцією $g$, Н.~Вінер сто років тому побудував міру в просторі неперервних функцій, яка є розподілом в тому просторі дифузійного процесу з локальними характеристиками $a(s,x)\equiv0$ та $b(s,x)\equiv I$ (див. \cite{b2}). Цей процес тепер називають вінеровим, або ж процесом броунівського руху.

Слід сказати, що згадана вище стаття А.~Н.~Колмогорова мала великий вплив на тогочасних математиків, які проводили свої дослідження в області теорії випадкових процесів. Вона вказувала на шлях, йдучи яким можна було сподіватись на побудову широкого класу дифузійних процесів. Головним на цьому шляху був аналіз задачі Коші \eqref{f2}~--- \eqref{f3}. Якщо за певних умов на коефіцієнти рівняння \eqref{f2} виявиться, що ця задача має розв'язок для широкого класу ``початкових'' функцій (настільки широкого, що кожний заряд на $\mathcal{B}$ однозначно визначається інтегралами від функцій цього класу по цьому заряду) і якщо при цьому розв'язок є невід'ємною функцією за умови, що $\varphi(x)\ge0$ при всіх $x\in\mathbb{R}^d$, тоді цей розв'язок запишеться у формі \eqref{f1} з деякою ймовірністю переходу, щодо якої залишиться лише перевірити, чи вона задовольняє умови 1)~--- 3). Якщо так, то цим і завершується побудова дифузійного процесу в сенсі Колмогорова з наперед заданими локальними характеристиками руху: вектором переносу $(a(s,x))_{s\ge0,x\in\mathbb{R}^d}$ та оператором дифузії $(b(s,x))_{s\ge0,x\in\mathbb{R}^d}$.

Першим цей шлях пройшов В.~Феллер \cite{b3}, який в одновимірному випадку, використавши метод параметрикс, а також принцип максимуму для рівнянь типу \eqref{f2}, зумів побудувати дифузійний процес в $\mathbb{R}^1$ з наперед заданими локальними характеристиками.

Метод параметрикс побудови фундаментальних розв'язків рівнянь типу \eqref{f2}, започаткований на початку минулого сторіччя, до його середини завдяки математикам різних країн і різних поколінь було розвинуто настільки, що можна було гарантувати існування фундаментальних розв'язків рівнянь \eqref{f2} за наступних умов на задані функції $(a(s,x))_{(s,x)\in[0,T]\times\mathbb{R}^d}$ та $(b(s,x))_{(s,x)\in[0,T]\times\mathbb{R}^d}$ ($T>0$ фіксоване):
\begin{enumerate}
	\item[$\alpha$)] при всіх $\theta\in\mathbb{R}^d$ виконується умова $c_1|\theta|^2\le(b(s,x)\theta,\theta)\le c_2|\theta|^2$ зі сталими $c_1>0$ та $c_2\ge c_1$, якими б не були $(s,x)\in[0,T]\times\mathbb{R}^d$;
	\item[$\beta$)] функція $(b(s,x))_{(s,x)\in[0,T]\times\mathbb{R}^d}$ задовольняє умову Гельдера $$\|b(s,x)-b(t,y)\|\le K\left(|t-s|^{\alpha/2}+|x-y|^\alpha\right)$$ зі сталими $\alpha\in(0,1]$ та $K>0$ при всіх $0\le s<t\le T$ та $x\in\mathbb{R}^d$, $y\in\mathbb{R}^d$ (тут використано операторну норму);
	\item[$\gamma$)] функція $(a(s,x))_{(s,x)\in[0,T]\times\mathbb{R}^d}$ є неперервною обмеженою і задовольняє умову $|a(s,x)-a(s,y)|\le K|x-y|^\alpha$ з тими ж сталими $K$ та $\alpha$ при всіх $s\in[0,T]$, $x\in\mathbb{R}^d$ та $y\in\mathbb{R}^d$.
\end{enumerate}

Принцип максимуму для рівнянь \eqref{f2} дозволяє гарантувати не\-від'\-єм\-ність фундаментального роз\-в'\-яз\-ку, а також забезпечує єдиність роз\-в'\-яз\-ку задачі Коші \eqref{f2}~--- \eqref{f3} в певних класах. Це в свою чергу приводило до існування дифузійного процесу з наперед заданими локальними характеристиками руху.

Таким чином, вже на середину минулого сторіччя в теорії дифузійних процесів панівними були виключно аналітичні методи побудови таких процесів та дослідження їх властивостей.

Цікаво, що до рівняння Фоккера-Планка прийшов також С.~Бернштейн, який до 1934 року працював у Харкові, перебравшись в тому році до Москви. В своїй статті \cite{b4} він запропонував розглянути певну різницеву схему, в яку було залучено задані поля $(a(s,x))_{(s,x)\in[0,T]\times\mathbb{R}^d}$ та $(\sigma(s,x))_{(s,x)\in[0,T]\times\mathbb{R}^d}$, векторне та операторне, відповідно. Нехай задана деяка послідовність $\left(t_k^{(n)}\right)_{k=0,1,\dots,k_n}$, $n=1,2,\dots$ розбиттів проміжка $[s,t]\subset[0,T]$, для якої $\lim\limits_{n\to\infty}\max\limits_{0\le k<k_n}(t_{k+1}^{(n)}-t_k^{(n)})=0$. Покладаємо для деякого $x\in\mathbb{R}^d$ $\xi_{s,x}^{(n)}(s)=x$, а при $\tau\in(t_k^{(n)}, t_{k+1}^{(n)}]$ ($k=0,1,\dots,k_n-1$)
\[
\begin{split}
\xi_{s,x}^{(n)}(\tau)=\xi_{s,x}^{(n)}(t_k^{(n)})+a\left(t_k^{(n)},\xi_{s,x}^{(n)}(t_k^{(n)})\right) \left(\tau-t_k^{(n)}\right)+\\
 \sigma\left(t_k^{(n)},\xi_{s,x}^{(n)}(t_k^{(n)})\right) \zeta_k^{(n)}\sqrt{\tau-t_k^{(n)}},
\end{split}
\]
де $\left(\zeta_k^{(n)}\right)_{k=0,1,\dots,k_n}$, $n=1,2,\dots$~--- послідовність серій незалежних в кожній серії нормальних випадкових векторів в $\mathbb{R}^d$ з нульовим середнім та одиничною коваріаційною матрицею. За певних умов на задані поля, С.~Бернштейн довів, що розподіл величини $\xi_{s,x}^{(n)}(t)$ збігається, коли $n\to\infty$, до граничного розподілу, який є абсолютно неперервним відносно лебегової міри в $\mathbb{R}^d$. Щільність того розподілу задовольняє рівняння Фоккера-Планка з коефіцієнтами $a(s,x)$ та $b(s,x)=\sigma(s,x)\sigma^*(s,x)$, $(s,x)\in[0,T]\times\mathbb{R}^d$.

Дуже важливою обставиною в цій розповіді є той факт, що в 1939 році до цього ж рівняння прийшов молодий київський математик (йому тоді було 30 років) М.~М.~Боголюбов разом зі своїм учителем М.~М.~Криловим. В своїй тогорішній публікації з промовистою назвою ``Про рівняння Фоккера-Планка, що виводиться в теорії пертурбацій методом, заснованим на спектральних властивостях пертурбаційного гамільтоніана'', вони також доводили, що певна динамічна система під впливом швидко змінних факторів, які переходять в ``білий шум'', в границі описується рівнянням Фоккера-Планка (див. \cite{b5}).

Як вже згадувалось, Й.~І.~Гіхман закінчив навчання в Київському університеті в 1939 році і, ставши в тому ж році аспірантом М.~М.~Боголюбова, змушений був підключатись до тих розмірковувань, якими в той час переймався його учитель. До війни Й.~І.~Гіхман встиг опублікувати дві статті, які були в руслі вже цитованої роботи Крилова-Боголюбова.

Й.~І.~Гіхман був учасником війни від її початку і до кінця. Цікаво, що свою кандидатську дисертацію він захистив під час війни (точніше, в лютому 1942 року) в Ташкенті. Його військова частина проходила там переформування, і він мав нагоду скористуватися тією сприятливою обставиною, що в Ташкенті на той час вже була сформована сильна ймовірнісна школа.

\section{Стохастичні диференціальні рівняння}

\subsection{Зародження ідеї}
Повернувшись після війни до Києва, Й.~І.~Гіхман  почав розмірковувати над тим, як будувати випадкові процеси типу тих, що звуться дифузійними. Підхід А.~Н.~Колмогорова~--- це намагання будувати розв'язки рівняння Колмогорова-Чепмена з допомогою лінеаризації того рівняння. Цей підхід дає змогу конструювати скінченно-вимірні розподіли процесу, а отже, і сам процес з допомогою теореми Колмогорова про узгодженість тих розподілів.

Тепер важко сказати, як саме прийшла до Й.~І.~Гіхмана надзвичайно смілива думка про те, що можна спробувати будувати не якісь ймовірнісні характеристики шуканого процесу, а безпосередньо його траекторії як розв'язки певного диференціального рівняння. Здається, що саме тут могла відіграти свою роль та обставина, що учителем Й.~І.~Гіхмана був такий корифей школи математичної фізики, як М.~М.~Боголюбов.

Ідея Й.~І.~Гіхмана полягала в тому, що має бути заданим векторне поле випадкових процесів $(\alpha(t,x))_{(t,x)\in[0,T]\times\mathbb{R}^d}$, яке має задавати локальну поведінку шуканого процесу $(x(t))_{t\in[0,T]}$ в тому розумінні, що при $0\le t<t+\Delta t\le T$ мусить виконуватись приблизна рівність
\[
x(t+\Delta t)-x(t)\cong\alpha(t+\Delta t,x(t))-\alpha(t,x(t)).
\]
Ця рівність має бути тим точнішою, чим меншим є значення $\Delta t>0$. Крім того, треба вибрати початковий момент часу $s\in[0,T)$, задати початкову умову $x(s)=\xi$ ($\xi$~--- довільна точка $\mathbb{R}^d$) і тоді можна сформулювати проблему.

Знайти умови на задане поле випадкових процесів $(\alpha(t,x))_{(t,x)\in[0,T]\times\mathbb{R}^d}$, за яких існує такий випадковий процес $(x(t))_{t\in[s,T]}$ в $\mathbb{R}^d$, що задовольняє наступні умови:
\begin{enumerate}
	\item[$A_1$)] $x(s)=\xi$;
	\item[$A_2$)] при всіх $t\in[s,T)$ виконуються рівності
	\[
	\lim_{\Delta t\downarrow0}\frac{1}{\Delta t}\mathbb{E} \left|x(t+\Delta t)-x(t)-(\alpha(t+\Delta t,x(t))-\alpha(t,x(t)))\right|^2=0,
	\]
	\[
	\lim_{\Delta t\downarrow0}\frac{1}{\Delta t^2}\mathbb{E} \left|\mathbb{E}\left[\left( x(t+\Delta t)-x(t)-(\alpha(t+\Delta t,x(t))-\alpha(t,x(t)))\right)/\mathcal{M}_t^s\right]\right|^2=0,
	\]
\end{enumerate}
де $(\mathcal{M}_t^s)_{s\le t}$ спільна історія (фільтрація), породжена усіма випадковими векторами $\alpha(\tau,x)$ при $\tau\in[s,t]$ та $x\in\mathbb{R}^d$.

Зауважимо, що друга границя в умові $A_2$) означає, що випадкові фактори приросту $\alpha(t+\Delta t,x(t))-\alpha(t,x(t))$, які могли би мати порядок $\sqrt{\Delta t}$, мусять ``зникати'', коли береться умовне середнє того приросту при фіксованому $\mathcal{M}_t^s$. Трохи згодом Й.~І.~Гіхман дійшов висновку, що задане поле випадкових процесів має бути, так би мовити, двопараметричним: $\alpha(t,x,h)$, $(t,x)\in[0,T]\times\mathbb{R}^d$, $h>0$; при фіксованих $(t,x)$ це має бути випадковий процес по змінній $h$, причому $\alpha(t,x,0+)=0$. Прикладом такого поля може бути
\begin{equation}
\label{f5}
\alpha(t,x,h)=a(t,x)h+\sigma(t,x)[w(t+h)-w(t)],
\end{equation}
де $(a(t,x))_{(t,x)\in[0,T]\times\mathbb{R}^d}$ та $(\sigma(t,x))_{(t,x)\in[0,T]\times\mathbb{R}^d}$ задані поля, відповідно векторне і операторне, а $(w(t))_{t\ge0}$ заданий вінерів процес в $\mathbb{R}^d$, тобто, найпростіший дифузійний процес, що описує рух мікроскопічної частинки в нерухомому ізотропному середовищі типу рідини чи газу. В цьому випадку друга умова в $A_2$) зводиться до такої
\[
\begin{split}
\mathbb{E} |\mathbb{E} [\left(x(t+\Delta t)-x(t)-a(t, x(t))\Delta t - \sigma(x(t))[w(t+\Delta t)- w(t)]\right)/ \mathcal{M}_{t}^{s}]|^{2}= \\\mathbb{E}|\mathbb{E} [\left(x(t+\Delta t)-x(t)-a(t, x(t))\Delta t \right)/ \mathcal{M}_{t}^{s}] |^{2}= o \left(\Delta t^2 \right)
\end{split}
\]
за умови, що випадковий вектор $w(t+\Delta t) - w(t)$ не залежить від подій із $\sigma$-алгебри $\mathcal{M}_{t}^{s}$.

Загальний результат Й.~І.~Гіхмана, сформульований в \cite{b6}, звучить так: за певних умов на поле $\alpha(t,x)$, $(t,x) \in [0,T]\times \mathbb{R}^d$, типу умови Ліпшиця по просторовій змінній, накладеної на умовні середні приростів поля, існує єдиний $(\mathcal{M}_{t}^s)$-узгоджений випадковий процес $(x_{s,\xi}(t))_{t\in[s,T]}$, який задовольняє умови $A_1$)~--- $A_2$).

В частинному випадку поля \eqref{f5} справа зводилась до простих умов Ліпшиця зі сталою $K > 0$ на функцію $\sigma$
\[
\| \sigma(t,x)- \sigma(t,y)\| \leq K |x-y|,\quad t \in [0,T],\ x \in \mathbb{R}^d,\ y \in \mathbb{R}^d,
\]
та функцію $a$
\[
(a(t,x)-a(t,y), x-y) \leq K |x-y|^2,\quad t \in [0,T],\ x \in \mathbb{R}^d,\ y \in \mathbb{R}^d
\]
(ця форма умови Ліпшиця враховує напрямок поля $a$; саме таку умову використовував С.~Бернштейн в \cite{b4}). В цьому випадку Й.~І.~Гіхман довів, що рівняння
\[
\mathrm{d}x(t) = a(t,x(t))\,\mathrm{d}t+\sigma(t,x(t))\,\mathrm{d}w(t),\quad t \in [s,T],
\]
при умові $x(s)= \xi$ ($\xi \in \mathbb{R}^d$ невипадковий вектор) має єдиний роз'язок, який є процесом Маркова. Позначимо цей роз'язок через $(x_{s,\xi}(t))_{t\in[s,T]}$. Й.~І.~Гіхман не зупинився на теоремі існування і єдиності роз'язку. За умови, що функції $a$ та $\sigma$ є тричі неперервно диференційовними по просторовій змінній, він довів, що функція $u(s,\xi) = \mathbb{E} \varphi(x_{s,\xi}(t))$, $(s,\xi)\in [0,t)\times \mathbb{R}^d$, (при заданій гладкій $(\varphi(x))_{x \in \mathbb{R}^d}$) є двічі неперервно диференційовною по змінній $\xi$, а отже, за теоремою Колмогорова вона диференційовна і по $s$, і задовольняє обернене рівняння Колмогорова
\[
u'_{s}(s,\xi )+\left(a(s,\xi ), u'_{\xi}(s,\xi)\right) + \frac{1}{2} \mathrm{Tr}\left(b \left(s, \xi \right)u''_{\xi\xi}(s,\xi)\right) = 0,
\]
в якому $b \left(s, \xi \right ) = \sigma\left(s, \xi \right) \sigma^*\left(s, \xi \right)$.

Це був надзвичайно сильний результат: те, що в теорії Колмогорова було, так би мовити, ``темним'' припущенням, у Й.~І.~Гіхмана стверджувалось. Цим відкривалося проникнення чисто ймовірнісних методів у таку класичну область математики, як диференціальні рівняння з частинними похідними параболічного типу. Справді, будується певний випадковий процес, а з його допомогою знаходиться розв’язок задачі Коші для відповідного рівняння Колмогорова. Це докорінно відрізнялось від того інструментарію, яким володіли спеціалісти з рівнянь такого типу.

В серії публікацій початку 1950 років (див. \cite{b7,b8}) Й.~І.~Гіхман розвинув ідеї статті \cite{b6}. Слід також сказати, що його заслугою перед теорією стохастичних диференціальних рівнянь є той факт, що під його впливом до серйозних занять в цій новій галузі математики долучився, починаючи з 1957 року, молодий тоді математик А.~В.~Скороход, який щойно повернувся до Києва після навчання в аспірантурі при Московському університеті протягом 1953-56 років.

\subsection{Перша книга А.В. Скорохода.}
Наприкінці 1961 року вийшла з друку монографія А.~В.~Скорохода \cite{b9}, яка була його докторською дисертацією. Значною мірою вона була присвячена теорії стохастичних диференціальних рівнянь. Слід сказати, що А.~В.~Скороход під час навчання в аспірантурі в Москві встиг ознайомитися з підходом К.~Іто до теорії стохастичних диференціальних рівнянь. Щодо підходу Й.~І.~Гіхмана до цієї теорії, він його осягнув уже в Києві під час розмов між ними, які за свідченням А.~В.~Скорохода стали регулярними після його повернення з Москви.

Тим більше вражає відвага і сміливість думки А.~В.~Скорохода, який запропонував цілком новий погляд на стохастичне диференціальне рівняння. Якщо у Й.~І.~Гіхмана і К.~Іто розв’язок такого рівняння був певним функціоналом від заданого вінерового процесу, то А.~В.~Скороход використав принцип компактності мір (в просторі неперервних функцій), що  відповідають випадковим процесам, і зумів побудувати розв’язки стохастичного диференціального рівняння за умови, що коефіцієнти є лише неперервними функціями. Саме цей революційний крок  А.~В.~Скорохода призвів до понять слабкого та сильного розв’язків. З цим пов'язаний захоплюючий період в історії розвитку теорії стохастичних диференціальних рівнянь. Повчальною в цьому сенсі є стаття \cite{b10} московських математиків А.~К.~Звонкіна та Н.~В.~Крилова: їх результат дає критерій існування сильного розв’язку даного стохастичного диференціального   рівняння в термінах його коефіцієнтів. Хоч скористуватись тим критерієм в конкретних ситуаціях непросто, проте сам факт, що питання про наявність (чи відсутність) сильного розв’язку визначається повністю коефіцієнтами, а не вдалим  (чи невдалим) вибором ймовірнісного простору, важив немало.

Другим важливим результатом книги можна назвати теорему порівняння розв’язків пари стохастичних диференціальних рівнянь, у яких збігаються коефіцієнти дифузії (в одновимірному випадку), а їх коефіцієнти переносу були пов’язані певною нерівністю. Доводилось, що якщо й початкові положення розв'язків цих рівнянь пов'язані тією ж нерівністю, то нею пов'язані ці розв'язки в будь-який момент часу після початкового. З цього факту автор книги зумів ефектно вивести єдиність розв'язку стохастичного диференціального рівняння за умов суттєво слабших, ніж умова Ліпшиця по просторовій змінній.

До книги не увійшли піонерські роботи А.~В.~Скорохода кінця {1950-х} --- початку {1960-х} років, присвячені теорії стохастичних диференціальних рівнянь в обмеженій області. Ці його дослідження стимулювали цікавість до проблеми в різних ймовірнісних центрах світу. Ця цікавість підтримується і донині, про що свідчить хоча б доповідь професора А.~Ю.~Пилипенка ``Про узагальнення проблеми відбиття Скорохода'', зроблена на засіданні семінару ``Числення Маллявена'' при Інституті математики НАН України, що відбулося 7 березня 2023 року.

\subsection{Перша книга з теорії стохастичних диференціальних рівнянь.}
В 1968 році в київському видавництві ``Наукова думка'' вийшла з друку монографія Й.~І.~Гіхмана   та А~.В.~Скорохода \cite{b11} присвячена повністю теорії стохастичних диференціальних рівнянь. В деяких книгах, опублікованих раніше, можна знайти фрагменти цієї теорії, наприклад, \cite{b9,b12,b13}.

Книга \cite{b11} складається з двох частин. В першій з них в одновимірному випадку розглядається рівняння
\[
\mathrm{d}x(t)=a(t,x(t))\,\mathrm{d}t+\sigma(t,x(t))\,\mathrm{d}w(t),
\]
в якому $a$ та $\sigma$~--- задані при $(t,x) \in [0,T]\times \mathbb{R}^1$  функції з дійсними значеннями, а $(w(t))_{t\geq 0}$ --- одновимірний вінерів процес. Викладена в цій книзі теорія таких рівнянь є надзвичайно багатою. Крім теорем існування та єдиності розв'язку, сформульовано низку тверджень про асимптотичну поведінку розв'язків, коли $t \rightarrow \infty$, детально описується взаємодія між теорією рівнянь цього типу та теорією дифузійних процесів у розумінні Колмогорова, а отже, і з теорією диференціальних рівнянь з частинними похідними. Особливу увагу приділено теорії стохастичних диференціальних рівнянь на скінченному просторовому проміжку (процеси з граничними точками).

В другій частині книги в багатовимірній ситуації розглянуто рівняння складнішого типу, зокрема, такі, коли до правої частини додаються ще стохастичні диференціали по центрованій і нецентрованій пуассоновій мірі. Сформульовано низку теорем існування та єдиності розв'язку таких рівнянь, досліджено їх асимптотичну поведінку, коли $t \rightarrow \infty$ (теореми про стійкість розв'язків, про їх обмеженість тощо). З теперішньої точки зору деякі місця цієї другої частини книги заслуговують на критику. Очевидно, що і в ті часи, коли книга з’явилася, авторам доводилось вислуховувати  критичні зауваження. Тому в 1982 р. автори публікують нову версію книги в якій намагаються врахувати тодішні досягнення світової школи теорії ймовірностей, особливо французької школи. Однак на той час така книга, здається, вже втратила актуальність.

\subsection{Розширений стохастичний інтеграл.}
Як вже зазначалось, в основі теорії стохастичних диференціальних рівнянь лежить поняття стохастичного інтеграла Іто, в якому вирішальну роль відіграє та обставина, що значення функції, яка інтегрується, в моменти часу $\tau \leq s$ не залежать від приростів вінерового процесу по якому будується інтеграл, після моменту часу $s$ (не залежать в ймовірнісному сенсі).

В 1975 році А.~В.~Скороход в статті \cite{b14} ввів поняття розширеного стохастичного інтеграла по гаусс
овій центрованій мірі від випадкових функцій з досить широкого класу. Це поняття виявилось надзвичайно цікавим з багатьох точок зору, зокрема, воно використовується під назвою ``інтеграл Скорохода'' в деяких теоріях сучасної фізики. Еволюція цього поняття протягом 30-и років після його введення в науковий обіг описана в статті \cite{b15} А.~А.~Дороговцева (одного з учнів А.~В.~Скорохода), який нині завідує відділом теорії випадкових процесів Інституту математики і є визнаним експертом в  стохастичному аналізі.

\subsection{Ю.~Л.~Далецький та його учні}
Юрію Львовичу належать такі слова: ``Неможливо не знати теорії ймовірностей, перебуваючи в одній компанії з Й.~І.~Гіхманом та А.~В.~Скороходом''. Ці слова можуть сприйматися як жарт, проте це --- саме той жарт, в якому лише частка жарту. Насправді автор цих слів був одним з перших математиків не ймовірнісного профілю, хто зрозумів всю важливість того нового, що з'явилось в роботах Й.~І.~Гіхмана та А.~В.~Скорохода, а саме, теорії стохастичних диференціальних рівнянь. І, до його честі, він зумів підхопити нові ідеї. Його результати з теорії стохастичних диференціальних рівнянь на гільбертових та банахових просторах дозволили йому та його учням по-новому підійти до дослідження еволюційних рівнянь в таких просторах. Що ж стосується теорії стохастичних диференціальних рівнянь на многовидах (як скінченої, так і нескінченної кількості вимірів), то тут Юрій Львович був одним з піонерів, а його результати, отримані спільно з учнями, стали тепер класичними в цій галузі (див. монографію \cite{b16}).

\subsection{Дві точки зору на стохастичне диференціальне рівняння}
Розглянемо стохастичне диференціальне рівняння
\begin{equation}
\label{f6}
\mathrm{d}x(t)=a(t, x(t))\,\mathrm{d}t + \sigma(t, x(t))\,\mathrm{d}w(t),
\end{equation}
в якому $(w(t))_{t\geq0}$~--- заданий вінерів процес в $\mathbb{R}^d$, а задані на множині
$(t, x)\in[0, T]\times \mathbb{R}^d$ векторне поле $a(t, x)$ та операторне поле $\sigma(t, x)$ задовольняють умови існування і єдиності розв'язку (локальна умова Ліпшиця по просторовій змінний і обмеження на можливе зростання коефіцієнтів, коли $|x|\to+\infty$: це зростання має не перевищувати зростання функції $K(1+|x|)$ зі сталою $K>0$).

Рівняння \eqref{f6} можна розглядати як результат збурення рівняння
\begin{equation}
\label{f7}
\mathrm{d}x(t)=a(t, x(t))\,\mathrm{d}t
\end{equation}
випадковим фактором, що породжується вінеровим процесом $(w(t))_{t\geq0}$. Якщо \eqref{f7}  визначає динамічну систему, то рівняння \eqref{f6} описує її поведінку під дією згаданих випадкових факторів.

Іншим поглядом на рівняння \eqref{f6} є той, згідно з яким воно є результатом збурення рівняння
\begin{equation}
\label{f8}
\mathrm{d}x(t)=\sigma(t, x(t))\mathrm{d}w(t)
\end{equation}
векторним полем $(a(t, x))_{t\geq0, x\in \mathbb{R}^d}$. Виявляється, що рівняння \eqref{f8} можна збурювати такими полями $(a(t, x))_{t\geq0, x\in \mathbb{R}^d}$, для яких рівняння \eqref{f7} не породжує жодної динамічної системи. Наприклад, таким полем може бути $a(t, x)=q(x)\delta_{S}(x)N(t, x)$, $t\geq0$, $x\in \mathbb{R}^d$, де $S$~--- задана досить гладенька гіперповерхня в $\mathbb{R}^d$, $(\delta_{S}(x))_{x\in \mathbb{R}^d}$~--- узагальнена функція, яка діє на тестову функцію $(\varphi(x))_{x\in \mathbb{R}^d}$ згідно з правилом $<\delta_S, \varphi>= \int_S\varphi(x)\,\mathrm{d}S$ (поверхневий інтеграл), $(q(x))_{x\in \mathbb{R}^d}$~--- задана неперервна функція зі значеннями в проміжку $[-1, 1]$, а $(N(t, x))_{t\geq0, x\in S}$~--- конормаль до гіперповерхні $S$, тобто, $N(t,x)=b(t,x)\nu(x)$ для $t\geq0$ та $x\in S$ (тут  $\nu(x)$~--- нормаль до $S$ в точці $x$, a $b(t,x)=\sigma(t, x)\sigma^*(t, x)$).


Ясна річ, що такий процес не може бути дифузійним в сенсі Колмогорова, проте він є таким в деякому узагальненому сенсі. Узагальнені дифузійні процеси є предметом розгляду монографії \cite{b17}, а також низки публікацій моїх колишніх учнів, а тепер~--- колег~--- О.~В.~Арясової, Б.~І.~Копитка, М.~М.~Осипчука.

\subsection{Граничні теореми}
В рамках української школи теорії стохастичних диференціальних рівнянь популярними є задачі наступного типу.


Припустимо, що рівняння \eqref{f6} має єдиний розв'язок, але для рівняння \eqref{f7} не виконується теорема єдиності розв'язку, а отже, рівняння \eqref{f7} визначає не одну динамічну систему. Припустимо, що оператор $\sigma(s, x)$, $s\in[0, T]$, $x\in \mathbb{R}^d$, в рівнянні \eqref{f6} множиться на $\varepsilon>0$, і ми цікавимось граничною поведінкою відповідного розв'язку, коли $\varepsilon\to0$. Низку надзвичайно цікавих тверджень такого типу одержано в роботах С.~Я.~Махна та його учнів, а також в роботах О.~М.~Кулика та А.~Ю.~Пилипенка разом з колегами (див. \cite{b18,b19,b20}).

\subsection{Приклад Г.~Л.~Кулініча}
Одним з учнів А.~В.~Скорохода був Г.~Л.~Кулініч. Він закінчив свій життєвий шлях 10 лютого 2022 року. Як знак пам'яті про цього прекрасного математика і вірного друга, наведу тут один з його яскравих результатів \cite{b21}, що є справжньою перлиною теорії стохастичних диференціальних рівнянь. Цим і закінчу цю мою статтю.

Розглянемо послідовність одновимірних стохастичних диференціальних рівнянь
\begin{equation}
\label{f9}
\mathrm{d}x_n(t)=a_n(x_n(t))\,\mathrm{d}t + \mathrm{d}w(t),
\end{equation}
де $(w(t))_{t\geq0}$~--- одновимірний вінерів процес, а функція $(a_n(x))_{x\in \mathbb{R}^1}$ для $n=1,2,\dots$ визначається рівністю $a_n(x)=n\cos(nx)$, $x\in \mathbb{R}^1$. Виявляється, і це~--- результат Григорія Логвиновича майже $50$-літньої давності, що послідовність випадкових процесів $(x_n(t))_{t\geq0}$ слабко збігається, коли $n\to\infty$, до процесу $(x(t))_{t\geq0}$, який можна записати в такій формі
\begin{equation}
\label{f10}
\mathrm{d}x(t)=k^{-1}\mathrm{d}w(t),
\end{equation}
де $k=\sum_{j=0}^{\infty}(j !)^{-2}$. Отже, як бачимо, граничний перехід від \eqref{f9} до \eqref{f10} приводить до того, що в границі коефіцієнт переносу зникає, а коефіцієнт дифузії зменшується. Цікаво, як реагує інтуїція читача на запитання, чому коефіцієнт дифузії мусить саме зменшуватись.


%---------------  bibliography

\bibliographystyle{plainurl}  % <-- bibliography style, do not change
\bibliography{Pmybiblio}       % <-- your BIB-file name

%%%%%%%%%%%%%%%%%%%%%%%%%%%%%%%%%%%%%%%%%%%%%%%%%
%!!!!!! do not change these lines - it will print the information about the authors
%!!!!!! and set correct number of the last page
\printArticleAuthorsInfo{\thearticlesnum}
\label{last_page:\thearticlesnum}
%%%%%%%%%%%%%%%%%%%%%%%%%%%%%%%%%%%%%%%%%%%%%%%%%


\end{document}

